\documentclass[11pt]{amsart}

\usepackage[margin=1.2in]{geometry}
\usepackage{amsmath,amssymb,amsthm,mathtools}
\usepackage{enumitem}
\usepackage[hidelinks]{hyperref}

% ---------- theorem environments ----------
\newtheorem{theorem}{Theorem}[section]
\newtheorem{proposition}[theorem]{Proposition}
\newtheorem{lemma}[theorem]{Lemma}
\newtheorem{corollary}[theorem]{Corollary}
\theoremstyle{definition}
\newtheorem{definition}[theorem]{Definition}
\newtheorem{remark}[theorem]{Remark}

% ---------- macros ----------
\newcommand{\Sn}{S_n}
\newcommand{\ZZ}{\mathbb{Z}}
\newcommand{\NN}{\mathbb{N}}
\newcommand{\R}{\mathbb{R}}
\newcommand{\Supp}{\operatorname{Supp}}
\newcommand{\abs}[1]{\left|#1\right|}
\newcommand{\bx}{\mathbf{x}}
\newcommand{\by}{\mathbf{y}}

\title[Interpolation ASEP weights at $q=1$]{Interpolation ASEP weights at $q=1$ on strict orbits:\\
a positivity obstruction and a conditional reversible Markov chain}
\author{}
\date{}

\begin{document}

\begin{abstract}
Let $\lambda=(\lambda_1>\cdots>\lambda_n\ge 0)$ be a strict partition which is \emph{restricted}
(unique part $0$ and no part $1$), and let $\Sn(\lambda)$ be the set of all rearrangements of the parts of $\lambda$.
A natural question asks whether there exists a \emph{nontrivial} Markov chain on $\Sn(\lambda)$ with stationary distribution
\[
\pi(\mu)=\frac{F^*_\mu(x_1,\dots,x_n; q=1,t)}{P^*_\lambda(x_1,\dots,x_n; q=1,t)}\qquad(\mu\in\Sn(\lambda)),
\]
where $F^*_\mu$ and $P^*_\lambda$ are the interpolation ASEP and interpolation Macdonald polynomials of Ben Dali--Williams.

We give a complete resolution once one makes explicit the (necessary) requirement that $\pi$ be a genuine probability vector.
First, we show that \emph{restrictedness does not guarantee positivity}: already for $n=2$ and $\lambda=(k,0)$ with $k\ge 2$,
the ratio $F^*_\mu/P^*_\lambda$ takes negative values for some positive parameters $(x_1,x_2,t)\in(0,\infty)^2\times(0,1)$.
Consequently, no Markov chain can realize the displayed formula \emph{uniformly over all positive parameters}.

Second, we prove a conditional existence theorem: on the (nonempty) parameter domain where all weights
$F^*_\mu(\bx;1,t)$ are strictly positive, we construct an explicit irreducible aperiodic nearest-neighbour
heat-bath walk on the adjacent-transposition graph of $\Sn(\lambda)$, reversible with respect to $\pi(\mu)\propto F^*_\mu(\bx;1,t)$.
The transition probabilities are described via signed multiline-queue partition functions (finite explicit sums of local weights),
not by invoking the polynomials $F^*_\mu$ as black boxes.
\end{abstract}

\maketitle
\vspace{-1cm}
\tableofcontents

%------------------------------------------------------------
\section{Introduction}

\subsection{The orbit and the target weights}

Fix $n\ge 2$.
A \emph{strict partition} is a vector $\lambda=(\lambda_1>\cdots>\lambda_n\ge 0)\in\ZZ_{\ge 0}^n$.
Its \emph{symmetric-group orbit} is
\[
\Sn(\lambda):=\Bigl\{\mu\in\ZZ_{\ge 0}^n:\{\mu_1,\dots,\mu_n\}=\{\lambda_1,\dots,\lambda_n\}\Bigr\},
\]
the set of all rearrangements of the parts of $\lambda$.

We assume throughout that $\lambda$ is \emph{restricted}:
\begin{enumerate}[label=(\roman*), leftmargin=2.3em]
\item $\lambda$ has a unique part of size $0$;
\item $\lambda$ has no part of size $1$.
\end{enumerate}

Ben Dali--Williams \cite{BDW} define interpolation ASEP polynomials and interpolation Macdonald polynomials.
We adopt the notation in the \emph{problem statement}:
\[
F^*_\mu(\bx;q,t)\quad(\mu\ \text{a composition}), \qquad P^*_\lambda(\bx;q,t)\quad(\lambda\ \text{a partition}),
\]
and we are interested in the specialization $q=1$.

The problem asks whether there exists a \emph{nontrivial} Markov chain on the finite state space $\Sn(\lambda)$
whose stationary distribution is
\begin{equation}\label{eq:pi-target}
\pi(\mu)=\frac{F^*_\mu(\bx;1,t)}{P^*_\lambda(\bx;1,t)}\qquad(\mu\in\Sn(\lambda)),
\end{equation}
with transition probabilities not described using the polynomials $F^*_\mu$ as primitives.

\subsection{A necessary clarification: parameters must be fixed}
A Markov chain requires \emph{numerical} transition probabilities.
Thus one must fix $(\bx,t)$ in some real parameter domain before the question makes literal probabilistic sense.
In integrable-probability applications one typically takes
\begin{equation}\label{eq:prob-domain}
x_i>0\quad(1\le i\le n),\qquad 0<t<1.
\end{equation}
In this regime $t$ is a deformation parameter and $\bx$ are inhomogeneous rates (cf.\ the inhomogeneous $t$-PushTASEP
interpretation at $q=1$ \cite{AMW}).

Even after fixing parameters, for \eqref{eq:pi-target} to be a stationary distribution of a genuine Markov chain,
it is necessary that $\pi(\mu)\ge 0$ for all $\mu$ and $\sum_\mu \pi(\mu)=1$.
The sum-to-one property holds whenever the denominator is nonzero, by the symmetrization identity
$P^*_\lambda=\sum_{\mu\in\Sn(\lambda)}F^*_\mu$ (see \S\ref{sec:BDWfacts}).
The sign constraint is the real issue.

\subsection{Main results}
Our results split into an obstruction and a conditional existence theorem.

\begin{theorem}[Positivity obstruction]\label{thm:obstruction}
In the standard probabilistic regime \eqref{eq:prob-domain}, the ratios \eqref{eq:pi-target} need not be nonnegative.
Already for $n=2$ and any restricted strict partition $\lambda=(k,0)$ with $k\ge 2$, there exist parameters
$(x_1,x_2,t)\in(0,\infty)^2\times(0,1)$ such that
\[
\frac{F^*_{(k,0)}(x_1,x_2;1,t)}{P^*_{(k,0)}(x_1,x_2;1,t)}<0.
\]
Consequently, there is no Markov chain on $\Sn(\lambda)$ whose stationary distribution equals \eqref{eq:pi-target}
\emph{for all} positive parameter choices \eqref{eq:prob-domain}.
\end{theorem}

\begin{theorem}[Conditional existence via a reversible nearest-neighbour chain]\label{thm:metropolis}
Fix $\lambda$ and fix parameters $(\bx,t)$ with $0<t<1$ such that
\begin{equation}\label{eq:positivity-assumption}
F^*_\mu(\bx;1,t)>0\qquad \text{for all }\mu\in\Sn(\lambda).
\end{equation}
Then there exists an explicit irreducible aperiodic Markov chain on $\Sn(\lambda)$ whose unique stationary distribution is
\eqref{eq:pi-target}. The chain is a nearest-neighbour heat-bath walk on the adjacent-transposition graph of $\Sn(\lambda)$.

Moreover, the transition probabilities can be written directly in terms of the signed multiline-queue partition functions of
\cite{BDW} (finite sums of explicit local weights), rather than by calling the polynomials $F^*_\mu$ as black boxes.
\end{theorem}

\begin{proposition}[Nonemptiness of the positivity domain]\label{prop:nonempty}
For each fixed $t\in(0,1)$, there exists $\bx\in(0,\infty)^n$ such that \eqref{eq:positivity-assumption} holds.
Hence Theorem~\ref{thm:metropolis} applies on a nonempty open subset of parameter space.
\end{proposition}

\begin{remark}[Resolution of the original question]
Theorem~\ref{thm:obstruction} shows that \emph{restrictedness does not ensure} the ratio \eqref{eq:pi-target}
defines a probability distribution on the full positive domain \eqref{eq:prob-domain}; hence a uniform-in-parameters Markov model
cannot exist.

On the other hand, Theorem~\ref{thm:metropolis} shows that whenever the ratios are actually nonnegative, a nontrivial Markov chain
with that stationary law exists. This is the strongest unconditional statement possible without proving a full positivity theorem
for the interpolation ASEP weights.
\end{remark}

%------------------------------------------------------------
\section{Facts from Ben Dali--Williams}\label{sec:BDWfacts}

\subsection{Notation and a caution on signs}
Ben Dali--Williams construct a \emph{signed} multiline-queue (signed MLQ) model for interpolation ASEP polynomials.
Individual signed-MLQ weights can be negative even at $q=1$ (see Example~1.13 in \cite{BDW}).
Thus, any attempt to deduce positivity term-by-term from local factors is invalid in general:
positivity, when it holds, must come from cancellations in the signed sum.

\subsection{Symmetrization identity}
Ben Dali--Williams prove that the interpolation Macdonald polynomial $P^*_\lambda$ symmetrizes the interpolation ASEP polynomials:
\begin{proposition}[Ben Dali--Williams \cite{BDW}, Prop.~2.15]\label{prop:symmetrization}
For any partition $\lambda$,
\[
P^*_\lambda(\bx;q,t)=\sum_{\mu\in\Sn(\lambda)} f^*_\mu(\bx;q,t),
\]
where $f^*_\mu$ denotes the interpolation ASEP polynomial in the notation of \cite{BDW}.
\end{proposition}

In this paper we do not distinguish $f^*_\mu$ and $F^*_\mu$, since \cite{BDW} identifies the polynomial with its signed-MLQ
weight generating function (Theorem~1.15 of \cite{BDW}); see \S\ref{sec:MLQ}.

\begin{corollary}\label{cor:sum1}
Fix $(\bx,t)$ such that $P^*_\lambda(\bx;1,t)\neq 0$.
Then the vector $\pi$ defined by \eqref{eq:pi-target} satisfies $\sum_{\mu\in\Sn(\lambda)}\pi(\mu)=1$.
\end{corollary}

\subsection{Signed-MLQ partition functions}\label{sec:MLQ}
Let $\mathrm{MLQ}^\pm(\mu)$ be the (finite) set of signed multiline queues of type $\mu$ in the sense of \cite{BDW},
and let $\mathrm{wt}(Q^\pm;\bx;q,t)$ be the weight of a signed MLQ.
Ben Dali--Williams define
\[
F^*_\mu(\bx;q,t):=\sum_{Q^\pm\in \mathrm{MLQ}^\pm(\mu)} \mathrm{wt}(Q^\pm;\bx;q,t)
\]
and show that this equals the interpolation ASEP polynomial (Theorem~1.15 in \cite{BDW}).
In particular, the values $F^*_\mu(\bx;1,t)$ can be computed from explicit local weights, without using the polynomials
as symbolic objects.

%------------------------------------------------------------
\section{The $q=1$ factorization identities}\label{sec:q1factor}

This section recalls the $q=1$ factorization formulas in \cite[\S7]{BDW}, which we use to obtain the explicit rank-two counterexample.

\subsection{Inhomogeneous elementary symmetric polynomials}
For $0\le k\le n$, define \cite[\S7]{BDW}
\begin{equation}\label{eq:estar}
e_k^*(x_1,\dots,x_n;t)
=
\sum_{S\subseteq [n],\,|S|=k}\ \prod_{i\in S}
\Bigl(x_i-\frac{t^{\#(S^c\cap [i-1])}}{t^{n-1}}\Bigr),
\end{equation}
where $[n]=\{1,2,\dots,n\}$, $S^c=[n]\setminus S$, and $[i-1]=\{1,2,\dots,i-1\}$.

\subsection{Support and partial symmetrization}
For $\mu\in\ZZ_{\ge 0}^n$, define the support
\[
\Supp(\mu):=\{i\in[n]: \mu_i>0\}.
\]

\begin{theorem}[Ben Dali--Williams \cite{BDW}, Theorem~7.1]\label{thm:BDW71}
Let $\lambda$ be a partition with $\lambda_1>0$ and let $S\subseteq[n]$ with $|S|=\ell(\lambda)$
(the number of positive parts of $\lambda$). Then at $q=1$,
\begin{equation}\label{eq:BDW50}
\sum_{\mu\in\Sn(\lambda):\,\Supp(\mu)=S} F^*_\mu(\bx;1,t)
=
\prod_{i\in S}\Bigl(x_i-\frac{t^{\#(S^c\cap[i-1])}}{t^{n-1}}\Bigr)\cdot
\prod_{2\le j\le \lambda_1} e^*_{\lambda'_j}(\bx;t),
\end{equation}
where $\lambda'$ denotes the conjugate partition.
\end{theorem}

Summing \eqref{eq:BDW50} over all supports yields the factorization of $P^*_\lambda(\bx;1,t)$.

\begin{corollary}[Ben Dali--Williams \cite{BDW}, Eq.~(51)]\label{cor:BDW51}
At $q=1$,
\begin{equation}\label{eq:BDW51}
P^*_\lambda(\bx;1,t)=\prod_{j=1}^{\lambda_1} e^*_{\lambda'_j}(\bx;t).
\end{equation}
\end{corollary}

%------------------------------------------------------------
\section{A rank-two counterexample: proof of Theorem~\ref{thm:obstruction}}\label{sec:rank2}

Fix $n=2$ and $\lambda=(k,0)$ with $k\ge 2$.
Then $\lambda$ is strict and restricted, and
\[
\Sn(\lambda)=\{(k,0),(0,k)\}.
\]

\subsection{Explicit formulas}
We first compute $e_1^*$ from \eqref{eq:estar}.
For $n=2$ and $k=1$, there are two subsets $S=\{1\}$ and $S=\{2\}$:
\begin{equation}\label{eq:e1star}
e_1^*(x_1,x_2;t)=(x_1-t^{-1})+(x_2-1).
\end{equation}

The conjugate partition of $(k)$ is $(1^k)$, hence $\lambda'_j=1$ for $1\le j\le k$.
By \eqref{eq:BDW51},
\begin{equation}\label{eq:Pstar-k0}
P^*_{(k,0)}(x_1,x_2;1,t)=\prod_{j=1}^k e_1^*(x_1,x_2;t) = \bigl(e_1^*(x_1,x_2;t)\bigr)^k.
\end{equation}

Now apply Theorem~\ref{thm:BDW71} with supports $S=\{1\}$ and $S=\{2\}$.
Since each support class contains a single orbit element, the sum in \eqref{eq:BDW50} has one term:
\begin{align}
F^*_{(k,0)}(x_1,x_2;1,t)
&=(x_1-t^{-1})\cdot\bigl(e_1^*(x_1,x_2;t)\bigr)^{k-1},\label{eq:Fk0}\\
F^*_{(0,k)}(x_1,x_2;1,t)
&=(x_2-1)\cdot\bigl(e_1^*(x_1,x_2;t)\bigr)^{k-1}.\label{eq:F0k}
\end{align}

\subsection{The normalized weights and a negativity region}
Combining \eqref{eq:Pstar-k0}, \eqref{eq:Fk0}, \eqref{eq:F0k}, and \eqref{eq:e1star}, we get
\begin{equation}\label{eq:pi-explicit}
\pi(k,0)=\frac{x_1-t^{-1}}{(x_1-t^{-1})+(x_2-1)},\qquad
\pi(0,k)=\frac{x_2-1}{(x_1-t^{-1})+(x_2-1)}.
\end{equation}
(These formulas are independent of $k\ge 2$.)

\begin{lemma}\label{lem:neg}
Fix $t\in(0,1)$.
If $0<x_1<t^{-1}$ and $x_2>1$ is chosen so that $(x_2-1)>(t^{-1}-x_1)$, then
the denominator in \eqref{eq:pi-explicit} is positive and $\pi(k,0)<0$.
\end{lemma}

\begin{proof}
Under the hypothesis, $x_1-t^{-1}<0$ but $(x_1-t^{-1})+(x_2-1)>0$.
\end{proof}

\begin{proof}[Proof of Theorem~\ref{thm:obstruction}]
Take for instance $t=\tfrac12$, $x_1=1$, $x_2=10$.
Then $t^{-1}=2$ and \eqref{eq:pi-explicit} gives
\[
\pi(k,0)=\frac{1-2}{(1-2)+(10-1)}=\frac{-1}{8}<0.
\]
Hence \eqref{eq:pi-target} fails to define a probability distribution at this positive parameter point, so it cannot be the stationary
distribution of any genuine Markov chain at this point. In particular, there is no Markov chain whose stationary distribution equals
\eqref{eq:pi-target} uniformly over all positive parameters \eqref{eq:prob-domain}.
\end{proof}

\begin{remark}[The positivity region when $n=2$]
From \eqref{eq:pi-explicit}, the normalized vector $\pi$ is a genuine probability distribution precisely on the region where both
weights are nonnegative and the denominator is nonzero, i.e.
either $x_1\ge t^{-1}$ and $x_2\ge 1$ (with not both equalities), or $x_1\le t^{-1}$ and $x_2\le 1$ (again excluding the zero denominator).
This illustrates that positivity is parameter-dependent even in the smallest restricted case.
\end{remark}

%------------------------------------------------------------
\section{A conditional Markov chain on the positivity domain}\label{sec:metropolis}

We now prove Theorem~\ref{thm:metropolis} and Proposition~\ref{prop:nonempty}.
The construction is standard in Markov chain Monte Carlo: a reversible heat-bath (Metropolis-type) walk on a connected graph,
with acceptance probabilities determined by the target weights.

\subsection{A nonempty positivity domain}

Fix $t\in(0,1)$ and define the positivity domain
\[
\mathcal{D}_\lambda(t):=\Bigl\{\bx\in(0,\infty)^n:\ F^*_\mu(\bx;1,t)>0\ \text{for all }\mu\in\Sn(\lambda)\Bigr\}.
\]

\begin{lemma}\label{lem:top-hom}
For each $\mu$, the polynomial $F^*_\mu(\bx;1,t)$ has total degree $|\mu|$, and its top homogeneous component is the (usual) ASEP polynomial
$f_\mu(\bx;1,t)$ (in the notation of \cite{BDW} and \cite{AMW}).
\end{lemma}

\begin{proof}
This is stated in \cite[Def.~1.2]{BDW}: the interpolation ASEP polynomial is defined as $f^*_\mu=T_{\sigma_\mu}\cdot E^*_\lambda$,
and its top homogeneous part is $f_\mu=T_{\sigma_\mu}\cdot E_\lambda$; moreover $\deg f^*_\mu=|\mu|$.
\end{proof}

\begin{lemma}\label{lem:nonempty}
Fix $t\in(0,1)$. Then $\mathcal{D}_\lambda(t)\neq\varnothing$.
\end{lemma}

\begin{proof}
Fix any $\by\in(0,\infty)^n$. For each $\mu\in\Sn(\lambda)$, write
\[
F^*_\mu(R\by;1,t)=R^{|\mu|} f_\mu(\by;1,t) + O\bigl(R^{|\mu|-1}\bigr)\qquad(R\to\infty),
\]
because the top homogeneous part is $f_\mu$ by Lemma~\ref{lem:top-hom}.

By \cite{AMW}, at $q=1$ the quantities $f_\mu(\by;1,t)$ appear as stationary weights of an inhomogeneous $t$-PushTASEP
and are therefore strictly positive for $\by>0$ and $0<t<1$.
Thus for each fixed $\mu$ there exists $R_\mu$ such that $F^*_\mu(R\by;1,t)>0$ for all $R\ge R_\mu$.
Since $\Sn(\lambda)$ is finite, taking $R\ge \max_\mu R_\mu$ yields $R\by\in\mathcal{D}_\lambda(t)$.
\end{proof}

\begin{proof}[Proof of Proposition~\ref{prop:nonempty}]
Take any $t\in(0,1)$ and apply Lemma~\ref{lem:nonempty}.
Openness follows from continuity of the finitely many polynomials $F^*_\mu(\bx;1,t)$.
\end{proof}

\subsection{The adjacent-transposition graph}
Because $\lambda$ has distinct parts, each $\mu\in\Sn(\lambda)$ determines a unique permutation $\sigma\in\Sn$ such that
$\mu=\sigma\cdot\lambda$. Adjacent transpositions generate $\Sn$, hence they connect $\Sn(\lambda)$.

Let $s_i$ be the adjacent transposition swapping coordinates $i$ and $i+1$, acting on $\mu$ by $(s_i\mu)_i=\mu_{i+1}$,
$(s_i\mu)_{i+1}=\mu_i$, other coordinates fixed. The \emph{adjacent-transposition graph} on $\Sn(\lambda)$ has an (undirected) edge
between $\mu$ and $s_i\mu$ for each $i\in\{1,\dots,n-1\}$.

\subsection{The heat-bath chain}
Assume \eqref{eq:positivity-assumption} and define positive weights
\begin{equation}\label{eq:Wdef}
W(\mu):=F^*_\mu(\bx;1,t)\qquad(\mu\in\Sn(\lambda)).
\end{equation}

\begin{definition}[Nearest-neighbour heat-bath chain]\label{def:HB}
Define a Markov chain $(X_m)_{m\ge 0}$ on $\Sn(\lambda)$ as follows.
Given $X_m=\mu$:
\begin{enumerate}[label=\textup{(\arabic*)}, leftmargin=2.3em]
\item Choose $i$ uniformly from $\{1,2,\dots,n-1\}$.
\item Let $\nu=s_i\mu$.
\item Set
\[
X_{m+1}=\begin{cases}
\nu, & \text{with probability }\displaystyle \frac{W(\nu)}{W(\mu)+W(\nu)},\\[1.25ex]
\mu, & \text{with probability }\displaystyle \frac{W(\mu)}{W(\mu)+W(\nu)}.
\end{cases}
\]
\end{enumerate}
\end{definition}

\begin{remark}[Nontriviality and ``not using $F^*_\mu$'']
The chain is nontrivial: it moves along adjacent-transposition edges with positive probability as long as $W$ is not constant.

Moreover, by \S\ref{sec:MLQ} the values $W(\mu)$ can be written as explicit finite signed sums of local signed-MLQ weights
(Definition~1.14 and Theorem~1.15 of \cite{BDW}), rather than by invoking $F^*_\mu$ as a symbolic polynomial.
Thus the transition probabilities are expressible entirely in terms of the signed-MLQ partition functions.
\end{remark}

\subsection{Irreducibility and aperiodicity}
\begin{lemma}\label{lem:irreducible}
Under \eqref{eq:positivity-assumption}, the chain in Definition~\ref{def:HB} is irreducible and aperiodic.
\end{lemma}

\begin{proof}
If $\nu=s_i\mu$, then $P(\mu\to\nu)=\frac{1}{n-1}\frac{W(\nu)}{W(\mu)+W(\nu)}>0$ since $W(\cdot)>0$.
The adjacent-transposition graph on $\Sn(\lambda)$ is connected, hence the chain is irreducible.

Also, for every state $\mu$ and any choice of $i$, there is a self-loop probability
$\frac{1}{n-1}\frac{W(\mu)}{W(\mu)+W(s_i\mu)}>0$, hence the period is $1$.
\end{proof}

\subsection{Stationarity via detailed balance}
Define
\begin{equation}\label{eq:piW}
\pi(\mu):=\frac{W(\mu)}{\sum_{\eta\in\Sn(\lambda)}W(\eta)}.
\end{equation}

\begin{proposition}[Detailed balance]\label{prop:DB}
The measure $\pi$ in \eqref{eq:piW} satisfies detailed balance for the chain in Definition~\ref{def:HB}.
In particular, $\pi$ is stationary.
\end{proposition}

\begin{proof}
Fix an edge $\mu\leftrightarrow\nu$ with $\nu=s_i\mu$.
Then
\[
P(\mu\to\nu)=\frac{1}{n-1}\frac{W(\nu)}{W(\mu)+W(\nu)},\qquad
P(\nu\to\mu)=\frac{1}{n-1}\frac{W(\mu)}{W(\mu)+W(\nu)}.
\]
Thus
\[
\pi(\mu)P(\mu\to\nu)
=\frac{W(\mu)}{\sum_\eta W(\eta)}\cdot \frac{1}{n-1}\frac{W(\nu)}{W(\mu)+W(\nu)}
=\pi(\nu)P(\nu\to\mu).
\]
All other off-diagonal transitions are $0$, so detailed balance holds for all pairs.
\end{proof}

\begin{corollary}[Identification with the ratio $F^*/P^*$]\label{cor:identify}
Under \eqref{eq:positivity-assumption}, the unique stationary distribution of Definition~\ref{def:HB} equals
\[
\pi(\mu)=\frac{F^*_\mu(\bx;1,t)}{P^*_\lambda(\bx;1,t)}\qquad(\mu\in\Sn(\lambda)).
\]
\end{corollary}

\begin{proof}
By Proposition~\ref{prop:DB} and Lemma~\ref{lem:irreducible}, the chain has a unique stationary distribution, given by \eqref{eq:piW}.
Using $W(\mu)=F^*_\mu(\bx;1,t)$ and Proposition~\ref{prop:symmetrization} at $q=1$,
\[
\sum_{\eta\in\Sn(\lambda)}W(\eta)=\sum_{\eta\in\Sn(\lambda)}F^*_\eta(\bx;1,t)=P^*_\lambda(\bx;1,t),
\]
so \eqref{eq:piW} matches \eqref{eq:pi-target}.
\end{proof}

\begin{proof}[Proof of Theorem~\ref{thm:metropolis}]
Combine Lemma~\ref{lem:irreducible}, Proposition~\ref{prop:DB}, and Corollary~\ref{cor:identify}.
\end{proof}

%------------------------------------------------------------
\section{Discussion and open directions}

\begin{remark}[What this does and does not solve]
Theorems~\ref{thm:obstruction} and \ref{thm:metropolis} together give a sharp ``if and only if'' at the level of basic feasibility:
a Markov chain with stationary law \eqref{eq:pi-target} exists precisely at those parameter points where the right-hand side defines
a probability distribution (equivalently, where $F^*_\mu(\bx;1,t)\ge 0$ for all $\mu$ and not all are zero).

What remains open---and is closer to the integrable-probability spirit of the $t$-PushTASEP correspondence \cite{AMW}---is to construct
a \emph{natural local interacting-particle dynamics} whose stationary law is \eqref{eq:pi-target} on a large parameter domain,
without resorting to generic Metropolis/heat-bath machinery.
\end{remark}

\begin{remark}[Why the ``restricted'' hypothesis does not force positivity]
Restrictedness (unique $0$ and no $1$) is useful in certain $q=1$ normalization identities in \cite{BDW}, which are relevant to attempts
to build stochastic kernels from queue recursions.
However, Theorem~\ref{thm:obstruction} shows that restrictedness alone does not prevent sign changes in the normalized weights.
\end{remark}

%------------------------------------------------------------
\begin{thebibliography}{99}

\bibitem{AMW}
A.~Ayyer, J.~B.~Martin, and L.~K.~Williams,
\emph{The inhomogeneous $t$-PushTASEP and Macdonald polynomials},
arXiv:2403.10485 (2024).

\bibitem{BDW}
H.~Ben Dali and L.~K.~Williams,
\emph{A combinatorial formula for interpolation Macdonald polynomials},
arXiv:2510.02587 (2025).

\end{thebibliography}

\end{document}
