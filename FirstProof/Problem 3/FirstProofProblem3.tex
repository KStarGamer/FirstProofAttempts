\documentclass[11pt]{amsart}

\usepackage[a4paper,margin=1in]{geometry}
\usepackage{amsmath,amssymb,amsthm,mathtools}
\usepackage{enumitem}
\usepackage[hidelinks]{hyperref}
\usepackage{microtype}

% ---------- theorem environments ----------
\newtheorem{theorem}{Theorem}[section]
\newtheorem{proposition}[theorem]{Proposition}
\newtheorem{lemma}[theorem]{Lemma}
\newtheorem{corollary}[theorem]{Corollary}
\theoremstyle{definition}
\newtheorem{definition}[theorem]{Definition}
\theoremstyle{remark}
\newtheorem{remark}[theorem]{Remark}

% ---------- macros ----------
\newcommand{\Sn}{S_n}
\newcommand{\NN}{\mathbb{N}}
\newcommand{\RR}{\mathbb{R}}
\newcommand{\bx}{\mathbf{x}}
\newcommand{\abs}[1]{\left|#1\right|}
\newcommand{\supp}{\operatorname{Supp}}
\newcommand{\ellpart}{\ell}

\title[No stochastic Markov chain for $q=1$ interpolation ASEP weights]{A sign obstruction for $q=1$ interpolation ASEP weights on strict restricted orbits}
\date{\today}

\begin{document}

\begin{abstract}
Let $\lambda=(\lambda_1>\cdots>\lambda_n\ge 0)$ be a strict partition with a unique part of size $0$ and no part of size $1$
(``restricted'' in the sense of the problem statement), and let $\Sn(\lambda)$ be the set of rearrangements of the parts of $\lambda$.
Ben Dali--Williams defined the interpolation ASEP polynomials $F^*_\mu(\bx;q,t)$ (nonsymmetric; indexed by compositions $\mu$) and
the interpolation Macdonald polynomials $P^*_\lambda(\bx;q,t)$ (symmetric; indexed by partitions $\lambda$).
A natural question asks whether there exists a \emph{nontrivial} Markov chain on $\Sn(\lambda)$ whose stationary distribution is
\[
\pi(\mu)\;=\;\frac{F^*_\mu(\bx;1,t)}{P^*_\lambda(\bx;1,t)}\qquad(\mu\in\Sn(\lambda)),
\]
with transition probabilities described without using the polynomials $F^*_\mu$ as primitives.

In the standard probabilistic parameter domain $x_i>0$ and $0<t<1$, we prove an unconditional negative answer:
for every restricted $\lambda$ and every $t\in(0,1)$, there exist positive parameters $\bx\in(0,\infty)^n$ such that
$P^*_\lambda(\bx;1,t)>0$ but the ratio vector has a negative entry (indeed, a certain ``zero-position'' marginal is negative).
Hence it cannot be the stationary distribution of any genuine Markov chain at those parameters, and therefore no Markov chain can
realize the displayed formula uniformly over the positive parameter domain.
The proof uses only the $q=1$ factorization and partial symmetrization identities of Ben Dali--Williams.
\end{abstract}

\maketitle

\tableofcontents

%------------------------------------------------------------
\section{Introduction}

\subsection{The orbit and the restricted hypothesis}

Fix $n\ge 2$ and let
\[
\lambda=(\lambda_1>\lambda_2>\cdots>\lambda_n\ge 0)
\]
be a strict partition. Its symmetric-group orbit is the finite set
\[
\Sn(\lambda):=\Bigl\{\mu\in\NN^n:\{\mu_1,\dots,\mu_n\}=\{\lambda_1,\dots,\lambda_n\}\Bigr\}.
\]
We assume throughout that $\lambda$ is \emph{restricted} in the sense of the problem statement:
\begin{enumerate}[label=(\roman*), leftmargin=2.3em]
\item $\lambda$ has a unique part equal to $0$;
\item $\lambda$ has no part equal to $1$.
\end{enumerate}

\subsection{The target ratios and the meaning of ``Markov chain''}

Ben Dali--Williams \cite{BDW} define interpolation ASEP polynomials and interpolation Macdonald polynomials.
They denote the interpolation ASEP polynomials by $f^*_\mu(\bx;q,t)$; in this paper we adopt the notation of the problem statement
and write
\[
F^*_\mu(\bx;q,t):=f^*_\mu(\bx;q,t),
\qquad
P^*_\lambda(\bx;q,t)
\]
for the interpolation Macdonald polynomial. We are interested in the specialization $q=1$.

A discrete-time Markov chain on a finite state space $\Omega$ is a stochastic matrix $K=(K(\omega,\omega'))_{\omega,\omega'\in\Omega}$
with $K(\omega,\omega')\ge 0$ and $\sum_{\omega'}K(\omega,\omega')=1$ for all $\omega$.
A stationary distribution is a probability vector $\pi$ with $\pi(\omega)\ge 0$, $\sum_\omega \pi(\omega)=1$, and $\pi K=\pi$.
Likewise, for a continuous-time Markov chain with generator $Q$, a stationary distribution is a probability vector $\pi$ solving $\pi Q=0$.
In either case, \emph{a stationary distribution must be entrywise nonnegative.}

The problem asks whether there exists a \emph{nontrivial} Markov chain on $\Sn(\lambda)$ whose stationary distribution is
\begin{equation}\label{eq:pi-def}
\pi(\mu)=\frac{F^*_\mu(\bx;1,t)}{P^*_\lambda(\bx;1,t)}\qquad(\mu\in\Sn(\lambda)),
\end{equation}
with the additional constraint that the transition probabilities should not be described using the polynomials $F^*_\mu$ themselves.

To interpret \eqref{eq:pi-def} probabilistically one must fix numerical parameters $(\bx,t)$.
In integrable probability, one typically works in the positive domain
\begin{equation}\label{eq:positive-domain}
x_i>0\ (1\le i\le n),\qquad 0<t<1.
\end{equation}
Our main result shows that restrictedness of $\lambda$ does \emph{not} force the ratios \eqref{eq:pi-def} to be nonnegative on this domain.

\subsection{Main result}

\begin{theorem}[Universal sign obstruction]\label{thm:main}
Let $n\ge 2$ and let $\lambda=(\lambda_1>\cdots>\lambda_n\ge 0)$ be restricted (unique $0$, no part $1$).
Fix any $t\in(0,1)$. Then there exists $\bx\in(0,\infty)^n$ such that:
\begin{enumerate}[label=\textup{(\alph*)}, leftmargin=2.3em]
\item $P^*_\lambda(\bx;1,t)>0$ (so \eqref{eq:pi-def} is well-defined);
\item there exists $\mu\in\Sn(\lambda)$ with $\displaystyle \frac{F^*_\mu(\bx;1,t)}{P^*_\lambda(\bx;1,t)}<0$.
\end{enumerate}
Consequently, there is no (discrete- or continuous-time) Markov chain on $\Sn(\lambda)$ whose stationary distribution equals
\eqref{eq:pi-def} \emph{uniformly for all} positive parameter choices \eqref{eq:positive-domain}.
\end{theorem}

\begin{remark}[Resolution of the problem]
Under the standard probabilistic interpretation (transition probabilities are nonnegative reals), Theorem~\ref{thm:main} gives a complete
negative answer: the prescribed ratios are not always probabilities even for restricted $\lambda$, so they cannot be stationary laws of genuine
Markov chains on the full positive parameter domain. This resolves the existence question as stated.
\end{remark}

%------------------------------------------------------------
\section{Ben Dali--Williams identities at $q=1$}\label{sec:BDW}

We recall the $q=1$ identities from \cite{BDW} that we need: a symmetrization identity (Proposition~2.15 in \cite{BDW}),
and a partial symmetrization plus factorization identity (Theorem~7.1 in \cite{BDW}).

\subsection{Support}
For $\mu=(\mu_1,\dots,\mu_n)\in\NN^n$ define the support
\[
\supp(\mu):=\{i\in[n]: \mu_i>0\},\qquad [n]:=\{1,2,\dots,n\}.
\]

\subsection{Symmetrization}
The interpolation Macdonald polynomial $P^*_\lambda$ symmetrizes the interpolation ASEP polynomials $F^*_\mu$.

\begin{proposition}\label{prop:sym}
For any partition $\lambda$,
\begin{equation}\label{eq:symmetrization}
P^*_\lambda(\bx;q,t)=\sum_{\mu\in\Sn(\lambda)} F^*_\mu(\bx;q,t).
\end{equation}
\end{proposition}

\begin{proof}
This is Proposition~2.15 in \cite{BDW}. For completeness: Ben Dali--Williams show that the sum on the right-hand side is symmetric and
satisfies the vanishing conditions that characterize $P^*_\lambda$.
\end{proof}

In particular, whenever $P^*_\lambda(\bx;1,t)\neq 0$, the ratios \eqref{eq:pi-def} sum to $1$.

\begin{corollary}\label{cor:sum1}
If $P^*_\lambda(\bx;1,t)\neq 0$, then $\sum_{\mu\in\Sn(\lambda)} \pi(\mu)=1$ for $\pi$ defined by \eqref{eq:pi-def}.
\end{corollary}

\begin{proof}
Divide \eqref{eq:symmetrization} (with $q=1$) by $P^*_\lambda(\bx;1,t)$.
\end{proof}

\subsection{Inhomogeneous elementary symmetric functions at $q=1$}

Fix $n\ge 1$. For each $k\ge 0$, define
\begin{equation}\label{eq:estar}
e_k^*(x_1,\dots,x_n;t)
:=
\sum_{\substack{S\subseteq[n]\\ |S|=k}}
\ \prod_{i\in S}
\left(x_i-\frac{t^{\#(S^c\cap [i-1])}}{t^{n-1}}\right),
\end{equation}
where $S^c=[n]\setminus S$ and $[i-1]=\{1,2,\dots,i-1\}$.
This is the definition in Section~7 of \cite{BDW}. (Ben Dali--Williams write $e_k^\ast$ for this function.)

\subsection{Partial symmetrization and factorization at $q=1$}

Let $\lambda'$ denote the conjugate partition of $\lambda$, and $\ellpart(\lambda)$ the number of positive parts of $\lambda$.

\begin{theorem}\label{thm:BDW71}
Let $\lambda$ be a partition with $\lambda_1>0$.
For any subset $S\subseteq[n]$ with $|S|=\ellpart(\lambda)$,
\begin{equation}\label{eq:BDW50}
\sum_{\mu\in\Sn(\lambda):\,\supp(\mu)=S} F^*_\mu(\bx;1,t)
=
\left(\prod_{i\in S}\left(x_i-\frac{t^{\#(S^c\cap [i-1])}}{t^{n-1}}\right)\right)
\cdot \prod_{2\le j\le \lambda_1} e^*_{\lambda'_j}(\bx;t).
\end{equation}
Consequently,
\begin{equation}\label{eq:BDW51}
P^*_\lambda(\bx;1,t)=\prod_{1\le j\le \lambda_1} e^*_{\lambda'_j}(\bx;t).
\end{equation}
\end{theorem}

\begin{proof}
This is Theorem~7.1 in \cite{BDW}. The identity \eqref{eq:BDW50} is equation (50) there, and \eqref{eq:BDW51} is equation (51) there.
\end{proof}

%------------------------------------------------------------
\section{A ``zero-position'' marginal formula for restricted $\lambda$}\label{sec:marginal}

The key point is that when $\lambda$ has a unique $0$, support is equivalent to specifying the position of that zero.
Theorem~\ref{thm:BDW71} then yields an explicit formula for the total weight of orbit elements with the zero at a given position.

\subsection{Restrictedness forces $\ellpart(\lambda)=n-1$}

\begin{lemma}\label{lem:length}
If $\lambda=(\lambda_1>\cdots>\lambda_n\ge 0)$ has a unique part equal to $0$, then $\ellpart(\lambda)=n-1$, hence $\lambda'_1=n-1$.
\end{lemma}

\begin{proof}
Exactly $n-1$ parts of $\lambda$ are positive. The first part of the conjugate partition equals the number of positive parts, so $\lambda'_1=n-1$.
\end{proof}

\subsection{Explicit support factors}

For each $i\in[n]$, define the $(n-1)$-element subset
\[
S_i:=[n]\setminus\{i\}.
\]
For a restricted $\lambda$, each $\mu\in\Sn(\lambda)$ has exactly one zero coordinate, and $\supp(\mu)=S_i$ if and only if $\mu_i=0$.

Define
\begin{equation}\label{eq:Ai-def}
A_i(\bx;t):=
\left(\prod_{j=1}^{i-1}\bigl(x_j-t^{-(n-1)}\bigr)\right)\cdot
\left(\prod_{j=i+1}^{n}\bigl(x_j-t^{-(n-2)}\bigr)\right),
\qquad (1\le i\le n),
\end{equation}
with the convention that an empty product is $1$.

\begin{lemma}\label{lem:support-sum}
Let $\lambda$ be restricted. Then for each $i\in[n]$,
\begin{equation}\label{eq:support-sum}
\sum_{\mu\in\Sn(\lambda):\,\mu_i=0} F^*_\mu(\bx;1,t)
=
A_i(\bx;t)\cdot \prod_{2\le j\le \lambda_1} e^*_{\lambda'_j}(\bx;t).
\end{equation}
\end{lemma}

\begin{proof}
By Lemma~\ref{lem:length}, $\ellpart(\lambda)=n-1$, so we may apply Theorem~\ref{thm:BDW71} with $S=S_i$.

On the left-hand side of \eqref{eq:BDW50}, $\supp(\mu)=S_i$ is equivalent to $\mu_i=0$.
On the right-hand side of \eqref{eq:BDW50}, we compute the product
\[
\prod_{j\in S_i}\left(x_j-\frac{t^{\#(S_i^c\cap [j-1])}}{t^{n-1}}\right),
\qquad S_i^c=\{i\}.
\]
If $j<i$ then $i\notin[j-1]$, so $\#(S_i^c\cap [j-1])=0$ and the factor equals $x_j-t^{-(n-1)}$.
If $j>i$ then $i\in[j-1]$, so $\#(S_i^c\cap [j-1])=1$ and the factor equals $x_j-t^{-(n-2)}$.
Multiplying over $j\neq i$ yields exactly $A_i(\bx;t)$.
\end{proof}

\subsection{Normalized ``zero-position'' marginal}

\begin{corollary}[Zero-position marginal]\label{cor:marginal}
Let $\lambda$ be restricted and assume $P^*_\lambda(\bx;1,t)\neq 0$. Then for each $i\in[n]$,
\begin{equation}\label{eq:marginal}
\sum_{\mu\in\Sn(\lambda):\,\mu_i=0}\pi(\mu)
=
\frac{A_i(\bx;t)}{e^*_{n-1}(\bx;t)}.
\end{equation}
Moreover,
\begin{equation}\label{eq:e-n-1-sumAi}
e^*_{n-1}(\bx;t)=\sum_{i=1}^n A_i(\bx;t).
\end{equation}
\end{corollary}

\begin{proof}
Divide \eqref{eq:support-sum} by $P^*_\lambda(\bx;1,t)$ and use the factorization \eqref{eq:BDW51}.
By Lemma~\ref{lem:length}, $\lambda'_1=n-1$, so
\[
P^*_\lambda(\bx;1,t)
=
e^*_{n-1}(\bx;t)\cdot \prod_{2\le j\le \lambda_1} e^*_{\lambda'_j}(\bx;t),
\]
and the common factor $\prod_{j\ge 2} e^*_{\lambda'_j}$ cancels, yielding \eqref{eq:marginal}.

Finally, \eqref{eq:e-n-1-sumAi} follows directly from the definition \eqref{eq:estar}: the $(n-1)$-subsets of $[n]$ are exactly
$S_i=[n]\setminus\{i\}$, and the corresponding product is $A_i(\bx;t)$ by the same computation as in Lemma~\ref{lem:support-sum}.
\end{proof}

\begin{remark}[Independence from the positive parts of $\lambda$]
The right-hand side of \eqref{eq:marginal} depends only on $n$ and $(\bx,t)$, and on the fact that $\lambda$ has exactly one zero.
In particular, for restricted strict $\lambda$ this marginal is independent of the actual positive parts of $\lambda$.
\end{remark}

%------------------------------------------------------------
\section{Proof of Theorem~\ref{thm:main}: making a marginal negative}\label{sec:proof}

Fix $t\in(0,1)$. We now choose explicit $\bx\in(0,\infty)^n$ such that
\begin{itemize}[leftmargin=2.3em]
\item $e^*_{n-1}(\bx;t)>0$ and $P^*_\lambda(\bx;1,t)>0$;
\item $A_i(\bx;t)<0$ for some $i$,
\end{itemize}
so that the marginal \eqref{eq:marginal} is negative. That forces at least one negative entry of $\pi$.

\subsection{Parameter choice}

Choose $x_1$ such that
\begin{equation}\label{eq:x1-choice}
t^{-(n-2)}<x_1<t^{-(n-1)}.
\end{equation}
(This interval is nonempty since $t^{-(n-2)}<t^{-(n-1)}$ for $0<t<1$.)

Let $M>0$ be a parameter and set
\begin{equation}\label{eq:Mchoice}
x_2=x_3=\cdots=x_n=M.
\end{equation}

\begin{lemma}\label{lem:Ai-signs}
Assume \eqref{eq:x1-choice} and \eqref{eq:Mchoice} with $M>t^{-(n-1)}$.
Then $A_1(\bx;t)>0$ and $A_i(\bx;t)<0$ for every $i\in\{2,3,\dots,n\}$.
\end{lemma}

\begin{proof}
For $i=1$,
\[
A_1(\bx;t)=\prod_{j=2}^{n}\bigl(M-t^{-(n-2)}\bigr)=\bigl(M-t^{-(n-2)}\bigr)^{n-1}>0.
\]
For $i\ge 2$, the product $A_i(\bx;t)$ includes the factor $(x_1-t^{-(n-1)})$, which is negative by \eqref{eq:x1-choice};
all other factors are of the form $(M-t^{-(n-1)})$ or $(M-t^{-(n-2)})$, positive since $M>t^{-(n-1)}$.
\end{proof}

\subsection{Positivity of $e^*_{n-1}$}

\begin{lemma}\label{lem:e-n-1-positive}
Assume \eqref{eq:x1-choice} and \eqref{eq:Mchoice}. If $M$ satisfies
\begin{equation}\label{eq:M-explicit-bound}
M-t^{-(n-2)}>(n-1)\bigl(t^{-(n-1)}-x_1\bigr),
\end{equation}
then $e^*_{n-1}(\bx;t)>0$.
\end{lemma}

\begin{proof}
Let
\[
B:=M-t^{-(n-1)},\qquad C:=M-t^{-(n-2)}.
\]
Then $C>B>0$ once $M>t^{-(n-1)}$.

We have $A_1(\bx;t)=C^{n-1}$ as above.
For $i\ge 2$, using \eqref{eq:Ai-def} and \eqref{eq:Mchoice},
\[
A_i(\bx;t)=(x_1-t^{-(n-1)})\cdot B^{\,i-2}\,C^{\,n-i}.
\]
Hence $\abs{A_i(\bx;t)}\le (t^{-(n-1)}-x_1)\,C^{n-2}$ since $B^{i-2}C^{n-i}\le C^{n-2}$.
Summing over $i\ge 2$ and using \eqref{eq:e-n-1-sumAi} gives
\[
e^*_{n-1}(\bx;t)=\sum_{i=1}^n A_i(\bx;t)
\ge C^{n-1}-(n-1)\bigl(t^{-(n-1)}-x_1\bigr)C^{n-2}
= C^{n-2}\Bigl(C-(n-1)(t^{-(n-1)}-x_1)\Bigr),
\]
which is positive under \eqref{eq:M-explicit-bound}.
\end{proof}

\subsection{Positivity of $P^*_\lambda(\bx;1,t)$ for large $M$}

We now ensure the factorization \eqref{eq:BDW51} is positive.
Since $\lambda$ is restricted, all column lengths $\lambda'_j$ satisfy $0\le \lambda'_j\le n-1$ (the height $n$ cannot occur because
$\lambda$ has only $n-1$ positive parts).

\begin{lemma}\label{lem:estar-positive-largeM}
Fix $t\in(0,1)$ and $x_1>0$, and set $x_2=\cdots=x_n=M$.
For each $k\in\{1,2,\dots,n-1\}$, one has $e^*_k(\bx;t)>0$ for all sufficiently large $M$.
\end{lemma}

\begin{proof}
By \eqref{eq:estar}, $e^*_k(\bx;t)$ is a polynomial in $M$.
Consider the contribution from subsets $S\subseteq\{2,3,\dots,n\}$ with $|S|=k$.
For such $S$, each factor in the product is $(M-\text{constant})$, so the product has leading term $M^k$ with coefficient $1$.
There are $\binom{n-1}{k}$ such subsets, hence the coefficient of $M^k$ in $e^*_k(\bx;t)$ is at least $\binom{n-1}{k}>0$.
All other subsets $S$ that contain $1$ contribute terms of degree at most $k-1$ in $M$.
Therefore $e^*_k(\bx;t)\to +\infty$ as $M\to\infty$, and so $e^*_k(\bx;t)>0$ for all sufficiently large $M$.
\end{proof}

\begin{corollary}\label{cor:P-positive}
Let $\lambda$ be restricted. Fix $t\in(0,1)$ and choose $\bx$ as in \eqref{eq:x1-choice}--\eqref{eq:Mchoice}.
If $M$ is sufficiently large and also satisfies \eqref{eq:M-explicit-bound}, then $P^*_\lambda(\bx;1,t)>0$.
\end{corollary}

\begin{proof}
By Lemma~\ref{lem:length}, $\lambda'_1=n-1$, and by \eqref{eq:BDW51},
\[
P^*_\lambda(\bx;1,t)=\prod_{1\le j\le \lambda_1} e^*_{\lambda'_j}(\bx;t).
\]
The factor $e^*_{\lambda'_1}=e^*_{n-1}$ is positive by Lemma~\ref{lem:e-n-1-positive}.
All remaining factors are among $e^*_0,e^*_1,\dots,e^*_{n-1}$; here $e^*_0=1$ and
$e^*_k>0$ for $1\le k\le n-1$ when $M$ is large by Lemma~\ref{lem:estar-positive-largeM}.
\end{proof}

\subsection{Negative marginal and conclusion}

\begin{lemma}\label{lem:negative-marginal}
Let $\lambda$ be restricted. Fix $t\in(0,1)$ and choose $\bx$ as in \eqref{eq:x1-choice}--\eqref{eq:Mchoice}.
If $M$ satisfies Corollary~\ref{cor:P-positive}, then for each $i\in\{2,\dots,n\}$,
\[
\sum_{\mu\in\Sn(\lambda):\,\mu_i=0}\pi(\mu)\;<\;0.
\]
\end{lemma}

\begin{proof}
By Corollary~\ref{cor:P-positive}, $P^*_\lambda(\bx;1,t)>0$ and hence $\pi$ is defined by \eqref{eq:pi-def}.
By Lemma~\ref{lem:Ai-signs}, $A_i(\bx;t)<0$ for all $i\ge 2$.
By Lemma~\ref{lem:e-n-1-positive}, $e^*_{n-1}(\bx;t)>0$.
Therefore \eqref{eq:marginal} is negative for $i\ge 2$.
\end{proof}

\begin{proof}[Proof of Theorem~\ref{thm:main}]
Fix $t\in(0,1)$ and choose $x_1$ as in \eqref{eq:x1-choice}.
Choose $M$ large enough that \eqref{eq:M-explicit-bound} holds and Corollary~\ref{cor:P-positive} applies.
Then $P^*_\lambda(\bx;1,t)>0$, and Lemma~\ref{lem:negative-marginal} gives some $i\ge 2$ with
\[
\sum_{\mu\in\Sn(\lambda):\,\mu_i=0}\pi(\mu)<0.
\]
If all $\pi(\mu)$ were nonnegative, this sum of some of the $\pi(\mu)$ would be $\ge 0$.
Hence there exists $\mu$ with $\mu_i=0$ and $\pi(\mu)<0$, proving (b).

Finally, a stationary distribution of any genuine Markov chain is a probability vector and cannot have a negative entry.
Thus, at these positive parameters there exists \emph{no} Markov chain with stationary distribution \eqref{eq:pi-def}.
In particular, there is no Markov chain realizing \eqref{eq:pi-def} uniformly on the full positive domain \eqref{eq:positive-domain}.
\end{proof}

%------------------------------------------------------------
\section{Discussion: the nontriviality constraint}\label{sec:discussion}

\begin{remark}[Why generic Metropolis--Hastings does not answer the problem]
If for some parameter point one \emph{assumes} all ratios \eqref{eq:pi-def} are strictly positive, then one can always manufacture a reversible
Markov chain by Metropolis--Hastings or heat-bath updates on any connected graph on $\Sn(\lambda)$, using acceptance probabilities expressed in terms
of the target weights.
However, such constructions necessarily require evaluating (or otherwise accessing) the global weights $F^*_\mu(\bx;1,t)$ as functions of the state $\mu$,
which is precisely what the problem statement rules out by its definition of ``nontrivial''.
Theorem~\ref{thm:main} shows that even before addressing nontriviality, there is a more basic obstruction: restrictedness does not guarantee positivity,
so the displayed ratios are not uniformly stationary distributions of genuine Markov chains on the positive parameter domain.
\end{remark}

\begin{remark}[A minimal two-state illustration]
For $n=2$ and $\lambda=(k,0)$ with $k\ge 2$, $\Sn(\lambda)=\{(k,0),(0,k)\}$.
Theorem~\ref{thm:BDW71} yields explicit formulas for the ratios, and one can write down local reversible two-state dynamics when those ratios are nonnegative.
This illustrates what a ``local rational-function'' construction can look like, but does not contradict Theorem~\ref{thm:main} since the obstruction is
global over the full parameter domain.
\end{remark}

%------------------------------------------------------------
\begin{thebibliography}{99}

\bibitem{BDW}
H.~Ben Dali and L.~K.~Williams,
\emph{A combinatorial formula for interpolation Macdonald polynomials},
arXiv:2510.02587v2 [math.CO] (2025).

\end{thebibliography}

\end{document}
