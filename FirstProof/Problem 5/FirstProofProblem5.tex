\documentclass[11pt,reqno]{amsart}

% ------------------------------------------------------------
% Page layout
% ------------------------------------------------------------
\usepackage[a4paper,margin=1in]{geometry}

% ------------------------------------------------------------
% Packages
% ------------------------------------------------------------
\usepackage{amsmath,amssymb,amsfonts,amsthm,mathtools}
\usepackage{mathrsfs}
\usepackage[mathcal]{eucal}
\usepackage{enumitem}
\usepackage[colorlinks=true,linkcolor=blue,citecolor=blue,urlcolor=blue]{hyperref}

% ------------------------------------------------------------
% Macros
% ------------------------------------------------------------
\newcommand{\Sp}{\mathrm{Sp}}
\newcommand{\SpG}{\Sp^{G}}
\newcommand{\SpH}{\Sp^{H}}

\newcommand{\cO}{\mathscr O}
\newcommand{\cP}{\mathcal P}
\newcommand{\tE}{\widetilde{E}}

\newcommand{\Res}{\mathrm{Res}}
\newcommand{\Ind}{\mathrm{Ind}}
\newcommand{\Inf}{\mathrm{Inf}}
\newcommand{\Loc}{\mathrm{Loc}}

\newcommand{\Z}{\mathbb Z}
\newcommand{\N}{\mathbb N}
\newcommand{\R}{\mathbb R}

\DeclareMathOperator{\conn}{conn}

\newcommand{\ceil}[1]{\left\lceil #1\right\rceil}
\newcommand{\floor}[1]{\left\lfloor #1\right\rfloor}

% ------------------------------------------------------------
% Theorem styles
% ------------------------------------------------------------
\theoremstyle{plain}
\newtheorem{theorem}{Theorem}[section]
\newtheorem{proposition}[theorem]{Proposition}
\newtheorem{lemma}[theorem]{Lemma}
\newtheorem{corollary}[theorem]{Corollary}

\theoremstyle{definition}
\newtheorem{definition}[theorem]{Definition}
\newtheorem{remark}[theorem]{Remark}
\newtheorem{example}[theorem]{Example}

\numberwithin{equation}{section}

% ------------------------------------------------------------
% Title data
% ------------------------------------------------------------
\title[\(\cO\)-slice filtrations and geometric fixed points]
{\(\cO\)-adapted slice filtrations for incomplete transfer systems\\
and a geometric fixed point criterion for slice connectivity}
\date{\today}

\subjclass[2020]{55N91, 55P91, 55Q10}
\keywords{equivariant stable homotopy theory, slice filtration, \(N_\infty\)-operads,
transfer systems, geometric fixed points}

% ------------------------------------------------------------
\begin{document}

\begin{abstract}
Fix a finite group \(G\) and let \(\cO\) be a transfer system arising from an \(N_\infty\)-operad.
We define an \(\cO\)-adapted analogue of the Hill--Hopkins--Ravenel slice-connectivity filtration on the
stable category of genuine \(G\)-spectra by restricting the allowed slice cells to those built from
\(\cO\)-admissible transitive orbits \(H/K\).
For a genuinely connective \(G\)-spectrum \(X\) and each integer \(n\ge 0\), we prove that \(\cO\)-slice
\(n\)-connectivity of \(X\) is equivalent to an explicit collection of Postnikov-connectivity bounds on the
geometric fixed points \(\Phi^H(X)\) for all subgroups \(H\le G\).
The scaling constant is the maximal admissible index
\[
M_{\cO}(H)\ :=\ \max\{[H:K]\mid K\le_{\cO}H\}.
\]
Precisely, \(X\) is \(\cO\)-slice \(n\)-connective if and only if
\(
\conn(\Phi^H(X))\ge \floor{n/M_{\cO}(H)}
\)
for every \(H\le G\).
A regular (even-cell) variant replaces \(\floor{\cdot}\) by \(\ceil{\cdot}\).
\end{abstract}

\maketitle
\tableofcontents

% ------------------------------------------------------------
\section{Introduction}

The slice filtration of Hill--Hopkins--Ravenel \cite{HHR} provides a powerful equivariant refinement of
Postnikov connectivity, measuring complexity by representation spheres induced from subgroups.
Independently, Blumberg--Hill \(N_\infty\)-operads \cite{BH} encode homotopy-commutative equivariant
multiplicative structures with \emph{incomplete} norms and transfers.  The orbit-level data of an
\(N_\infty\)-operad can be packaged as a \emph{transfer system} on the subgroup lattice (see Rubin \cite{Rub}).

\smallskip

The purpose of this paper is twofold:
\begin{itemize}[leftmargin=2.2em]
\item We define a slice-connectivity filtration on genuine \(G\)-spectra adapted to a given transfer system
\(\cO\) by restricting the allowed slice cells to those associated to \(\cO\)-admissible orbits.
\item For genuinely connective \(G\)-spectra, we prove that this \(\cO\)-slice connectivity is detected by
geometric fixed points, with a sharp scaling constant determined by \(\cO\).
\end{itemize}

\subsection*{The scaling constant}
For each subgroup \(H\le G\), define the \emph{maximal admissible index}
\[
M_{\cO}(H)\ :=\ \max\{[H:K]\mid K\le_{\cO}H\}.
\]
This depends only on the restriction of \(\cO\) to subgroups of \(H\) and is invariant under conjugation.

\subsection*{Main theorem}
We define \(\tau^{G,\cO}_{\ge n}\subseteq \Sp^G\) as the localizing preaisle generated by the
\(\cO\)-slice cells of degree \(\ge n\) (Definition~\ref{def:tau} below).  Membership in
\(\tau^{G,\cO}_{\ge n}\) is called \(\cO\)-slice \(n\)-connectivity.

\begin{theorem}[Geometric fixed point criterion]\label{thm:main}
Let \(G\) be a finite group and let \(\cO\) be a transfer system arising from an \(N_\infty\)-operad.
Let \(X\in \Sp^G\) be \emph{genuinely connective} (Definition~\ref{def:connective}) and let \(n\ge 0\).
Then the following are equivalent:
\begin{enumerate}[label=\textup{(\arabic*)},leftmargin=2.2em]
\item \(X\in\tau^{G,\cO}_{\ge n}\).
\item For every subgroup \(H\le G\),
\[
\conn\bigl(\Phi^{H}(X)\bigr)\ \ge\ \floor{\frac{n}{M_{\cO}(H)}}.
\]
\end{enumerate}
\end{theorem}

A regular (even-cell) variant is stated in Remark~\ref{rem:regular-scaling}.

\subsection*{A word on rigor}
A key technical point is the \emph{geometric reduction} argument in
Lemma~\ref{lem:geometricreduction}.  Its proof must carefully separate categorical and geometric fixed
points; an earlier draft conflated these in one step.  We correct this by invoking a
tom Dieck splitting argument (Lemma~\ref{lem:rep-sphere-connective}) to verify genuine connectivity for
certain representation spheres.

\subsection*{Conventions}
We work in a standard symmetric monoidal model for genuine \(G\)-spectra (e.g.\ orthogonal \(G\)-spectra
in a complete universe), or equivalently in the underlying stable presentable \(\infty\)-category.
All functors are derived, and all colimits are homotopy colimits.

For an ordinary spectrum \(E\), we write \(\conn(E)\ge q\) if \(\pi_i(E)=0\) for all \(i<q\).

% ------------------------------------------------------------
\section{Transfer systems and maximal admissible index}\label{sec:transfer}

\subsection{Transfer systems}

\begin{definition}[Transfer system]\label{def:transfersystem}
A \emph{transfer system} on a finite group \(G\) is a relation \(\le_{\cO}\) on the set of subgroups of \(G\)
such that for all subgroups \(K,H,L\le G\):
\begin{enumerate}[label=(\roman*),leftmargin=2.2em]
\item \textup{(Refines inclusion)} \(K\le_{\cO}H\Rightarrow K\le H\).
\item \textup{(Reflexive)} \(H\le_{\cO}H\).
\item \textup{(Transitive)} If \(L\le_{\cO}K\le_{\cO}H\) then \(L\le_{\cO}H\).
\item \textup{(Conjugation)} If \(K\le_{\cO}H\) then \(gKg^{-1}\le_{\cO}gHg^{-1}\) for all \(g\in G\).
\item \textup{(Restriction/intersection)} If \(K\le_{\cO}H\) and \(L\le H\), then \(K\cap L\le_{\cO}L\).
\end{enumerate}
\end{definition}

\begin{remark}
An \(N_\infty\)-operad determines an indexing system of admissible finite \(H\)-sets \cite{BH}.
Restricting to transitive admissible \(H\)-sets \(H/K\) yields a transfer system:
\(H/K\) is admissible if and only if \(K\le_{\cO}H\).  See \cite{Rub} for a combinatorial account.
In this paper we only use the axioms of Definition~\ref{def:transfersystem}.
\end{remark}

\subsection{Maximal admissible index}

\begin{definition}[Maximal admissible index]\label{def:MO}
For each subgroup \(H\le G\), define
\[
M_{\cO}(H)\ :=\ \max\{[H:K]\mid K\le_{\cO}H\}\ \in\ \N.
\]
\end{definition}

\begin{lemma}\label{lem:MO-basic}
For all \(H\le G\), \(M_{\cO}(H)\ge 1\) and \(M_{\cO}(H)\) is conjugation invariant:
\(M_{\cO}(gHg^{-1})=M_{\cO}(H)\).
Moreover, if \(L\le H\), then \(M_{\cO}(L)\) is unchanged when computed using the restricted transfer
system on \(H\).
\end{lemma}

\begin{proof}
The set \(\{K\le H\mid K\le_{\cO}H\}\) is finite and nonempty (it contains \(H\) by reflexivity), so the
maximum exists and is at least \([H:H]=1\).
Conjugation invariance follows from Definition~\ref{def:transfersystem}(iv) and the equality
\([gHg^{-1}:gKg^{-1}]=[H:K]\).
Restriction to \(H\) does not change which relations among subgroups of \(H\) hold, hence the maxima agree.
\end{proof}

\begin{example}\label{ex:complete-trivial}
If \(\cO\) is trivial (only \(H\le_{\cO}H\)), then \(M_{\cO}(H)=1\) for all \(H\).
If \(\cO\) is complete (all subgroup inclusions are \(\cO\)-admissible), then \(M_{\cO}(H)=|H|\).
\end{example}

% ------------------------------------------------------------
\section{Localizing preaisles and Postnikov connectivity}\label{sec:preaisles}

Slice-connectivity conditions are closed under suspension and colimits but not generally under
desuspension.  It is therefore convenient to use a one-sided notion of localizing subcategory.

\begin{definition}[Localizing preaisle]\label{def:preaisle}
Let \(\mathcal C\) be a stable presentable \(\infty\)-category.
A full subcategory \(\tau\subseteq \mathcal C\) is a \emph{localizing preaisle} if it is closed under:
\begin{enumerate}[label=(\roman*),leftmargin=2.2em]
\item equivalences and retracts,
\item all small colimits,
\item suspension \(\Sigma\),
\item extensions: if \(A\to B\to C\) is a cofiber sequence with \(A,C\in\tau\), then \(B\in\tau\).
\end{enumerate}
Given a set of objects \(S\subseteq\mathcal C\), we write \(\Loc(S)\) for the smallest localizing preaisle
containing \(S\).
\end{definition}

\begin{lemma}[Cofiber closure]\label{lem:cofiber-closure}
If \(\tau\) is a localizing preaisle and \(f:A\to B\) is a map with \(A,B\in\tau\), then
\(\mathrm{cofib}(f)\in\tau\).
\end{lemma}

\begin{proof}
Let \(A\to B\to C\) be a cofiber sequence with \(C=\mathrm{cofib}(f)\).
Rotate to \(B\to C\to \Sigma A\).
Since \(\Sigma A\in\tau\) and \(B\in\tau\), extension closure gives \(C\in\tau\).
\end{proof}

\begin{definition}[Postnikov-connective spectra]\label{def:postnikov}
For \(q\in\Z\), let \(\Sp_{\ge q}\subseteq \Sp\) denote the full subcategory of \(q\)-connective spectra:
\(\pi_i(E)=0\) for all \(i<q\).
Equivalently, \(\Sp_{\ge q}=\Loc(\{S^k\mid k\ge q\})\).
We write \(\conn(E)\ge q\) to mean \(E\in\Sp_{\ge q}\).
\end{definition}

% ------------------------------------------------------------
\section{\(\cO\)-slice cells and the \(\cO\)-adapted slice filtration}\label{sec:filtration}

\subsection{Permutation representations}
For \(K\le H\), let \(\rho_{H/K}=\R[H/K]\) be the real permutation representation of \(H\) on the coset set
\(H/K\). Then \(\dim(\rho_{H/K})=[H:K]\).
For a (virtual) real \(H\)-representation \(V\), write \(S^V\) for the associated representation sphere.

\subsection{\(\cO\)-slice cells}

\begin{definition}[\(\cO\)-slice cells]\label{def:O-slice-cells}
Let \(K\le_{\cO}H\le G\), let \(m\in\N\), and let \(\varepsilon\in\{0,1\}\), with the convention that
\(m\ge 1\) if \(\varepsilon=1\).
Define the \emph{\(\cO\)-slice cell}
\[
C(H,K;m,\varepsilon)\ :=\ G_+\wedge_H S^{m\rho_{H/K}-\varepsilon}\ \in\ \Sp^G,
\]
and define its \emph{\(\cO\)-slice degree} by
\[
\deg_{\cO}\bigl(C(H,K;m,\varepsilon)\bigr)\ :=\ m[H:K]-\varepsilon\ \in\ \N.
\]
\end{definition}

\begin{remark}[Regular variant]\label{rem:regular-variant}
Restricting to \(\varepsilon=0\) yields the \emph{regular} \(\cO\)-slice filtration.
In the regular case, the geometric fixed point criterion replaces \(\floor{\cdot}\) by \(\ceil{\cdot}\);
see Remark~\ref{rem:regular-scaling}.
\end{remark}

\subsection{The filtration}

\begin{definition}[\(\cO\)-adapted slice filtration]\label{def:tau}
For \(n\in\N\), define \(\tau^{G,\cO}_{\ge n}\subseteq \Sp^G\) to be the localizing preaisle generated by all
\(\cO\)-slice cells of degree at least \(n\):
\[
\tau^{G,\cO}_{\ge n}
\ :=\
\Loc\Bigl\{
C(H,K;m,\varepsilon)\ \Bigm|\ K\le_{\cO}H\le G,\ m\in\N,\ \varepsilon\in\{0,1\},\
\deg_{\cO}(C(H,K;m,\varepsilon))\ge n
\Bigr\}.
\]
A \(G\)-spectrum \(X\) is \emph{\(\cO\)-slice \(n\)-connective} if \(X\in\tau^{G,\cO}_{\ge n}\).
\end{definition}

\begin{remark}
The filtration is decreasing: \(\tau^{G,\cO}_{\ge n+1}\subseteq\tau^{G,\cO}_{\ge n}\).
It is tailored to connectivity: it is closed under suspension and colimits but not under desuspension.
\end{remark}

% ------------------------------------------------------------
\section{Geometric fixed points}\label{sec:geom}

\subsection{Universal spaces for families}
For a finite group \(H\), let \(\cP_H\) denote the family of proper subgroups of \(H\).
A universal \(\cP_H\)-space \(E\cP_H\) is an \(H\)-CW complex characterized up to \(H\)-equivalence by
\[
(E\cP_H)^K\simeq *
\quad\text{for }K\in\cP_H,
\qquad
(E\cP_H)^H=\varnothing.
\]
Let \(\tE\cP_H\) denote the cofiber of the based map \((E\cP_H)_+\to S^0\).

\subsection{Definition and standard properties}

\begin{definition}[Geometric fixed points]\label{def:Phi}
For \(H\le G\), the geometric fixed point functor \(\Phi^H:\Sp^G\to\Sp\) is defined by
\[
\Phi^H(X)\ :=\ \bigl(\tE\cP_H\wedge \Res^G_H X\bigr)^H .
\]
\end{definition}

\begin{proposition}[Standard properties]\label{prop:Phi-standard}
For each \(H\le G\), \(\Phi^H\) is exact, preserves all colimits, and is strong symmetric monoidal.
In particular, \(\Phi^H(S^V)\simeq S^{V^H}\) for any finite-dimensional real \(G\)-representation \(V\).
Moreover, the family \(\{\Phi^H\}_{H\le G}\) jointly detects equivalences in \(\Sp^G\).
\end{proposition}

\begin{proof}
Exactness, colimit preservation, and strong symmetric monoidality are standard; see, for example,
Schwede \cite[\S7]{Schwede}.
Detection of equivalences by geometric fixed points is part of Schwede's Theorem~7.12 \cite{Schwede}.
\end{proof}

\subsection{Geometric fixed points kill proper induction}

\begin{lemma}\label{lem:PhiKillsInd}
Let \(H\) be a finite group and \(K<H\) a proper subgroup. Then for any \(Z\in\Sp^K\),
\[
\Phi^H\bigl(H_+\wedge_K Z\bigr)\ \simeq\ 0.
\]
\end{lemma}

\begin{proof}
Using Definition~\ref{def:Phi} and the projection formula for induction,
\[
\Phi^H(H_+\wedge_K Z)
\simeq \bigl(\tE\cP_H\wedge H_+\wedge_K Z\bigr)^H
\simeq \Bigl(H_+\wedge_K\bigl(\Res^H_K\tE\cP_H\wedge Z\bigr)\Bigr)^H .
\]
Since \(K<H\), every subgroup of \(K\) is proper in \(H\), hence \(\Res^H_K(E\cP_H)\) is \(K\)-contractible and
\(\Res^H_K(\tE\cP_H)\simeq *\). Therefore the induced spectrum is contractible, hence its \(H\)-fixed points
are contractible.
\end{proof}

\subsection{A double-coset formula}

\begin{proposition}[Geometric fixed points of induction]\label{prop:PhiInd}
Let \(J\le G\), let \(Y\in\Sp^J\), and let \(H\le G\).
Then there is a natural equivalence
\[
\Phi^H\bigl(G_+\wedge_J Y\bigr)
\ \simeq\
\bigvee_{\substack{[g]\in H\backslash G/J\\ g^{-1}Hg\le J}}\ \Phi^{g^{-1}Hg}(Y).
\]
\end{proposition}

\begin{proof}
Restrict to \(H\) and apply the Mackey decomposition for \(\Res^G_H\Ind_J^G\):
\[
\Res^G_H(G_+\wedge_J Y)
\simeq
\bigvee_{[g]\in H\backslash G/J}
H_+\wedge_{H\cap gJg^{-1}}
\Res^{gJg^{-1}}_{H\cap gJg^{-1}}\,c_g(Y),
\]
where \(c_g\) is conjugation by \(g\).
Applying \(\Phi^H\) and using Lemma~\ref{lem:PhiKillsInd}, all summands with \(H\cap gJg^{-1}<H\) vanish.
The surviving summands are exactly those with \(H\le gJg^{-1}\), i.e.\ \(g^{-1}Hg\le J\).
For such \(g\), the remaining term identifies (up to conjugation) with \(\Phi^{g^{-1}Hg}(Y)\).
\end{proof}

% ------------------------------------------------------------
\section{Orbit bounds and connectivity of generators}\label{sec:cellconnect}

We next extract a uniform lower bound on orbit counts of restricted admissible \(H\)-sets, and use it to
control the connectivity of geometric fixed points of the generators of \(\tau^{G,\cO}_{\ge n}\).

\begin{lemma}[Orbit counting bound]\label{lem:orbitcount}
Let \(H\le J\le G\) and \(K\le_{\cO}J\).
Then
\[
|H\backslash J/K|\ \ge\ \ceil{\frac{[J:K]}{M_{\cO}(H)}}.
\]
\end{lemma}

\begin{proof}
Decompose the restricted \(H\)-set \(\Res^J_H(J/K)\) into \(H\)-orbits:
\[
\Res^J_H(J/K)\ \cong\ \bigsqcup_{i\in I} H/L_i,
\qquad L_i=H\cap g_iKg_i^{-1}\text{ for some }g_i\in J.
\]
Since \(K\le_{\cO}J\), conjugation invariance gives \(g_iKg_i^{-1}\le_{\cO}J\).
Then the restriction axiom implies \(L_i=H\cap g_iKg_i^{-1}\le_{\cO}H\).
Hence \([H:L_i]\le M_{\cO}(H)\) for all \(i\).
Counting cardinalities,
\[
[J:K]=\sum_{i\in I}[H:L_i]\ \le\ |I|\cdot M_{\cO}(H),
\]
so \(|I|\ge \ceil{[J:K]/M_{\cO}(H)}\).
Finally \(|I|=|H\backslash J/K|\).
\end{proof}

\begin{lemma}[Fixed vectors in permutation representations]\label{lem:perm-fixed-dim}
Let \(L\) be a finite group and \(T\) a finite \(L\)-set. Then \(\dim(\R[T]^L)=|L\backslash T|\).
\end{lemma}

\begin{proof}
An element of \(\R[T]\) is a real-valued function on \(T\); it is \(L\)-fixed iff it is constant on each
\(L\)-orbit. Thus \(\R[T]^L\cong \R^{(L\backslash T)}\).
\end{proof}

\begin{lemma}[An integer inequality]\label{lem:integerineq}
Let \(m\in\N\), \(a,M\in\N\) with \(M\ge 1\), and \(\varepsilon\in\{0,1\}\).
Then
\[
m\ceil{\frac{a}{M}}-\varepsilon\ \ge\ \floor{\frac{ma-\varepsilon}{M}}.
\]
\end{lemma}

\begin{proof}
Let \(t=\ceil{a/M}\), so \(tM\ge a\). Then \(mtM-\varepsilon\ge ma-\varepsilon\), hence
\[
mt-\varepsilon/M\ \ge\ (ma-\varepsilon)/M.
\]
Taking floors gives
\[
\floor{\frac{ma-\varepsilon}{M}}\ \le\ \floor{mt-\varepsilon/M}.
\]
Since \(mt\in\Z\) and \(\varepsilon\in\{0,1\}\) with \(M\ge 1\), we have \(\floor{-\varepsilon/M}=-\varepsilon\)
(in fact, \(\floor{0}=0\) and \(\floor{-1/M}=-1\) for all \(M\ge 1\)). Therefore
\[
\floor{mt-\varepsilon/M}
= mt+\floor{-\varepsilon/M}
= mt-\varepsilon,
\]
which is the desired inequality.
\end{proof}

\begin{proposition}[Connectivity of geometric fixed points of \(\cO\)-slice cells]\label{prop:generatorconnectivity}
Let \(C=C(J,K;m,\varepsilon)\) be an \(\cO\)-slice cell with
\(\deg_{\cO}(C)=m[J:K]-\varepsilon\ge n\).
Then for every subgroup \(H\le G\),
\[
\conn\bigl(\Phi^H(C)\bigr)\ \ge\ \floor{\frac{n}{M_{\cO}(H)}}.
\]
\end{proposition}

\begin{proof}
By Proposition~\ref{prop:PhiInd}, \(\Phi^H(C)\) is a finite wedge of spectra of the form
\[
\Phi^{L}\!\bigl(S^{m\rho_{J/K}-\varepsilon}\bigr)\ \simeq\ S^{m\dim((\rho_{J/K})^{L})-\varepsilon},
\]
where \(L=g^{-1}Hg\le J\).
By Lemma~\ref{lem:perm-fixed-dim} with \(T=J/K\), \(\dim((\rho_{J/K})^{L})=|L\backslash J/K|\).
By Lemma~\ref{lem:orbitcount},
\[
|L\backslash J/K|\ \ge\ \ceil{\frac{[J:K]}{M_{\cO}(L)}}
=\ceil{\frac{[J:K]}{M_{\cO}(H)}},
\]
using conjugation invariance of \(M_{\cO}(-)\) (Lemma~\ref{lem:MO-basic}).
Therefore each sphere summand has dimension at least
\[
m\ceil{\frac{[J:K]}{M_{\cO}(H)}}-\varepsilon
\ \ge\
\floor{\frac{m[J:K]-\varepsilon}{M_{\cO}(H)}}
\ \ge\
\floor{\frac{n}{M_{\cO}(H)}},
\]
by Lemma~\ref{lem:integerineq} and the assumption \(\deg_{\cO}(C)\ge n\).
A wedge of \(q\)-connective spectra is \(q\)-connective since \(\Sp_{\ge q}\) is closed under colimits.
\end{proof}

% ------------------------------------------------------------
\section{Genuine connectivity and a tom Dieck splitting estimate}\label{sec:connective}

\begin{definition}[Genuine connectivity]\label{def:connective}
A genuine \(G\)-spectrum \(X\in\Sp^G\) is \emph{connective} if for every subgroup \(H\le G\), the categorical
fixed point spectrum \(X^H\) is \(0\)-connective, i.e.\ \(\pi_i(X^H)=0\) for all \(i<0\).
\end{definition}

\begin{remark}
Equivalently, the negative homotopy Mackey functors \(\underline{\pi}_k(X)\) vanish for \(k<0\).
This is strictly stronger than underlying (nonequivariant) connectivity.
\end{remark}

\begin{lemma}[Shifted representation spheres are connective]\label{lem:rep-sphere-connective}
Let \(H\) be a finite group and let \(V\) be a finite-dimensional real \(H\)-representation.
Assume that \(\dim(V^K)\ge 1\) for every subgroup \(K\le H\).
Then the genuine \(H\)-spectrum \(S^{V-1}\) is connective in the sense of
Definition~\ref{def:connective}.
\end{lemma}

\begin{proof}
Fix a subgroup \(L\le H\). We must show that the categorical fixed points \((S^{V-1})^L\) are
\(0\)-connective.

Restricting from \(H\) to \(L\) identifies \(S^{V-1}\) with \(S^{(\Res^H_LV)-1}\) as \(L\)-spectra, so we may
work in \(\Sp^L\).
Write \(S^{V-1}\simeq \Sigma^{-1}\Sigma^\infty_L S^{V}\) as \(L\)-spectra.
By the tom Dieck splitting for categorical fixed points of equivariant suspension spectra
(see, e.g., Schwede \cite[Example~7.7 and \S6]{Schwede} or \cite{MM}),
there is a natural splitting
\[
\bigl(\Sigma^\infty_L S^{V}\bigr)^L
\ \simeq\
\bigvee_{(K\le L)}\ \Sigma^\infty\!\Bigl(EW_LK_+\wedge_{W_LK} (S^{V})^K\Bigr),
\]
where \(W_LK=N_L(K)/K\) and the wedge runs over \(L\)-conjugacy classes of subgroups \(K\le L\).
For each \(K\le L\), we have \((S^{V})^K\simeq S^{V^K}\) and \(\dim(V^K)\ge 1\) by hypothesis.
Since \(EW_LK_+\) is a based CW complex with cells in dimensions \(\ge 0\), the smash
\(EW_LK_+\wedge_{W_LK} S^{V^K}\) is a based CW complex with all cells in dimensions \(\ge \dim(V^K)\ge 1\).
Hence each summand \(\Sigma^\infty(EW_LK_+\wedge_{W_LK} S^{V^K})\) is at least \(1\)-connective, and so
\[
\conn\Bigl(\bigl(\Sigma^\infty_L S^{V}\bigr)^L\Bigr)\ \ge\ 1.
\]
Desuspending once yields
\[
\conn\bigl((S^{V-1})^L\bigr)\ =\ \conn\Bigl(\Sigma^{-1}\bigl(\Sigma^\infty_L S^{V}\bigr)^L\Bigr)\ \ge\ 0.
\]
Since this holds for every \(L\le H\), the spectrum \(S^{V-1}\) is connective.
\end{proof}

% ------------------------------------------------------------
\section{Main theorem: \(\cO\)-slice connectivity detected by geometric fixed points}\label{sec:main}

For \(H\le G\) and \(n\in\N\), define
\[
q_{\cO}(H,n)\ :=\ \floor{\frac{n}{M_{\cO}(H)}}\ \in\ \N.
\]

\subsection{A functoriality lemma}

\begin{lemma}[Induction preserves \(\cO\)-slice connectivity]\label{lem:IndPreservesTau}
Let \(H\le G\) and let \(\cO|_H\) denote the restricted transfer system on \(H\).
Then for every \(n\ge 0\), induction \(\Ind_H^G:\Sp^H\to\Sp^G\) carries
\(\tau^{H,\cO|_H}_{\ge n}\) into \(\tau^{G,\cO}_{\ge n}\).
\end{lemma}

\begin{proof}
Induction is exact and preserves all colimits, hence preserves the closure operations defining \(\Loc(-)\).
A generating \(\cO|_H\)-cell has the form
\(
H_+\wedge_{H'}S^{m\rho_{H'/K}-\varepsilon}
\)
with \(K\le_{\cO}H'\le H\) and degree \(\ge n\).
Inducing to \(G\) yields
\[
\Ind_H^G\bigl(H_+\wedge_{H'}S^{m\rho_{H'/K}-\varepsilon}\bigr)
\simeq
G_+\wedge_{H'}S^{m\rho_{H'/K}-\varepsilon},
\]
which is a generating \(\cO\)-cell of the same degree.
\end{proof}

\subsection{Isotropy separation and geometric spectra}

Let \(\cP=\cP_G\) be the family of proper subgroups of \(G\).
The isotropy separation cofiber sequence is
\begin{equation}\label{eq:isosep}
(E\cP)_+\wedge X \longrightarrow X \longrightarrow \tE\cP\wedge X .
\end{equation}
Write \(L_{\mathrm{geom}}=\tE\cP\wedge(-)\).
A \(G\)-spectrum \(Y\) is \emph{geometric} if \(Y\simeq L_{\mathrm{geom}}(Y)\), equivalently if
\(\Phi^H(Y)\simeq 0\) for all proper \(H<G\).
Let \(\Sp^G_{\mathrm{geom}}\subseteq\Sp^G\) denote the full subcategory of geometric \(G\)-spectra.

\begin{lemma}\label{lem:PhiG-tildeEP}
There is a natural equivalence \(\Phi^G(\tE\cP)\simeq S^0\).
\end{lemma}

\begin{proof}
Apply \(\Phi^G\) to the cofiber sequence \((E\cP)_+\to S^0\to \tE\cP\).
Every cell of \(E\cP\) has isotropy a proper subgroup, so \((E\cP)_+\) is built from induced spectra from
proper subgroups. By Lemma~\ref{lem:PhiKillsInd} and exactness, \(\Phi^G((E\cP)_+)\simeq 0\).
Since \(\Phi^G(S^0)\simeq S^0\), the cofiber identifies \(\Phi^G(\tE\cP)\simeq S^0\).
\end{proof}

\begin{lemma}[Geometric fixed points are an equivalence on geometric spectra]\label{lem:geom-equivalence}
The functor \(\Phi^G:\Sp^G_{\mathrm{geom}}\to\Sp\) is an equivalence.
A quasi-inverse \(F:\Sp\to\Sp^G_{\mathrm{geom}}\) is given by
\[
F(E)\ :=\ \tE\cP\wedge \Inf(E),
\]
where \(\Inf:\Sp\to\Sp^G\) denotes inflation (trivial \(G\)-action).
\end{lemma}

\begin{proof}
We first show \(\Phi^G\circ F\simeq \mathrm{id}_{\Sp}\).
Using Lemma~\ref{lem:PhiG-tildeEP} and strong monoidality of \(\Phi^G\),
\[
\Phi^G(F(E))
\simeq
\Phi^G(\tE\cP)\wedge \Phi^G(\Inf(E))
\simeq
S^0\wedge \Phi^G(\Inf(E)).
\]
The composite \(\Phi^G\circ\Inf:\Sp\to\Sp\) is exact and preserves all colimits (Proposition~\ref{prop:Phi-standard})
and sends \(S^0\) to \(S^0\).
In the stable presentable \(\infty\)-category \(\Sp\), any exact colimit-preserving endofunctor is determined
(up to equivalence) by its value on \(S^0\), and is given by smashing with that value.
Hence \(\Phi^G\circ\Inf\simeq \mathrm{id}_{\Sp}\), so \(\Phi^G(F(E))\simeq E\).

Conversely, let \(Y\in\Sp^G_{\mathrm{geom}}\).
The \emph{counit} map \(F(\Phi^G(Y)) \to Y\) induces an equivalence on \(\Phi^G\) by the previous paragraph.
For every proper \(H<G\), both \(\Phi^H(F(\Phi^G(Y)))\) and \(\Phi^H(Y)\) are zero because both spectra are
geometric.
Since geometric fixed points jointly detect equivalences (Proposition~\ref{prop:Phi-standard}),
the counit map \(F(\Phi^G(Y))\to Y\) is an equivalence.
\end{proof}

\subsection{Geometric reduction for \(\tau^{G,\cO}_{\ge n}\)}

\begin{lemma}[Geometric reduction]\label{lem:geometricreduction}
Fix \(n\ge 0\) and set \(q=\floor{n/M_{\cO}(G)}\).
Assume Theorem~\ref{thm:main} holds for all proper subgroups of \(G\).
If \(Y\in\Sp^G\) is geometric, then
\[
Y\in\tau^{G,\cO}_{\ge n}
\quad\Longleftrightarrow\quad
\conn(\Phi^G(Y))\ge q.
\]
\end{lemma}

\begin{proof}
Let \(\mathcal U\subseteq\Sp\) be the essential image
\(\mathcal U:=\Phi^G(\tau^{G,\cO}_{\ge n}\cap \Sp^G_{\mathrm{geom}})\).
Since \(\Phi^G:\Sp^G_{\mathrm{geom}}\to\Sp\) is an equivalence (Lemma~\ref{lem:geom-equivalence}) and
\(\tau^{G,\cO}_{\ge n}\cap \Sp^G_{\mathrm{geom}}\) is a localizing preaisle in \(\Sp^G_{\mathrm{geom}}\),
the subcategory \(\mathcal U\) is a localizing preaisle in \(\Sp\).

\smallskip\noindent
\emph{Step 1: \(\mathcal U\subseteq \Sp_{\ge q}\).}
If \(Y\in \tau^{G,\cO}_{\ge n}\cap \Sp^G_{\mathrm{geom}}\), then \(Y\in\tau^{G,\cO}_{\ge n}\).
By Proposition~\ref{prop:generatorconnectivity} and exactness/colimit preservation of \(\Phi^G\),
the argument \((1)\Rightarrow(2)\) in the proof of Theorem~\ref{thm:main} (given below) shows that
\(\conn(\Phi^G(Y))\ge q\).
Thus \(\Phi^G(Y)\in \Sp_{\ge q}\), proving \(\mathcal U\subseteq \Sp_{\ge q}\).

\smallskip\noindent
\emph{Step 2: \(\Sp_{\ge q}\subseteq \mathcal U\).}
Choose \(K_{\max}\le_{\cO}G\) with \([G:K_{\max}]=M_{\cO}(G)\); such a subgroup exists by
Definition~\ref{def:MO}.
Write \(M:=M_{\cO}(G)\) and \(n=qM+r\) with \(0\le r<M\).
For each integer \(t\ge q\), define the odd \(\cO\)-slice cell
\[
C_t\ :=\ S^{(t+1)\rho_{G/K_{\max}}-1}\ \in\ \Sp^G.
\]
Its \(\cO\)-slice degree is \((t+1)M-1\), and since \(t\ge q\) we have
\[
(t+1)M-1\ \ge\ (q+1)M-1\ =\ qM+(M-1)\ \ge\ qM+r\ =\ n,
\]
hence \(C_t\in \tau^{G,\cO}_{\ge n}\) by Definition~\ref{def:tau}.

Set \(Y_t:=\tE\cP\wedge C_t\). Then \(Y_t\) is geometric, and
\[
\Phi^G(Y_t)\ \simeq\ \Phi^G(\tE\cP)\wedge \Phi^G(C_t)\ \simeq\ S^0\wedge S^{t}\ \simeq\ S^t,
\]
using Lemma~\ref{lem:PhiG-tildeEP} and the fact that
\(\dim((\rho_{G/K_{\max}})^G)=|G\backslash G/K_{\max}|=1\)
(Lemma~\ref{lem:perm-fixed-dim}).

It remains to prove \(Y_t\in\tau^{G,\cO}_{\ge n}\).
We do \emph{not} assume that smashing with \(\tE\cP\) preserves \(\tau^{G,\cO}_{\ge n}\).
Instead we prove membership via isotropy separation and the inductive hypothesis.

Consider the isotropy separation cofiber sequence \eqref{eq:isosep} for \(X=C_t\):
\[
(E\cP)_+\wedge C_t \longrightarrow C_t \longrightarrow \tE\cP\wedge C_t \ =\ Y_t.
\]
We already know \(C_t\in\tau^{G,\cO}_{\ge n}\).
By Lemma~\ref{lem:cofiber-closure}, it suffices to show \((E\cP)_+\wedge C_t\in\tau^{G,\cO}_{\ge n}\).

The based \(G\)-CW complex \((E\cP)_+\) is built from cells of the form \(G/H_+\wedge S^k\) with \(H<G\) and
\(k\ge 0\).
Smashing with \(C_t\) expresses \((E\cP)_+\wedge C_t\) as a filtered colimit built from extensions of
suspensions of spectra \(G/H_+\wedge C_t\) with \(H<G\).
Since \(\tau^{G,\cO}_{\ge n}\) is closed under suspensions, colimits, and extensions, it suffices to show
\(G/H_+\wedge C_t\in\tau^{G,\cO}_{\ge n}\) for each proper \(H<G\).

For such \(H\), we have \(G/H_+\wedge C_t\simeq \Ind_H^G(\Res^G_H C_t)\).
We claim \(\Res^G_H C_t\in \tau^{H,\cO|_H}_{\ge n}\).
Indeed:
\begin{itemize}[leftmargin=2.2em]
\item \emph{Connectivity:} Write \(V=(t+1)\rho_{G/K_{\max}}\). Then \(\Res^G_H C_t\simeq S^{\Res^G_H V-1}\).
For any subgroup \(L\le H\),
\[
\dim\bigl((\Res^G_H V)^L\bigr)=\dim(V^L)=(t+1)\dim\bigl((\rho_{G/K_{\max}})^L\bigr)
=(t+1)\,|L\backslash G/K_{\max}|\ \ge\ t+1\ \ge\ 1.
\]
Therefore Lemma~\ref{lem:rep-sphere-connective} applies and implies that \(\Res^G_H C_t\) is connective.
\item \emph{Geometric fixed point bounds:} For each \(L\le H\), we have
\(\Phi^L(\Res^G_H C_t)\simeq \Phi^L(C_t)\).
Since \(C_t\) is itself a generating \(\cO\)-slice cell of degree \(\ge n\),
Proposition~\ref{prop:generatorconnectivity} gives
\(
\conn(\Phi^L(C_t))\ge \floor{n/M_{\cO}(L)}.
\)
\end{itemize}
Thus \(\Res^G_H C_t\) satisfies condition \textup{(2)} of Theorem~\ref{thm:main} for the proper subgroup \(H\).
By the inductive hypothesis, we conclude
\(\Res^G_H C_t\in\tau^{H,\cO|_H}_{\ge n}\).
Lemma~\ref{lem:IndPreservesTau} then yields \(G/H_+\wedge C_t\in\tau^{G,\cO}_{\ge n}\), as required.

Therefore \((E\cP)_+\wedge C_t\in\tau^{G,\cO}_{\ge n}\), and hence \(Y_t=\tE\cP\wedge C_t\in\tau^{G,\cO}_{\ge n}\).

Consequently, \(S^t\simeq \Phi^G(Y_t)\in\mathcal U\) for every \(t\ge q\).
Since \(\Sp_{\ge q}=\Loc(\{S^t\mid t\ge q\})\) is the smallest localizing preaisle containing these spheres,
we obtain \(\Sp_{\ge q}\subseteq \mathcal U\).

\smallskip
Combining Steps~1 and~2 yields \(\mathcal U=\Sp_{\ge q}\).
Transporting back along the equivalence \(\Phi^G:\Sp^G_{\mathrm{geom}}\simeq\Sp\) gives the claimed
criterion for geometric \(Y\).
\end{proof}

\subsection{Proof of Theorem~\ref{thm:main}}

\begin{proof}[Proof of Theorem~\ref{thm:main}]
We prove \((1)\Rightarrow(2)\) for all finite \(G\) and then \((2)\Rightarrow(1)\) by induction on \(|G|\).

\smallskip\noindent
\emph{\((1)\Rightarrow(2)\).}
Assume \(X\in\tau^{G,\cO}_{\ge n}\). Fix \(H\le G\).
Since \(\Phi^H\) is exact and preserves colimits (Proposition~\ref{prop:Phi-standard}), the image
\(\Phi^H(\tau^{G,\cO}_{\ge n})\) is contained in the localizing preaisle generated by the spectra
\(\Phi^H(C)\) for \(\cO\)-slice cells \(C\) of degree \(\ge n\).
By Proposition~\ref{prop:generatorconnectivity}, each such \(\Phi^H(C)\) is
\(q_{\cO}(H,n)=\floor{n/M_{\cO}(H)}\)-connective.
Since \(\Sp_{\ge q_{\cO}(H,n)}\) is a localizing preaisle, we conclude
\(\conn(\Phi^H(X))\ge \floor{n/M_{\cO}(H)}\).

\smallskip\noindent
\emph{\((2)\Rightarrow(1)\).}
We argue by induction on \(|G|\).

If \(|G|=1\), then \(\Sp^G\simeq\Sp\), \(M_{\cO}(G)=1\), and
\(\tau^{G,\cO}_{\ge n}=\Sp_{\ge n}\) by Definition~\ref{def:tau}, so the statement is ordinary connectivity.

Assume \(|G|>1\) and that the theorem holds for all proper subgroups of \(G\).
Let \(X\in\Sp^G\) be connective and satisfy condition \textup{(2)}.
Consider isotropy separation \eqref{eq:isosep}.
Since \(\tau^{G,\cO}_{\ge n}\) is extension-closed, it suffices to show that both
\((E\cP)_+\wedge X\) and \(\tE\cP\wedge X\) lie in \(\tau^{G,\cO}_{\ge n}\).

\smallskip\noindent
\emph{Step 1: \((E\cP)_+\wedge X\in\tau^{G,\cO}_{\ge n}\).}
The based \(G\)-CW complex \((E\cP)_+\) is built from cells \(G/H_+\wedge S^k\) with \(H<G\) and \(k\ge 0\).
Smashing with \(X\) expresses \((E\cP)_+\wedge X\) as a filtered colimit built from extensions of suspensions
of spectra \(G/H_+\wedge X\) with \(H<G\).
Since \(\tau^{G,\cO}_{\ge n}\) is closed under colimits, suspensions, and extensions, it suffices to show
\(G/H_+\wedge X\in\tau^{G,\cO}_{\ge n}\) for each proper \(H<G\).

For such \(H\), \(G/H_+\wedge X\simeq \Ind_H^G(\Res^G_H X)\).
The restriction \(\Res^G_H X\) is connective as an \(H\)-spectrum, since for every \(L\le H\),
\((\Res^G_H X)^L \simeq X^L\).
Moreover, for every \(L\le H\),
\(\Phi^L(\Res^G_H X)\simeq \Phi^L(X)\), so condition \textup{(2)} for \(X\) gives
\[
\conn(\Phi^L(\Res^G_H X))\ \ge\ \floor{\frac{n}{M_{\cO}(L)}}.
\]
By the inductive hypothesis applied to \(H\), we obtain \(\Res^G_H X\in\tau^{H,\cO|_H}_{\ge n}\).
Lemma~\ref{lem:IndPreservesTau} then yields \(G/H_+\wedge X\in\tau^{G,\cO}_{\ge n}\), completing Step~1.

\smallskip\noindent
\emph{Step 2: \(\tE\cP\wedge X\in\tau^{G,\cO}_{\ge n}\).}
Set \(Y=\tE\cP\wedge X\). Then \(Y\) is geometric.
Let \(q=\floor{n/M_{\cO}(G)}\).
By Lemma~\ref{lem:PhiG-tildeEP} and monoidality,
\(\Phi^G(Y)\simeq \Phi^G(X)\).
Condition \textup{(2)} with \(H=G\) gives \(\conn(\Phi^G(X))\ge q\), hence \(\conn(\Phi^G(Y))\ge q\).
Applying Lemma~\ref{lem:geometricreduction} shows \(Y\in\tau^{G,\cO}_{\ge n}\), completing Step~2.

Finally, since both ends of \eqref{eq:isosep} lie in \(\tau^{G,\cO}_{\ge n}\),
extension closure gives \(X\in\tau^{G,\cO}_{\ge n}\).
\end{proof}

% ------------------------------------------------------------
\section{Variants and special cases}

\begin{remark}[Regular \(\cO\)-slice filtration and ceil-scaling]\label{rem:regular-scaling}
Define the \emph{regular} \(\cO\)-slice filtration by restricting Definition~\ref{def:O-slice-cells} to
\(\varepsilon=0\), i.e.\ generators \(G_+\wedge_H S^{m\rho_{H/K}}\) only.
The same argument as above (with the obvious modifications) yields the analogous characterization:
for connective \(X\),
\[
X\in \bar\tau^{G,\cO}_{\ge n}
\quad\Longleftrightarrow\quad
\conn(\Phi^H(X))\ge \ceil{\frac{n}{M_{\cO}(H)}}\ \text{ for all }H\le G.
\]
In the complete-transfer case \(M_{\cO}(H)=|H|\), this recovers the Hill--Yarnall criterion for regular
slice connectivity \cite{HY}.
\end{remark}

\begin{remark}[Trivial and complete transfer systems]
If \(\cO\) is trivial, then \(M_{\cO}(H)=1\) and Theorem~\ref{thm:main} becomes:
\[
X\in\tau^{G,\cO}_{\ge n}\ \Longleftrightarrow\ \conn(\Phi^H(X))\ge n\text{ for all }H\le G,
\]
i.e.\ the filtration reduces to simultaneous Postnikov connectivity of all geometric fixed points.
If \(\cO\) is complete, then \(M_{\cO}(H)=|H|\) and the scaling constant specializes to the familiar order
of \(H\).
\end{remark}

% ------------------------------------------------------------
\begin{thebibliography}{99}

\bibitem{BH}
A.~J.~Blumberg and M.~A.~Hill,
\emph{Operadic multiplications in equivariant spectra, norms, and transfers},
Adv.\ Math.\ \textbf{285} (2015), 658--708.
\href{https://arxiv.org/abs/1309.1750}{arXiv:1309.1750}.

\bibitem{HHR}
M.~A.~Hill, M.~J.~Hopkins, and D.~C.~Ravenel,
\emph{On the nonexistence of elements of {K}ervaire invariant one},
Ann.\ of Math.\ (2) \textbf{184} (2016), no.~1, 1--262.
\href{https://arxiv.org/abs/0908.3724}{arXiv:0908.3724}.

\bibitem{HY}
M.~A.~Hill and C.~Yarnall,
\emph{A new formulation of the equivariant slice filtration with applications to \(C_p\)-slices},
Proc.\ Amer.\ Math.\ Soc.\ \textbf{146} (2018), no.~8, 3605--3614.
\href{https://arxiv.org/abs/1703.10526}{arXiv:1703.10526}.

\bibitem{MM}
M.~A.~Mandell and J.~P.~May,
\emph{Equivariant orthogonal spectra and \(S\)-modules},
Mem.\ Amer.\ Math.\ Soc.\ \textbf{159} (2002), no.~755, x+108.

\bibitem{Rub}
J.~Rubin,
\emph{Combinatorial \(N_\infty\) operads},
Algebr.\ Geom.\ Topol.\ \textbf{21} (2021), 3513--3568.
\href{https://arxiv.org/abs/1705.03585}{arXiv:1705.03585}.

\bibitem{Schwede}
S.~Schwede,
\emph{Lectures on equivariant stable homotopy theory},
available at
\href{https://www.math.uni-bonn.de/~schwede/equivariant.pdf}{\texttt{https://www.math.uni-bonn.de/\string~schwede/equivariant.pdf}}.

\bibitem{Wil}
D.~Wilson,
\emph{On categories of slices},
Algebr.\ Geom.\ Topol.\ \textbf{18} (2018), 4633--4706.
\href{https://arxiv.org/abs/1711.03472}{arXiv:1711.03472}.

\end{thebibliography}

\end{document}
