\documentclass[11pt]{amsart}

% ------------------- page layout (expanded margins) -------------------
\usepackage[a4paper,margin=1in]{geometry}

% ------------------- packages -------------------
\usepackage{amsmath,amssymb,amsthm,mathtools}
\usepackage{hyperref}

% ------------------- theorem environments -------------------
\newtheorem{theorem}{Theorem}[section]
\newtheorem{proposition}[theorem]{Proposition}
\newtheorem{lemma}[theorem]{Lemma}
\newtheorem{corollary}[theorem]{Corollary}
\theoremstyle{definition}
\newtheorem{definition}[theorem]{Definition}
\theoremstyle{remark}
\newtheorem{remark}[theorem]{Remark}

% ------------------- macros -------------------
\newcommand{\GL}{\mathrm{GL}}
\newcommand{\Ind}{\mathrm{Ind}}
\newcommand{\cInd}{\mathrm{c\text{-}Ind}}
\newcommand{\diag}{\mathrm{diag}}
\newcommand{\abs}[1]{\left|#1\right|}
\newcommand{\vol}{\mathrm{vol}}

\newcommand{\oo}{\mathfrak{o}}
\newcommand{\pp}{\mathfrak{p}}
\newcommand{\qq}{\mathfrak{q}}
\newcommand{\W}{\mathcal{W}}

\newcommand{\G}[1]{\GL_{#1}(F)}
\newcommand{\N}[1]{N_{#1}}
\newcommand{\K}[1]{K_{#1}}

% ------------------- title data -------------------
\title[Conductor-bounded Whittaker test vectors for shifted Rankin--Selberg integrals]
{Conductor-bounded Whittaker test vectors\\
for shifted local Rankin--Selberg integrals on $\GL_{n+1}\times \GL_n$}
\date{\today}

\subjclass[2020]{11F70, 22E50}
\keywords{Whittaker models, Kirillov models, newforms, conductors, Rankin--Selberg integrals, test vectors}

\begin{document}

\begin{abstract}
Let $F$ be a non-archimedean local field, fix an additive character $\psi$ of conductor $\oo$,
and let $\Pi$ be a generic irreducible admissible representation of $\GL_{n+1}(F)$.
For each integer $m\ge 0$ we construct a Whittaker function $W_m\in \W(\Pi,\psi^{-1})$
depending only on $(\Pi,\psi,m)$ with the following uniform property:
for every generic irreducible admissible representation $\pi$ of $\GL_n(F)$ with conductor exponent
$a(\pi)\le m$, for the normalized Whittaker newform $V\in\W(\pi,\psi)$, and for any generator
$Q$ of the inverse conductor ideal $\qq(\pi)^{-1}$, the shifted local Rankin--Selberg integral
\[
  \int_{\N{n}\backslash \GL_n(F)} W_m(\diag(g,1)\,u_Q)\,V(g)\,\abs{\det g}^{\,s-\frac12}\,dg,
  \qquad u_Q:=I_{n+1}+Q E_{n,n+1},
\]
is absolutely convergent for all $s\in\mathbb{C}$ and evaluates to the explicit nonzero constant
$\psi^{-1}(Q)\cdot \vol(\N{n}(\oo)\backslash K_1(\pp^m))$.
In particular, for each fixed $\pi$ one obtains a nonvanishing pair $(W,V)$ by taking $m=a(\pi)$.
We also record a topological obstruction showing that the ``exact mirabolic'' subgroup in $\GL_n(\oo)$
has empty interior (for all $n\ge 1$), so it cannot support nonzero locally constant functions.
\end{abstract}

\maketitle

\section{Introduction: the shifted integral and the quantifiers}

Let $F$ be a non-archimedean local field with ring of integers $\oo$, maximal ideal $\pp$, and
residue cardinality $q$. Fix a nontrivial additive character $\psi:F\to\mathbb{C}^\times$
of conductor $\oo$ (i.e.\ $\psi$ is trivial on $\oo$ and nontrivial on $\pp^{-1}$).
For $r\ge 1$, let $\N{r}\subset \GL_r(F)$ be the standard upper-triangular unipotent subgroup and
let $\psi_r:\N{r}\to\mathbb{C}^\times$ be the standard generic character
$\psi_r(u)=\psi(\sum_{i=1}^{r-1}u_{i,i+1})$.

Let $\Pi$ be a generic irreducible admissible representation of $\GL_{n+1}(F)$, realized in its
$\psi_{n+1}^{-1}$-Whittaker model $\W(\Pi,\psi_{n+1}^{-1})$.
Let $\pi$ be a generic irreducible admissible representation of $\GL_n(F)$, realized in its
$\psi_n$-Whittaker model $\W(\pi,\psi_n)$.
Let $\qq(\pi)=\pp^{a(\pi)}$ be its conductor ideal (so $a(\pi)\ge 0$ is the conductor exponent),
and choose $Q\in F^\times$ with $Q\oo=\qq(\pi)^{-1}=\pp^{-a(\pi)}$.

Define
\[
u_Q:=I_{n+1}+Q E_{n,n+1}\in \GL_{n+1}(F), \qquad
\iota:\GL_n(F)\hookrightarrow \GL_{n+1}(F),\ \iota(g)=\diag(g,1).
\]
For $W\in\W(\Pi,\psi_{n+1}^{-1})$ and $V\in\W(\pi,\psi_n)$, consider the shifted local
Rankin--Selberg integral
\begin{equation}\label{eq:shifted-RS}
Z(W,V,s;Q):=\int_{\N{n}\backslash \GL_n(F)} W(\iota(g)u_Q)\,V(g)\,\abs{\det g}^{\,s-\frac12}\,dg.
\end{equation}

\smallskip
The natural existence question can be read in (at least) two ways:

\medskip
\noindent\textbf{(Pairwise problem).}
For fixed $(\Pi,\pi)$, must there exist $W\in\W(\Pi,\psi_{n+1}^{-1})$ and $V\in\W(\pi,\psi_n)$
such that $Z(W,V,s;Q)$ is finite and nonzero for all $s\in\mathbb{C}$?

\medskip
\noindent\textbf{(Universal-$W$ problem).}
For fixed $\Pi$, must there exist a single $W\in\W(\Pi,\psi_{n+1}^{-1})$ such that
\emph{for every} generic $\pi$ there exists $V\in\W(\pi,\psi_n)$ with $Z(W,V,s;Q)$ finite and nonzero
for all $s\in\mathbb{C}$?

\medskip
The \emph{universal-$W$ problem} is substantially stronger (it places a uniformity constraint on $W$
across all conductor exponents). The present note establishes a precise, strong
\emph{conductor-bounded} uniformity statement. This resolves the pairwise problem immediately
(by choosing the bound $m=a(\pi)$), and it clarifies exactly where conductor dependence enters.

\begin{theorem}[Main theorem: conductor-bounded uniform test vectors]\label{thm:main}
Fix $n\ge 1$ and a generic irreducible admissible representation $\Pi$ of $\GL_{n+1}(F)$.
For each integer $m\ge 0$ there exists a Whittaker function $W_m\in\W(\Pi,\psi_{n+1}^{-1})$
with the following property:

For every generic irreducible admissible representation $\pi$ of $\GL_n(F)$ with conductor exponent
$a(\pi)\le m$, let $V\in\W(\pi,\psi_n)$ be the normalized Whittaker newform and let
$Q\in F^\times$ satisfy $Q\oo=\pp^{-a(\pi)}$.
Then the shifted integral \eqref{eq:shifted-RS} is absolutely convergent for every $s\in\mathbb{C}$ and equals
\[
Z(W_m,V,s;Q)=\psi^{-1}(Q)\cdot \vol\!\bigl(\N{n}(\oo)\backslash K_1(\pp^m)\bigr),
\]
a nonzero constant independent of $s$.
\end{theorem}

\begin{corollary}[Resolution of the pairwise existence problem]\label{cor:pairwise}
For every pair $(\Pi,\pi)$ as above, taking $m=a(\pi)$ and $W=W_m$ from Theorem~\ref{thm:main},
and taking $V$ to be the normalized Whittaker newform of $\pi$, the integral $Z(W,V,s;Q)$ is finite and nonzero
for all $s\in\mathbb{C}$.
\end{corollary}

The construction of $W_m$ uses only standard features of the Kirillov model
(restriction of Whittaker functions to the mirabolic subgroup) and an elementary ``phase collapse''
computation forced by the inequality $m\ge a(\pi)$.

\section{Notation}

Fix Haar measures once and for all; the numerical value of the volume terms depends on these choices,
but positivity and nonvanishing do not.

Let $\K{n}=\GL_n(\oo)$. For $m\ge 0$ define the usual congruence subgroup
\begin{equation}\label{eq:K1}
K_1(\pp^m):=\left\{k\in \K{n}:\ e_n k\equiv e_n \pmod{\pp^m}\right\},\qquad e_n=(0,\dots,0,1)\in F^n.
\end{equation}
Equivalently, $k\in K_1(\pp^m)$ if and only if $k_{n,j}\in \pp^m$ for $1\le j<n$ and $k_{n,n}\in 1+\pp^m$.
Each $K_1(\pp^m)$ is open and compact in $\GL_n(F)$.

\section{A smoothness obstruction: the exact mirabolic has empty interior}

This section is not used in the proof of Theorem~\ref{thm:main}, but it explains why one cannot hope to build
a compactly supported test function by insisting the last row be \emph{exactly} $e_n$.

\begin{definition}
Let $n\ge 1$. The \emph{exact mirabolic} subgroup of $\K{n}=\GL_n(\oo)$ is
\[
K_n^{\mathrm{mir}}:=\{k\in \K{n}:\ e_n k = e_n\}.
\]
\end{definition}

\begin{lemma}\label{lem:mir-empty-interior}
For every $n\ge 1$, the subgroup $K_n^{\mathrm{mir}}$ is closed and has empty interior in $\K{n}$.
For $n\ge 2$ one moreover has
\[
K_n^{\mathrm{mir}}=\bigcap_{m\ge 1}K_1(\pp^m).
\]
\end{lemma}

\begin{proof}
For $n\ge 2$, the condition $e_nk=e_n$ is equivalent to the congruences
$e_nk\equiv e_n\pmod{\pp^m}$ for all $m\ge 1$, because $\bigcap_{m\ge 1}\pp^m=\{0\}$ in $\oo$.
This gives $K_n^{\mathrm{mir}}=\bigcap_{m\ge 1}K_1(\pp^m)$, hence $K_n^{\mathrm{mir}}$ is closed.

Each $K_1(\pp^m)$ is open. For $n\ge 2$ the inclusions are strict, so the intersection cannot contain any open neighborhood
(of the identity, hence of any point). Thus it has empty interior.

For $n=1$, one has $\K{1}=\oo^\times$ and $K_1^{\mathrm{mir}}=\{1\}$, which is closed but not open in the non-discrete compact group
$\oo^\times$. Hence it also has empty interior.
\end{proof}

\begin{proposition}\label{prop:no-locally-constant-support}
Let $G$ be a totally disconnected locally compact group and let $H\subseteq G$ be a closed subset with empty interior.
Then every locally constant function $f:G\to\mathbb{C}$ supported on $H$ is identically zero.
\end{proposition}

\begin{proof}
If $f(x)\neq 0$ at some $x$, then local constancy gives an open neighborhood $U$ of $x$ on which $f$ is constant and nonzero,
hence $U\subseteq \mathrm{supp}(f)\subseteq H$, contradicting that $H$ has empty interior.
\end{proof}

\section{The shift $u_Q$ and the Whittaker character computation}

\begin{lemma}[Shift factor]\label{lem:shift-factor}
Let $W\in \W(\Pi,\psi_{n+1}^{-1})$. For all $g\in\GL_n(F)$ and all $Q\in F$,
\[
W(\diag(g,1)\,u_Q)=\psi^{-1}(Q\,g_{n,n})\,W(\diag(g,1)).
\]
\end{lemma}

\begin{proof}
Set $\iota(g)=\diag(g,1)$.
Write
\[
\iota(g)\,u_Q=\bigl(\iota(g)\,u_Q\,\iota(g)^{-1}\bigr)\,\iota(g).
\]
Since $u_Q=I_{n+1}+Q E_{n,n+1}$, a direct computation gives
\[
\iota(g)\,E_{n,n+1}\,\iota(g)^{-1}=\sum_{i=1}^n g_{i,n}\,E_{i,n+1},
\]
hence
\[
x:=\iota(g)\,u_Q\,\iota(g)^{-1}
=I_{n+1}+Q\sum_{i=1}^n g_{i,n}E_{i,n+1}\in \N{n+1}.
\]
The generic character $\psi_{n+1}$ depends only on the superdiagonal entries $(i,i+1)$.
Among the matrix units $E_{i,n+1}$, only $E_{n,n+1}$ lies on the superdiagonal; thus
\[
\psi_{n+1}(x)=\psi(Q\,g_{n,n}).
\]
Using Whittaker equivariance $W(ux)=\psi_{n+1}^{-1}(u)\,W(x)$ for $u\in \N{n+1}$ yields
\[
W(\iota(g)u_Q)=W(x\,\iota(g))=\psi_{n+1}^{-1}(x)\,W(\iota(g))=\psi^{-1}(Q\,g_{n,n})\,W(\iota(g)),
\]
as claimed.
\end{proof}

\section{Kirillov models and extension of compactly supported data}

Let $P_{n+1}\subset \GL_{n+1}(F)$ be the \emph{mirabolic} subgroup (matrices whose last row equals $e_{n+1}$).
Every element of $P_{n+1}$ has a unique block decomposition
\[
p=\begin{pmatrix} g & x \\ 0 & 1\end{pmatrix},\qquad g\in \GL_n(F),\ x\in F^n.
\]

\begin{definition}
Let $H\subseteq G$ be closed and $\chi$ a character of $H$.
We write $\cInd_H^G(\chi)$ for the space of locally constant functions $f:G\to\mathbb{C}$
such that $f(hg)=\chi(h)f(g)$ for $h\in H$, and whose support is compact modulo $H$.
\end{definition}

The following identification is a standard and elementary observation.

\begin{lemma}\label{lem:cind-identification}
Restriction to the Levi embedding $\iota:\GL_n(F)\hookrightarrow P_{n+1}$ induces an isomorphism
\[
\cInd_{\N{n+1}}^{P_{n+1}}(\psi_{n+1}^{-1})\ \xrightarrow{\ \sim\ }\ \cInd_{\N{n}}^{\GL_n(F)}(\psi_n^{-1}),
\qquad F\longmapsto \bigl(g\mapsto F(\diag(g,1))\bigr).
\]
\end{lemma}

\begin{proof}
Given $f\in \cInd_{\N{n}}^{\GL_n(F)}(\psi_n^{-1})$, define $F:P_{n+1}\to\mathbb{C}$ by
\[
F\!\left(\begin{pmatrix} g & x \\ 0 & 1\end{pmatrix}\right):=\psi^{-1}(x_n)\,f(g),
\]
where $x_n$ is the last coordinate of the column vector $x$.
One checks directly that $F$ is locally constant, compactly supported modulo $\N{n+1}$ if and only if
$f$ is compactly supported modulo $\N{n}$, and that $F$ satisfies $F(up)=\psi_{n+1}^{-1}(u)F(p)$ for all $u\in\N{n+1}$.
Moreover, $F(\diag(g,1))=f(g)$. This constructs an inverse to restriction, and the two maps are inverse isomorphisms.
\end{proof}

We now invoke the Kirillov model theorem for $p$-adic $\GL_{n+1}$ in the precise form needed here.

\begin{theorem}[Gelfand--Kazhdan; Kirillov model contains compact induction]\label{thm:GK}
Let $\Pi$ be a generic irreducible admissible representation of $\GL_{n+1}(F)$ and realize it in its Whittaker model
$\W(\Pi,\psi_{n+1}^{-1})\subset \Ind_{\N{n+1}}^{\GL_{n+1}(F)}(\psi_{n+1}^{-1})$.
Then the restriction map to the mirabolic subgroup,
\[
\mathrm{Res}_{P_{n+1}}:\W(\Pi,\psi_{n+1}^{-1})\longrightarrow \Ind_{\N{n+1}}^{P_{n+1}}(\psi_{n+1}^{-1}),
\qquad W\mapsto W|_{P_{n+1}},
\]
has image containing the compact induction $\cInd_{\N{n+1}}^{P_{n+1}}(\psi_{n+1}^{-1})$.
\end{theorem}

\begin{proof}[Proof (reference)]
This is a classical Kirillov model statement for $p$-adic $\GL_{n+1}$ due to Gelfand--Kazhdan \cite{GK75}.
For a modern discussion of Kirillov models via Bernstein--Zelevinsky derivatives and restriction to mirabolic subgroups,
see also \cite[\S 3]{KM18}. (We do not reprove the Kirillov model theorem here.)
\end{proof}

\begin{corollary}[Extension to the embedded Levi]\label{cor:extension-to-Levi}
For every $f\in \cInd_{\N{n}}^{\GL_n(F)}(\psi_n^{-1})$ there exists $W\in \W(\Pi,\psi_{n+1}^{-1})$ such that
\[
W(\diag(g,1))=f(g)\qquad \text{for all }g\in \GL_n(F).
\]
\end{corollary}

\begin{proof}
By Lemma~\ref{lem:cind-identification}, $f$ corresponds to a function $F\in \cInd_{\N{n+1}}^{P_{n+1}}(\psi_{n+1}^{-1})$
whose restriction to $\diag(\GL_n(F),1)$ is $f$. By Theorem~\ref{thm:GK} there exists
$W\in\W(\Pi,\psi_{n+1}^{-1})$ with $W|_{P_{n+1}}=F$, hence $W(\diag(g,1))=f(g)$ for all $g$.
\end{proof}

\section{A conductor-bounded test function on $\GL_n$ and its lift to $\GL_{n+1}$}

Fix $m\ge 0$. Define $f_m:\GL_n(F)\to\mathbb{C}$ by
\begin{equation}\label{eq:def-fm}
f_m(uk)=\psi_n^{-1}(u)\quad (u\in \N{n},\ k\in K_1(\pp^m)),\qquad
f_m(g)=0\quad \text{if }g\notin \N{n}K_1(\pp^m).
\end{equation}

\begin{lemma}\label{lem:fm-well-defined}
The function $f_m$ is well-defined and lies in $\cInd_{\N{n}}^{\GL_n(F)}(\psi_n^{-1})$.
Moreover, $\N{n}\cap K_1(\pp^m)=\N{n}(\oo)$.
\end{lemma}

\begin{proof}
If $uk=u'k'$ with $u,u'\in\N{n}$ and $k,k'\in K_1(\pp^m)$, then $k^{-1}k'=u^{-1}u'\in \N{n}\cap K_1(\pp^m)$.
Since $K_1(\pp^m)\subseteq \GL_n(\oo)$, we have $\N{n}\cap K_1(\pp^m)\subseteq \N{n}(\oo)$.
Conversely, any $u\in\N{n}(\oo)$ has last row $e_n$, hence lies in $K_1(\pp^m)$, so $\N{n}(\oo)\subseteq \N{n}\cap K_1(\pp^m)$.
Thus $\N{n}\cap K_1(\pp^m)=\N{n}(\oo)$.

Because $\psi$ has conductor $\oo$, it is trivial on $\oo$, hence $\psi_n$ is trivial on $\N{n}(\oo)$.
Therefore $\psi_n^{-1}(u)=\psi_n^{-1}(u')$, and \eqref{eq:def-fm} is well-defined.

The support of $f_m$ modulo $\N{n}$ is contained in $K_1(\pp^m)$, which is compact, and $f_m$ is locally constant
(right $K_1(\pp^m)$-invariant). Hence $f_m\in \cInd_{\N{n}}^{\GL_n(F)}(\psi_n^{-1})$.
\end{proof}

\begin{proposition}\label{prop:Wm-exists}
For each $m\ge 0$ there exists $W_m\in\W(\Pi,\psi_{n+1}^{-1})$ such that
\[
W_m(\diag(g,1))=f_m(g)\qquad (g\in \GL_n(F)).
\]
\end{proposition}

\begin{proof}
Apply Corollary~\ref{cor:extension-to-Levi} to $f=f_m$.
\end{proof}

\section{Whittaker newforms on $\GL_n$}

Let $\pi$ be a generic irreducible admissible representation of $\GL_n(F)$ with conductor ideal $\qq(\pi)=\pp^{a(\pi)}$.

\begin{proposition}[Whittaker newforms]\label{prop:newform}
There exists $V\in\W(\pi,\psi_n)$ (unique up to scalars) that is right $K_1(\pp^{a(\pi)})$-invariant.
Moreover, one may normalize it so that $V(1)=1$, and then $V(k)=1$ for all $k\in K_1(\pp^{a(\pi)})$.
\end{proposition}

\begin{proof}[Proof (reference)]
The existence and uniqueness (up to scalars) of the $K_1(\pp^{a(\pi)})$-fixed newvector is standard local newform theory
for $\GL_n$ \cite{JPSS81}. Translating the newvector to the Whittaker model and normalizing $V(1)=1$ requires that the
corresponding Whittaker newform does not vanish at the identity, a nontrivial fact proved in explicit newform computations;
see, e.g., \cite{Miyauchi14} and \cite{Matringe13}. For such a normalized $V$, right invariance implies $V(k)=V(1)=1$ on $K_1(\pp^{a(\pi)})$.
\end{proof}

\section{Phase collapse and evaluation of the shifted integral}

\begin{lemma}\label{lem:psi-constant}
Let $m\ge 0$ and let $a\ge 0$. Suppose $Q\in \pp^{-a}$.
If $m\ge a$ and $k\in K_1(\pp^m)$, then
\[
\psi(Q\,k_{n,n})=\psi(Q).
\]
\end{lemma}

\begin{proof}
For $k\in K_1(\pp^m)$ we have $k_{n,n}\equiv 1\pmod{\pp^m}$, hence $k_{n,n}=1+t$ with $t\in \pp^m$.
Then $Qk_{n,n}=Q+Qt$ with $Qt\in \pp^{m-a}\subseteq \oo$ because $m\ge a$.
Since $\psi$ is trivial on $\oo$, we get $\psi(Q+Qt)=\psi(Q)$.
\end{proof}

\begin{proof}[Proof of Theorem~\ref{thm:main}]
Let $W_m$ be as in Proposition~\ref{prop:Wm-exists} and $V$ be the normalized Whittaker newform of $\pi$.
By Lemma~\ref{lem:shift-factor} and the defining property of $W_m$,
\[
W_m(\diag(g,1)u_Q)=\psi^{-1}(Q g_{n,n})\,W_m(\diag(g,1))
=\psi^{-1}(Q g_{n,n})\,f_m(g).
\]
Hence the integrand in \eqref{eq:shifted-RS} is supported on $\N{n}K_1(\pp^m)$.
Since $K_1(\pp^m)$ is compact, the quotient $\N{n}\backslash \N{n}K_1(\pp^m)$ is compact; thus the integral is absolutely
convergent for all $s\in\mathbb{C}$.

On the support, write $g=uk$ with $u\in\N{n}$ and $k\in K_1(\pp^m)$.
Because the last row of $u$ is $e_n$, one has $(uk)_{n,n}=k_{n,n}$.
Moreover $\det(u)=1$ and $k\in \GL_n(\oo)$, hence $\abs{\det g}=\abs{\det k}=1$, so $\abs{\det g}^{s-\frac12}\equiv 1$.

Using the definition of $f_m$ and the Whittaker transformation of $V$,
\[
f_m(uk)=\psi_n^{-1}(u),\qquad V(uk)=\psi_n(u)\,V(k),
\]
so $f_m(uk)\,V(uk)=V(k)$.
Also Lemma~\ref{lem:psi-constant} (with $a=a(\pi)\le m$) gives $\psi(Qk_{n,n})=\psi(Q)$, hence
$\psi^{-1}(Q(uk)_{n,n})=\psi^{-1}(Q)$.
Therefore, on $\N{n}K_1(\pp^m)$ the integrand equals the constant $\psi^{-1}(Q)\,V(k)$.

Since $\N{n}\cap K_1(\pp^m)=\N{n}(\oo)$ by Lemma~\ref{lem:fm-well-defined}, we have the standard identification
\[
\N{n}\backslash \N{n}K_1(\pp^m)\ \simeq\ (\N{n}\cap K_1(\pp^m))\backslash K_1(\pp^m)
\ =\ \N{n}(\oo)\backslash K_1(\pp^m),
\]
and hence
\[
Z(W_m,V,s;Q)=\psi^{-1}(Q)\int_{\N{n}(\oo)\backslash K_1(\pp^m)} V(k)\,dk.
\]
Finally, since $a(\pi)\le m$, we have $K_1(\pp^m)\subseteq K_1(\pp^{a(\pi)})$, so $V$ is right $K_1(\pp^m)$-invariant and normalized by $V(1)=1$;
thus $V(k)=1$ for all $k\in K_1(\pp^m)$. Therefore
\[
Z(W_m,V,s;Q)=\psi^{-1}(Q)\cdot \vol\!\bigl(\N{n}(\oo)\backslash K_1(\pp^m)\bigr),
\]
which is a nonzero constant (a nonzero complex unit times a strictly positive real volume).
\end{proof}

\begin{proof}[Proof of Corollary~\ref{cor:pairwise}]
Given $\pi$, take $m=a(\pi)$ and apply Theorem~\ref{thm:main}.
\end{proof}

\begin{remark}[On Haar measures]
The constant $\vol(\N{n}(\oo)\backslash K_1(\pp^m))$ depends on Haar normalizations.
However, it is always finite and strictly positive; hence the nonvanishing is independent of the choice of measures.
\end{remark}

\begin{remark}[On the universal-$W$ problem]
Theorem~\ref{thm:main} produces a family $\{W_m\}_{m\ge 0}$ uniform for representations $\pi$ with $a(\pi)\le m$.
It does \emph{not} assert the existence of a single $W$ working simultaneously for all conductors.
The topological obstruction in \S 3 shows why naive attempts to force compact support by restricting to an ``exact mirabolic''
subgroup cannot work in the smooth setting.
\end{remark}

% ------------------- bibliography -------------------
\begin{thebibliography}{99}

\bibitem{BZ77}
I.\,N.~Bernstein and A.\,V.~Zelevinsky,
\emph{Induced representations of reductive $p$-adic groups. I},
Ann.\ Sci.\ \'Ec.\ Norm.\ Sup.\ (4) \textbf{10} (1977), 441--472.

\bibitem{GK75}
I.\,M.~Gelfand and D.\,A.~Kazhdan,
\emph{Representations of the group $\GL(n,K)$ where $K$ is a local field},
in: \emph{Lie Groups and Their Representations} (Proc.\ Summer School, Bolyai J\'anos Math.\ Soc., Budapest, 1971),
Halsted Press, New York, 1975, pp.~95--118.

\bibitem{JPSS81}
H.~Jacquet, I.\,I.~Piatetski-Shapiro, and J.~Shalika,
\emph{Conducteur des repr\'esentations du groupe lin\'eaire},
Math.\ Ann.\ \textbf{256} (1981), 199--214.

\bibitem{KM18}
R.~Kurinczuk and N.~Matringe,
\emph{Extension of Whittaker functions and test vectors},
Research in Number Theory \textbf{4} (2018), Article~31.

\bibitem{Matringe13}
N.~Matringe,
\emph{Essential Whittaker functions for $\GL(n)$},
Documenta Math.\ \textbf{18} (2013), 1191--1214.

\bibitem{Miyauchi14}
M.~Miyauchi,
\emph{Whittaker functions associated to newforms for $\GL(n)$ over $p$-adic fields},
J.\ Math.\ Soc.\ Japan \textbf{66} (2014), no.~1, 17--24.

\end{thebibliography}

\end{document}
