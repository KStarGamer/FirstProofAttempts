\documentclass[11pt]{amsart}
\usepackage[a4paper,margin=1in]{geometry}

\usepackage{amsmath,amssymb,amsthm,mathtools}
\usepackage[hidelinks]{hyperref}

% --- theorem environments
\newtheorem{theorem}{Theorem}[section]
\newtheorem{proposition}[theorem]{Proposition}
\newtheorem{lemma}[theorem]{Lemma}
\newtheorem{corollary}[theorem]{Corollary}
\theoremstyle{definition}
\newtheorem{definition}[theorem]{Definition}
\theoremstyle{remark}
\newtheorem{remark}[theorem]{Remark}

% --- operators and macros
\DeclareMathOperator{\rank}{rank}
\DeclareMathOperator{\colsp}{colsp}
\DeclareMathOperator{\spanop}{span}
\newcommand{\R}{\mathbb{R}}
\newcommand{\A}{\mathbb{A}}
\newcommand{\odotop}{\odot}

\title[Determinantal tests for separable scalings]{Universal determinantal tests for separable block scalings of stacked $4$-view determinant tensors}
\date{\today}

\begin{document}

\begin{abstract}
Fix $n\ge 5$ and let $A^{(1)},\dots,A^{(n)}\in \R^{3\times 4}$ be Zariski-generic.
For each quadruple $(\alpha,\beta,\gamma,\delta)\in[n]^4$ we form a tensor
$Q^{(\alpha\beta\gamma\delta)}\in \R^{3\times 3\times 3\times 3}$ whose entries are $4\times 4$ determinants of selected rows of
$A^{(\alpha)},A^{(\beta)},A^{(\gamma)},A^{(\delta)}$.
Given a blockwise scaling tensor $\lambda\in \R^{n\times n\times n\times n}$ that vanishes precisely on the diagonal
$\{(m,m,m,m)\}$, we construct a \emph{universal} family of polynomial relations, with degree bounded independently of $n$, that
vanish on the scaled family $\big(\lambda_{\alpha\beta\gamma\delta}Q^{(\alpha\beta\gamma\delta)}\big)$ if and only if
$\lambda$ factors as a pure tensor $u\otimes v\otimes w\otimes x$ on the off-diagonal.
Concretely, we stack the blocks into a $(3n)^4$-tensor $\mathcal Z$ and take $\mathbf F_n$ to be the vector of all $5\times 5$
minors of the four standard mode-flattenings of $\mathcal Z$. Each coordinate has degree $5$ (independent of $n$), and for
Zariski-generic cameras $\mathbf F_n(\mathcal Z)=0$ holds if and only if
$\lambda_{\alpha\beta\gamma\delta}=u_\alpha v_\beta w_\gamma x_\delta$ for all $(\alpha,\beta,\gamma,\delta)\neq(m,m,m,m)$.
\end{abstract}

\maketitle

\section{Introduction}

Let $n\ge 5$. For each $\alpha\in[n]:=\{1,\dots,n\}$, let $A^{(\alpha)}\in \R^{3\times 4}$.
For $\alpha,\beta,\gamma,\delta\in[n]$, define $Q^{(\alpha\beta\gamma\delta)}\in \R^{3\times 3\times 3\times 3}$ by
\begin{equation}\label{eq:defQ}
Q^{(\alpha\beta\gamma\delta)}_{ijkl}
=\det
\begin{bmatrix}
A^{(\alpha)}(i,:)\\
A^{(\beta)}(j,:)\\
A^{(\gamma)}(k,:)\\
A^{(\delta)}(\ell,:)
\end{bmatrix}
\qquad (1\le i,j,k,\ell\le 3).
\end{equation}
Write the $i$th row of $A^{(\alpha)}$ as a row vector $a_{\alpha i}\in\R^4$.

We are interested in universal algebraic relations on the family $\{Q^{(\alpha\beta\gamma\delta)}\}$ under blockwise scaling.
Let $\lambda\in \R^{n\times n\times n\times n}$ satisfy
\begin{equation}\label{eq:lambdaSupport}
\lambda_{\alpha\beta\gamma\delta}\ne 0
\ \ \Longleftrightarrow\ \
(\alpha,\beta,\gamma,\delta)\neq (m,m,m,m)\ \text{ for all }m\in[n].
\end{equation}
Define $Z^{(\alpha\beta\gamma\delta)}:=\lambda_{\alpha\beta\gamma\delta}Q^{(\alpha\beta\gamma\delta)}$.
The diagonal blocks $Q^{(mmmm)}$ vanish identically:
\begin{equation}\label{eq:diagQzero}
Q^{(mmmm)}\equiv 0\qquad\text{for all }m\in[n],
\end{equation}
since any $4\times 4$ matrix made from four rows of a $3\times 4$ matrix repeats a row.
Thus the values $\lambda_{mmmm}$ do not affect the scaled data, and the factorization condition is meaningful only off the diagonal.

\medskip
\noindent\textbf{Problem.}
Does there exist a polynomial map $\mathbf F:\R^{81n^4}\to\R^N$ (for some $N$) such that:
\begin{enumerate}
\item $\mathbf F$ does not depend on $A^{(1)},\dots,A^{(n)}$;
\item the degrees of the coordinate functions of $\mathbf F$ do not depend on $n$;
\item for $\lambda$ satisfying \eqref{eq:lambdaSupport},
\[
\mathbf F\big(\lambda_{\alpha\beta\gamma\delta}Q^{(\alpha\beta\gamma\delta)}\big)=0
\ \Longleftrightarrow\
\exists\,u,v,w,x\in(\R^\ast)^n:\ 
\lambda_{\alpha\beta\gamma\delta}=u_\alpha v_\beta w_\gamma x_\delta
\ \text{ for all off-diagonal }(\alpha,\beta,\gamma,\delta).
\]
\end{enumerate}

We answer this in the affirmative, by a bounded-degree determinantal test on the mode-flattenings of an explicitly constructed
stacked tensor.

\section{Stacking and flattenings}\label{sec:stack}

\subsection{Stacked tensor}

Let $I:=[n]\times\{1,2,3\}$, so $|I|=3n$.
Given tensors $T^{(\alpha\beta\gamma\delta)}\in\R^{3\times 3\times 3\times 3}$, define the stacked tensor
$\mathcal T\in\R^{I\times I\times I\times I}\cong \R^{(3n)^4}$ by
\begin{equation}\label{eq:stack}
\mathcal T_{(\alpha,i),(\beta,j),(\gamma,k),(\delta,\ell)}
:=T^{(\alpha\beta\gamma\delta)}_{ijkl}.
\end{equation}
Stacking $\{Q^{(\alpha\beta\gamma\delta)}\}$ gives $\mathcal Q$, and stacking $\{Z^{(\alpha\beta\gamma\delta)}\}$ gives $\mathcal Z$.

\subsection{Mode-flattenings and determinantal rank test}

For $r\in\{1,2,3,4\}$, let $M_r(\mathcal T)$ denote the standard mode-$r$ flattening of $\mathcal T$, i.e.\ the matrix obtained by using the
$r$th index of $\mathcal T$ as the row index and concatenating the other three indices as the column index. Thus
$M_r(\mathcal T)\in \R^{(3n)\times (3n)^3}$.

We use the basic determinantal criterion:

\begin{lemma}\label{lem:rankDet}
Let $B$ be a matrix over a field. Then $\rank(B)\le 4$ if and only if every $5\times 5$ minor of $B$ vanishes.
\end{lemma}
\begin{proof}
A matrix has rank at least $5$ if and only if it contains a nonsingular $5\times 5$ submatrix.
\end{proof}

\section{The polynomial map \texorpdfstring{$\mathbf F_n$}{Fn}}\label{sec:F}

For fixed $n$, define $\mathbf F_n:\R^{(3n)^4}\cong \R^{81n^4}\to\R^{N_n}$ to be the polynomial map whose coordinates are \emph{all}
$5\times 5$ minors of each of the four flattenings $M_1(\mathcal T),M_2(\mathcal T),M_3(\mathcal T),M_4(\mathcal T)$ of the input tensor
$\mathcal T$. (Here $N_n$ is the total number of such minors; its value is irrelevant.)

\begin{proposition}\label{prop:F12}
For each $n$, the map $\mathbf F_n$ does not depend on $A^{(1)},\dots,A^{(n)}$, and each coordinate of $\mathbf F_n$ is homogeneous of degree $5$.
In particular, the degree bound is independent of $n$.
\end{proposition}
\begin{proof}
Each coordinate is a $5\times 5$ determinant in the entries of a flattening of the input tensor, hence is a homogeneous polynomial of degree $5$ with coefficients in $\{0,\pm 1\}$.
\end{proof}

\section{A Hodge-star factorization}\label{sec:hodge}

\subsection{Definition of the $\star$-map}

Fix the standard dot product on $\R^4$.
For any fixed $b,c,d\in\R^4$, the map $a\mapsto \det[a;b;c;d]$ is linear in $a$, hence there is a unique vector
$\star(b\wedge c\wedge d)\in \R^4$ such that
\begin{equation}\label{eq:hodge}
\det\begin{bmatrix}a\\ b\\ c\\ d\end{bmatrix}
= a\cdot \star(b\wedge c\wedge d)
\qquad\text{for all }a\in\R^4.
\end{equation}
This defines a linear map $\star:\Lambda^3(\R^4)\to \R^4$.

\subsection{Block matrices $W_{\beta\gamma\delta}$}

For each triple $(\beta,\gamma,\delta)\in[n]^3$, define the matrix
\[
W_{\beta\gamma\delta}\in \R^{4\times 27}
\]
whose columns are the vectors $\star(a_{\beta j}\wedge a_{\gamma k}\wedge a_{\delta\ell})$ for all $(j,k,\ell)\in\{1,2,3\}^3$ in some fixed order.

\begin{lemma}[Block formula]\label{lem:blockFormula}
Let $\mathcal Q$ be the stacked tensor of $\{Q^{(\alpha\beta\gamma\delta)}\}$.
For each $(\alpha;\beta,\gamma,\delta)$ the $(3\times 27)$ block of $M_1(\mathcal Q)$ indexed by row-camera $\alpha$ and column-triple $(\beta,\gamma,\delta)$ equals
\begin{equation}\label{eq:blockM1Q}
\big(M_1(\mathcal Q)\big)_{\alpha;\beta\gamma\delta}
= A^{(\alpha)}\,W_{\beta\gamma\delta}.
\end{equation}
Consequently, for $\mathcal Z=\lambda\odotop \mathcal Q$,
\begin{equation}\label{eq:blockM1Z}
\big(M_1(\mathcal Z)\big)_{\alpha;\beta\gamma\delta}
= \lambda_{\alpha\beta\gamma\delta}\,A^{(\alpha)}\,W_{\beta\gamma\delta}.
\end{equation}
\end{lemma}
\begin{proof}
Fix $\alpha,\beta,\gamma,\delta$. The entry in row $i$ and column $(j,k,\ell)$ of $A^{(\alpha)}W_{\beta\gamma\delta}$ equals
$a_{\alpha i}\cdot \star(a_{\beta j}\wedge a_{\gamma k}\wedge a_{\delta\ell})$, which is
$\det[a_{\alpha i};a_{\beta j};a_{\gamma k};a_{\delta\ell}]$ by \eqref{eq:hodge}, i.e.\ $Q^{(\alpha\beta\gamma\delta)}_{ijkl}$.
Stacking yields \eqref{eq:blockM1Q}. Scaling yields \eqref{eq:blockM1Z}.
\end{proof}

\begin{corollary}\label{cor:rankBoundQ}
For any cameras $A^{(1)},\dots,A^{(n)}$, $\rank(M_1(\mathcal Q))\le 4$.
By symmetry, $\rank(M_r(\mathcal Q))\le 4$ for $r=1,2,3,4$.
\end{corollary}
\begin{proof}
Equation \eqref{eq:blockM1Q} gives the factorization
\[
M_1(\mathcal Q)
=
\underbrace{\begin{bmatrix}A^{(1)}\\ \vdots\\ A^{(n)}\end{bmatrix}}_{3n\times 4}
\cdot
\underbrace{\big[\,W_{\beta\gamma\delta}\,\big]_{(\beta,\gamma,\delta)\in[n]^3}}_{4\times 27n^3},
\]
hence $\rank(M_1(\mathcal Q))\le 4$. The other modes follow by permuting the four indices in \eqref{eq:defQ}.
\end{proof}

\section{Genericity: kernel vectors and ranks of \texorpdfstring{$W_{\beta\gamma\delta}$}{W}}\label{sec:generic}

\subsection{Zariski-genericity}

\begin{definition}
A property holds for \emph{Zariski-generic} $(A^{(1)},\dots,A^{(n)})\in (\R^{3\times 4})^n$ if it holds on the complement of a proper Zariski-closed subset of the affine space $(\R^{3\times 4})^n\cong \R^{12n}$.
\end{definition}

\subsection{Kernel vectors}

Define
\begin{equation}\label{eq:zalpha}
z_\alpha := \star(a_{\alpha 1}\wedge a_{\alpha 2}\wedge a_{\alpha 3})\in\R^4.
\end{equation}
Then $A^{(\alpha)}z_\alpha=0$ by \eqref{eq:hodge}. If $\rank(A^{(\alpha)})=3$ then $\ker(A^{(\alpha)})=\spanop\{z_\alpha\}$.

\begin{lemma}\label{lem:genericKernels}
There exists a nonempty Zariski-open subset $U_n^{\mathrm{ker}}\subset (\R^{3\times 4})^n$ such that for all $(A^{(1)},\dots,A^{(n)})\in U_n^{\mathrm{ker}}$:
\begin{enumerate}
\item $\rank(A^{(\alpha)})=3$ and $\ker(A^{(\alpha)})=\spanop\{z_\alpha\}$ for all $\alpha$;
\item for $\alpha\neq \alpha'$, the kernel lines $\spanop\{z_\alpha\}$ and $\spanop\{z_{\alpha'}\}$ are distinct;
\item for any three distinct indices $\alpha_1,\alpha_2,\alpha_3$, the vectors $z_{\alpha_1},z_{\alpha_2},z_{\alpha_3}$ are linearly independent.
\end{enumerate}
\end{lemma}
\begin{proof}
Each item defines a Zariski-open condition (nonvanishing of appropriate minors).
Nonemptiness is witnessed by the moment-curve family: choose distinct $t_1,\dots,t_n\in\R$ and set
\[
A(t):=
\begin{bmatrix}
-t & 1 & 0 & 0\\
-t^2 & 0 & 1 & 0\\
-t^3 & 0 & 0 & 1
\end{bmatrix}.
\]
Then $\rank(A(t))=3$ and $\ker(A(t))=\spanop\{(1,t,t^2,t^3)^T\}$, and any three such vectors are independent for distinct parameters (Vandermonde).
\end{proof}

\subsection{Ranks of $W_{\beta\gamma\delta}$}

\begin{lemma}\label{lem:rankW}
Fix $(\beta,\gamma,\delta)\in[n]^3$.
\begin{enumerate}
\item If $\beta=\gamma=\delta=m$, then $\rank(W_{mmm})=1$ and $\colsp(W_{mmm})=\spanop\{z_m\}$.
\item If $(\beta,\gamma,\delta)$ are not all equal, then the condition $\rank(W_{\beta\gamma\delta})=4$ is Zariski-open and nonempty (hence holds for Zariski-generic cameras).
\end{enumerate}
\end{lemma}

\begin{proof}
(1) If $\beta=\gamma=\delta=m$, then each column of $W_{mmm}$ is either $0$ (when two of $j,k,\ell$ coincide) or
$\pm \star(a_{m1}\wedge a_{m2}\wedge a_{m3})=\pm z_m$ (when $(j,k,\ell)$ is a permutation of $(1,2,3)$).
Thus $\colsp(W_{mmm})=\spanop\{z_m\}$ and $\rank(W_{mmm})=1$ provided $\rank(A^{(m)})=3$.

\medskip\noindent
(2) The condition $\rank(W_{\beta\gamma\delta})=4$ is the nonvanishing of some $4\times 4$ minor of the $4\times 27$ matrix
$W_{\beta\gamma\delta}$, hence is Zariski-open. To show it is nonempty, it suffices to exhibit one assignment of the relevant
matrices for which $\rank(W_{\beta\gamma\delta})=4$. There are two essential patterns up to permuting $\beta,\gamma,\delta$.

\smallskip\noindent
\emph{Distinct cameras.} Take three matrices
\[
A^{(\beta)}=
\begin{bmatrix} e_1^T\\ e_2^T\\ e_3^T\end{bmatrix},\quad
A^{(\gamma)}=
\begin{bmatrix} e_1^T\\ e_2^T\\ e_4^T\end{bmatrix},\quad
A^{(\delta)}=
\begin{bmatrix} e_1^T\\ e_3^T\\ e_4^T\end{bmatrix},
\]
where $e_1,\dots,e_4$ is the standard basis of $\R^4$.
Then among the columns of $W_{\beta\gamma\delta}$ appear the four vectors
\[
\star(e_1\wedge e_2\wedge e_3)=-e_4,\qquad
\star(e_1\wedge e_2\wedge e_4)=e_3,\qquad
\star(e_1\wedge e_4\wedge e_3)=e_2,\qquad
\star(e_3\wedge e_2\wedge e_4)=-e_1,
\]
which form a basis of $\R^4$. (For example, using \eqref{eq:hodge},
$\det[e_4;e_1;e_2;e_3]=-1$, hence $e_4\cdot \star(e_1\wedge e_2\wedge e_3)=-1$, so $\star(e_1\wedge e_2\wedge e_3)=-e_4$;
the remaining identities are checked similarly.) Therefore $\rank(W_{\beta\gamma\delta})=4$ in this instance.

\smallskip\noindent
\emph{Two equal indices.} Consider $(\beta,\gamma,\delta)=(m,m,m')$ with $m\ne m'$.
Take
\[
A^{(m)}=
\begin{bmatrix} e_1^T\\ e_2^T\\ e_3^T\end{bmatrix},\quad
A^{(m')}=
\begin{bmatrix} e_3^T\\ e_4^T\\ e_1^T\end{bmatrix}.
\]
Then among the columns of $W_{mm m'}$ appear
\[
\star(e_1\wedge e_2\wedge e_3)=-e_4,\qquad
\star(e_1\wedge e_2\wedge e_4)=e_3,\qquad
\star(e_1\wedge e_3\wedge e_4)=-e_2,\qquad
\star(e_2\wedge e_3\wedge e_4)=e_1,
\]
again a basis of $\R^4$, so $\rank(W_{mm m'})=4$ in this instance.

\smallskip
In either pattern we have produced a point at which $\rank(W_{\beta\gamma\delta})=4$, so the rank-$4$ locus is nonempty.
Hence, by Zariski-openness, $\rank(W_{\beta\gamma\delta})=4$ holds for Zariski-generic cameras whenever $(\beta,\gamma,\delta)$
are not all equal.
\end{proof}

\subsection{A single generic open set}

For each triple $(\beta,\gamma,\delta)$ not all equal, let $U^{\mathrm W}_{\beta\gamma\delta}$ be the Zariski-open set on which $\rank(W_{\beta\gamma\delta})=4$.
Define
\[
U_n^{\mathrm W}:=\bigcap_{\substack{(\beta,\gamma,\delta)\in[n]^3\\ \text{not all equal}}}U^{\mathrm W}_{\beta\gamma\delta},
\qquad
U_n:=U_n^{\mathrm{ker}}\cap U_n^{\mathrm W}.
\]
As $(\R^{3\times 4})^n$ is irreducible (its coordinate ring is a domain), any finite intersection of nonempty Zariski-open subsets is nonempty.
Thus $U_n$ is nonempty, Zariski-open, and dense.

From now on we assume $(A^{(1)},\dots,A^{(n)})\in U_n$.

\section{Main theorem and proof}\label{sec:main}

\begin{theorem}\label{thm:main}
Fix $n\ge 5$ and let $A^{(1)},\dots,A^{(n)}\in \R^{3\times 4}$ be Zariski-generic.
Let $\lambda$ satisfy \eqref{eq:lambdaSupport} and let $\mathcal Z=\lambda\odotop \mathcal Q$ be the stacked scaled tensor.
Let $\mathbf F_n$ be as in \S\ref{sec:F}. Then:
\begin{enumerate}
\item $\mathbf F_n$ does not depend on the cameras.
\item Every coordinate of $\mathbf F_n$ has degree $5$, independent of $n$.
\item One has $\mathbf F_n(\mathcal Z)=0$ if and only if there exist $u,v,w,x\in(\R^\ast)^n$ such that
\[
\lambda_{\alpha\beta\gamma\delta}=u_\alpha v_\beta w_\gamma x_\delta
\qquad\text{for all }(\alpha,\beta,\gamma,\delta)\neq(m,m,m,m).
\]
\end{enumerate}
\end{theorem}

\subsection{The ``if'' direction}

\begin{proof}[Proof of Theorem \ref{thm:main}, ``if'' direction]
Items (1) and (2) follow from Proposition \ref{prop:F12}.

Assume $\lambda_{\alpha\beta\gamma\delta}=u_\alpha v_\beta w_\gamma x_\delta$ for all off-diagonal quadruples, with $u,v,w,x\in(\R^\ast)^n$.
Let $\mathcal Z=\lambda\odotop \mathcal Q$.
In the mode-$1$ flattening, the $(\alpha;\beta,\gamma,\delta)$ block of $M_1(\mathcal Z)$ equals
$u_\alpha (v_\beta w_\gamma x_\delta)$ times the corresponding block of $M_1(\mathcal Q)$, by \eqref{eq:blockM1Z}.
Thus $M_1(\mathcal Z)=D_1\,M_1(\mathcal Q)\,E_1$ for suitable invertible diagonal matrices $D_1,E_1$, hence
$\rank(M_1(\mathcal Z))=\rank(M_1(\mathcal Q))\le 4$ by Corollary \ref{cor:rankBoundQ}.
Therefore all $5\times 5$ minors of $M_1(\mathcal Z)$ vanish.

The same argument applies to $M_2,M_3,M_4$ (each flattening block-scales by products of $u,v,w,x$ along its row and column blocks), so all coordinates of $\mathbf F_n(\mathcal Z)$ vanish.
\end{proof}

\subsection{Mode-wise rank constraints force separability}

The remaining direction uses the special structure \eqref{eq:blockM1Z} and genericity.

\begin{proposition}\label{prop:mode1Separation}
Assume $(A^{(1)},\dots,A^{(n)})\in U_n$ and $\lambda$ satisfies \eqref{eq:lambdaSupport}.
If $\rank(M_1(\mathcal Z))\le 4$, then there exist $u\in(\R^\ast)^n$ and $s\in(\R^\ast)^{n\times n\times n}$ such that
\begin{equation}\label{eq:mode1Sep}
\lambda_{\alpha\beta\gamma\delta}=u_\alpha\,s_{\beta\gamma\delta}
\qquad\text{for all off-diagonal }(\alpha,\beta,\gamma,\delta).
\end{equation}
\end{proposition}

\begin{proof}
Assume $\rank(M_1(\mathcal Z))\le 4$. Then there exist matrices
$C\in\R^{3n\times 4}$ and $V\in\R^{4\times 27n^3}$ such that $M_1(\mathcal Z)=CV$.
Partition $C$ into blocks $C_\alpha\in\R^{3\times 4}$ and $V$ into blocks $V_{\beta\gamma\delta}\in\R^{4\times 27}$.
Comparing with \eqref{eq:blockM1Z}, for all $\alpha,\beta,\gamma,\delta$,
\begin{equation}\label{eq:blockEq}
C_\alpha V_{\beta\gamma\delta}
=\lambda_{\alpha\beta\gamma\delta}\,A^{(\alpha)}W_{\beta\gamma\delta}.
\end{equation}

Choose a reference triple $(\beta_0,\gamma_0,\delta_0)$ with three distinct indices (possible since $n\ge 5$).
Then $\rank(W_{\beta_0\gamma_0\delta_0})=4$ by Lemma \ref{lem:rankW}, so there is a right inverse
$W_{\beta_0\gamma_0\delta_0}^+\in\R^{27\times 4}$ with $W_{\beta_0\gamma_0\delta_0}W_{\beta_0\gamma_0\delta_0}^+=I_4$.
Let
\[
M:=V_{\beta_0\gamma_0\delta_0}W_{\beta_0\gamma_0\delta_0}^+\in\R^{4\times 4}.
\]
Multiplying \eqref{eq:blockEq} on the right by $W_{\beta_0\gamma_0\delta_0}^+$ yields
\begin{equation}\label{eq:CalphaM}
C_\alpha M=\lambda_{\alpha\beta_0\gamma_0\delta_0}\,A^{(\alpha)}.
\end{equation}
Since $(\beta_0,\gamma_0,\delta_0)$ are distinct, $(\alpha,\beta_0,\gamma_0,\delta_0)$ is never diagonal, hence
$\lambda_{\alpha\beta_0\gamma_0\delta_0}\neq 0$ for all $\alpha$ by \eqref{eq:lambdaSupport}.

\smallskip\noindent
\emph{Claim:} $M$ is invertible.
If $0\neq y\in\ker(M)$, then \eqref{eq:CalphaM} gives
$0=C_\alpha My=\lambda_{\alpha\beta_0\gamma_0\delta_0}A^{(\alpha)}y$, hence $A^{(\alpha)}y=0$ for all $\alpha$.
Thus $y\in\bigcap_\alpha \ker(A^{(\alpha)})$.
But for cameras in $U_n$, the kernel lines are $\ker(A^{(\alpha)})=\spanop\{z_\alpha\}$ and are pairwise distinct
(Lemma \ref{lem:genericKernels}); the intersection of two distinct lines is $\{0\}$, so the intersection over all $\alpha$ is $\{0\}$.
Contradiction. Hence $M\in \mathrm{GL}_4(\R)$.

Solving \eqref{eq:CalphaM} gives
\begin{equation}\label{eq:Csolve}
C_\alpha=\lambda_{\alpha\beta_0\gamma_0\delta_0}\,A^{(\alpha)}M^{-1}.
\end{equation}
Substitute \eqref{eq:Csolve} into \eqref{eq:blockEq}, divide by $\lambda_{\alpha\beta_0\gamma_0\delta_0}$, and set
\[
c_{\alpha;\beta\gamma\delta}:=\frac{\lambda_{\alpha\beta\gamma\delta}}{\lambda_{\alpha\beta_0\gamma_0\delta_0}}
\qquad\big((\alpha,\beta,\gamma,\delta)\ \text{off-diagonal}\big),
\]
to obtain
\begin{equation}\label{eq:AalphaConstraint}
A^{(\alpha)}\Big(M^{-1}V_{\beta\gamma\delta}-c_{\alpha;\beta\gamma\delta}\,W_{\beta\gamma\delta}\Big)=0.
\end{equation}
Since $\ker(A^{(\alpha)})=\spanop\{z_\alpha\}$ is one-dimensional, each column of the bracketed $4\times 27$ matrix lies in $\spanop\{z_\alpha\}$.
Hence there exists a row vector $r_{\alpha;\beta\gamma\delta}^\top\in\R^{1\times 27}$ such that
\begin{equation}\label{eq:rank1Correction}
M^{-1}V_{\beta\gamma\delta}-c_{\alpha;\beta\gamma\delta}\,W_{\beta\gamma\delta}
=z_\alpha r_{\alpha;\beta\gamma\delta}^\top.
\end{equation}
Fix $(\beta,\gamma,\delta)$ and subtract \eqref{eq:rank1Correction} for two distinct cameras $\alpha_1\ne\alpha_2$:
\begin{equation}\label{eq:diffEq}
(c_{\alpha_1;\beta\gamma\delta}-c_{\alpha_2;\beta\gamma\delta})W_{\beta\gamma\delta}
=z_{\alpha_2}r_{\alpha_2;\beta\gamma\delta}^\top - z_{\alpha_1}r_{\alpha_1;\beta\gamma\delta}^\top.
\end{equation}
The right-hand side has column space contained in $\spanop\{z_{\alpha_1},z_{\alpha_2}\}$.

If $(\beta,\gamma,\delta)$ are not all equal, then $\rank(W_{\beta\gamma\delta})=4$ by Lemma \ref{lem:rankW}. If the scalar on the left were nonzero,
the left-hand side would have rank $4$, contradicting the rank $\le 2$ of the right-hand side. Hence
$c_{\alpha;\beta\gamma\delta}$ is independent of $\alpha$.

If $(\beta,\gamma,\delta)=(m,m,m)$, then $\colsp(W_{mmm})=\spanop\{z_m\}$ by Lemma \ref{lem:rankW}.
Choose $\alpha_1,\alpha_2\in[n]\setminus\{m\}$ distinct (possible since $n\ge 5$). By Lemma \ref{lem:genericKernels}(3),
$z_m,z_{\alpha_1},z_{\alpha_2}$ are linearly independent, so
$\spanop\{z_m\}\cap \spanop\{z_{\alpha_1},z_{\alpha_2}\}=\{0\}$.
In \eqref{eq:diffEq}, the left-hand side has column space in $\spanop\{z_m\}$ and the right-hand side has column space in $\spanop\{z_{\alpha_1},z_{\alpha_2}\}$,
so both sides must be zero, and thus $c_{\alpha_1;mmm}=c_{\alpha_2;mmm}$.
Therefore $c_{\alpha;mmm}$ is independent of $\alpha\ne m$, which is exactly the range where $(\alpha,m,m,m)$ is off-diagonal.

We conclude: for each triple $(\beta,\gamma,\delta)$ there exists $s_{\beta\gamma\delta}\in\R^\ast$ such that
$c_{\alpha;\beta\gamma\delta}=s_{\beta\gamma\delta}$ for all $\alpha$ with $(\alpha,\beta,\gamma,\delta)$ off-diagonal.
Setting $u_\alpha:=\lambda_{\alpha\beta_0\gamma_0\delta_0}$ gives \eqref{eq:mode1Sep}.
\end{proof}

\begin{proposition}\label{prop:mode2mode3Separation}
Assume $(A^{(1)},\dots,A^{(n)})\in U_n$ and $\lambda$ satisfies \eqref{eq:lambdaSupport}.
\begin{enumerate}
\item If $\rank(M_2(\mathcal Z))\le 4$, then there exist $v\in(\R^\ast)^n$ and $t\in(\R^\ast)^{n\times n\times n}$ such that
$\lambda_{\alpha\beta\gamma\delta}=v_\beta\,t_{\alpha\gamma\delta}$ for all off-diagonal $(\alpha,\beta,\gamma,\delta)$.
\item If $\rank(M_3(\mathcal Z))\le 4$, then there exist $w\in(\R^\ast)^n$ and $p\in(\R^\ast)^{n\times n\times n}$ such that
$\lambda_{\alpha\beta\gamma\delta}=w_\gamma\,p_{\alpha\beta\delta}$ for all off-diagonal $(\alpha,\beta,\gamma,\delta)$.
\item If $\rank(M_4(\mathcal Z))\le 4$, then there exist $x\in(\R^\ast)^n$ and $q\in(\R^\ast)^{n\times n\times n}$ such that
$\lambda_{\alpha\beta\gamma\delta}=x_\delta\,q_{\alpha\beta\gamma}$ for all off-diagonal $(\alpha,\beta,\gamma,\delta)$.
\end{enumerate}
\end{proposition}

\begin{proof}
The argument is the same as Proposition \ref{prop:mode1Separation} after permuting the roles of the four indices in \eqref{eq:defQ}.
Concretely, for mode $2$, one uses the identity
\[
\det[a_{\alpha i};a_{\beta j};a_{\gamma k};a_{\delta\ell}]
= -\,a_{\beta j}\cdot \star(a_{\alpha i}\wedge a_{\gamma k}\wedge a_{\delta\ell}),
\]
obtained by swapping the first two rows, and similarly for modes $3$ and $4$ (with the appropriate sign).
These signs only multiply entire block-columns by $\pm 1$ and therefore do not affect the rank arguments. The generic rank statements required
for the corresponding block matrices reduce to Lemma \ref{lem:rankW}.
\end{proof}

\begin{lemma}\label{lem:combine12}
Assume \eqref{eq:lambdaSupport}. Suppose for all off-diagonal quadruples
\[
\lambda_{\alpha\beta\gamma\delta}=u_\alpha s_{\beta\gamma\delta}
\quad\text{and}\quad
\lambda_{\alpha\beta\gamma\delta}=v_\beta t_{\alpha\gamma\delta},
\]
with $u,v\in(\R^\ast)^n$. Then there exists $r\in(\R^\ast)^{n\times n}$ such that
\[
\lambda_{\alpha\beta\gamma\delta}=u_\alpha v_\beta r_{\gamma\delta}
\quad\text{for all off-diagonal }(\alpha,\beta,\gamma,\delta).
\]
\end{lemma}

\begin{proof}
Fix $(\gamma,\delta)\in[n]^2$. Choose $\alpha_0\in[n]\setminus\{\gamma,\delta\}$ (possible since $n\ge 5$).
Then for every $\beta$ the quadruple $(\alpha_0,\beta,\gamma,\delta)$ is off-diagonal, so
\[
u_{\alpha_0}s_{\beta\gamma\delta}=\lambda_{\alpha_0\beta\gamma\delta}=v_\beta t_{\alpha_0\gamma\delta}.
\]
Define $r_{\gamma\delta}:=t_{\alpha_0\gamma\delta}/u_{\alpha_0}\in\R^\ast$ to obtain $s_{\beta\gamma\delta}=v_\beta r_{\gamma\delta}$ for all $\beta$.
Hence $\lambda_{\alpha\beta\gamma\delta}=u_\alpha v_\beta r_{\gamma\delta}$ on the off-diagonal.
\end{proof}

\begin{lemma}\label{lem:combine123}
Assume \eqref{eq:lambdaSupport}. Suppose for all off-diagonal quadruples
\[
\lambda_{\alpha\beta\gamma\delta}=u_\alpha v_\beta r_{\gamma\delta}
\quad\text{and}\quad
\lambda_{\alpha\beta\gamma\delta}=w_\gamma p_{\alpha\beta\delta},
\]
with $u,v,w\in(\R^\ast)^n$. Then there exists $x\in(\R^\ast)^n$ such that
\[
\lambda_{\alpha\beta\gamma\delta}=u_\alpha v_\beta w_\gamma x_\delta
\quad\text{for all off-diagonal }(\alpha,\beta,\gamma,\delta).
\]
\end{lemma}

\begin{proof}
Choose $\alpha_0\neq \beta_0$ (possible since $n\ge 5$). Then for all $(\gamma,\delta)$ the quadruple $(\alpha_0,\beta_0,\gamma,\delta)$ is off-diagonal, so
\[
u_{\alpha_0}v_{\beta_0}r_{\gamma\delta}=\lambda_{\alpha_0\beta_0\gamma\delta}=w_\gamma p_{\alpha_0\beta_0\delta}.
\]
Define
\[
x_\delta:=\frac{p_{\alpha_0\beta_0\delta}}{u_{\alpha_0}v_{\beta_0}}\in\R^\ast.
\]
Then $r_{\gamma\delta}=w_\gamma x_\delta$ for all $\gamma,\delta$, and substituting into $\lambda_{\alpha\beta\gamma\delta}=u_\alpha v_\beta r_{\gamma\delta}$
gives the desired factorization.
\end{proof}

\begin{proof}[Proof of Theorem \ref{thm:main}, ``only if'' direction]
Assume $\mathbf F_n(\mathcal Z)=0$. By Lemma \ref{lem:rankDet}, all $5\times 5$ minors of each flattening vanish, hence
$\rank(M_r(\mathcal Z))\le 4$ for $r=1,2,3,4$.

Apply Proposition \ref{prop:mode1Separation} to obtain $\lambda_{\alpha\beta\gamma\delta}=u_\alpha s_{\beta\gamma\delta}$ on the off-diagonal.
Apply Proposition \ref{prop:mode2mode3Separation}(1) to obtain $\lambda_{\alpha\beta\gamma\delta}=v_\beta t_{\alpha\gamma\delta}$ on the off-diagonal.
Then Lemma \ref{lem:combine12} yields $\lambda_{\alpha\beta\gamma\delta}=u_\alpha v_\beta r_{\gamma\delta}$ on the off-diagonal.
Apply Proposition \ref{prop:mode2mode3Separation}(2) to obtain $\lambda_{\alpha\beta\gamma\delta}=w_\gamma p_{\alpha\beta\delta}$ on the off-diagonal.
Then Lemma \ref{lem:combine123} yields $\lambda_{\alpha\beta\gamma\delta}=u_\alpha v_\beta w_\gamma x_\delta$ on the off-diagonal.

Finally, $u,v,w,x$ lie in $(\R^\ast)^n$ because each is defined via off-diagonal entries of $\lambda$, which are nonzero by \eqref{eq:lambdaSupport}.
\end{proof}

\section{Remarks}\label{sec:remarks}

\begin{remark}[Diagonal entries of $\lambda$]
Since $Q^{(mmmm)}\equiv 0$, the diagonal values $\lambda_{mmmm}$ do not affect $\mathcal Z$.
Theorem \ref{thm:main} therefore asserts and proves factorization only on the off-diagonal, which is exactly the identifiable part.
\end{remark}

\begin{remark}[On the hypothesis $n\ge 5$]
The argument in fact requires only:
(i) existence of three distinct indices to form a reference triple; and
(ii) for each $m$, existence of two indices distinct from $m$ to handle the $(m,m,m)$ triple in Proposition \ref{prop:mode1Separation};
together with the generic condition that any three kernel vectors $z_\alpha$ are independent.
Thus, after minor bookkeeping, the proof works for $n\ge 3$. We retain $n\ge 5$ to match the problem statement.
\end{remark}

\begin{remark}[Degree]
The degree $5$ is dictated by the ambient dimension $4$: the relevant flattenings have rank $\le 4$ precisely when all $5\times 5$ minors vanish.
\end{remark}

\end{document}
