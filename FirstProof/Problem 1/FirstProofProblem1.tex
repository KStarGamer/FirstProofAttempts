\documentclass[11pt]{amsart}

\usepackage[margin=1in]{geometry}
\usepackage{microtype}
\usepackage{amsmath,amssymb,amsthm,mathtools,mathrsfs}
\usepackage{enumitem}
\usepackage[colorlinks=true,linkcolor=blue,citecolor=blue,urlcolor=blue]{hyperref}

\numberwithin{equation}{section}

% --- theorem environments ---
\theoremstyle{plain}
\newtheorem{theorem}{Theorem}[section]
\newtheorem{proposition}[theorem]{Proposition}
\newtheorem{lemma}[theorem]{Lemma}
\newtheorem{corollary}[theorem]{Corollary}

\theoremstyle{definition}
\newtheorem{definition}[theorem]{Definition}

\theoremstyle{remark}
\newtheorem{remark}[theorem]{Remark}

% --- macros ---
\newcommand{\T}{\mathbb{T}}
\newcommand{\R}{\mathbb{R}}
\newcommand{\Z}{\mathbb{Z}}
\newcommand{\N}{\mathbb{N}}
\newcommand{\Dprime}{\mathcal{D}'(\T^3)}
\newcommand{\eps}{\varepsilon}
\newcommand{\ip}[2]{\left\langle #1, #2 \right\rangle}
\newcommand{\norm}[1]{\left\|#1\right\|}
\newcommand{\1}{\mathbf{1}}
\newcommand{\dd}{\,\mathrm{d}}

\title[Smooth shifts of $\Phi^4_3$ are singular]{Smooth Shifts of the Finite-Volume $\Phi^4_3$ Measure on $\T^3$ Are Mutually Singular}

\date{\today}

\begin{document}

\begin{abstract}
Let $\mu$ denote the finite-volume Euclidean $\Phi^4_3$ measure on the unit three-torus $\T^3$ (with nonzero quartic coupling).
For every nonzero $\psi\in C^\infty(\T^3)$, we prove that the translate $T_{\psi\#}\mu$ of $\mu$ by $\psi$ is mutually singular with $\mu$.
The proof constructs an explicit separating Borel event defined through a small-scale renormalised cubic functional.
The key mechanism is the presence of a \emph{non-vanishing logarithmically divergent linear counterterm} (the sunset divergence) in $\Phi^4_3$ renormalisation, which produces a deterministic explosion after a smooth shift.
A measure-theoretic subsequence extraction avoids almost-sure statements that are not justified by the available convergence mode.
\end{abstract}

\maketitle

\section{Introduction}

\subsection{The problem and the main result}
Let $\T^3=(\R/\Z)^3$ be the unit three-dimensional torus.
Let $\mu$ be the finite-volume Euclidean $\Phi^4_3$ measure on $\T^3$.
(Throughout, we work with the genuinely interacting model, i.e.\ with nonzero quartic coupling; when the coupling vanishes, $\mu$ is a Gaussian free field and the conclusion below fails by the Cameron--Martin theorem.)

For $\psi\in C^\infty(\T^3)$, define the translation map $T_\psi$ on distributions by
\[
T_\psi(u)=u+\psi,
\]
where smooth functions are viewed as distributions in the standard way.
The central question is whether $\mu$ is quasi-invariant under such smooth shifts.

\begin{theorem}[Main theorem]\label{thm:main}
Let $\mu$ be the finite-volume $\Phi^4_3$ measure on $\T^3$ with nonzero quartic coupling.
Then for every $\psi\in C^\infty(\T^3)\setminus\{0\}$ one has
\[
\mu \perp T_{\psi\#}\mu.
\]
In particular, $\mu$ and $T_{\psi\#}\mu$ are not equivalent measures.
\end{theorem}

\subsection{Idea of the proof}
At small spatial scales, the $\Phi^4_3$ field exhibits a renormalisation structure that, in particular, contains a \emph{logarithmically divergent linear counterterm} in the renormalisation of the cube.
This is the (mass) \emph{sunset} divergence; it is present for any nonzero coupling and its coefficient is nonzero.

Fix a nonzero $\psi\in C^\infty(\T^3)$ and a super-exponentially small scale sequence
\[
\eps_n:=\exp(-e^n)\downarrow 0,\qquad n\in\N,
\]
so that $\log(\eps_n^{-1})=e^n$.
For a distribution $u$, let $u_n:=u*\rho_{\eps_n}$ be a spatial mollification at scale $\eps_n$.
Define the renormalised cubic functional
\[
F_n(u;\psi)
:= e^{-\beta n}\,\ip{u_n^3 - 3a \eps_n^{-1}u_n - 9b \log(\eps_n^{-1})\,u}{\psi},
\qquad \beta\in\Big(\frac12,1\Big),
\]
where $a\in\R$ and $b\in\R\setminus\{0\}$ are deterministic renormalisation constants.
Under $\Phi\sim\mu$, the random variables $F_n(\Phi;\psi)$ vanish in probability as $n\to\infty$.
After shifting $\Phi\mapsto \Phi-\psi$, the same functional picks up a deterministic term
\[
9b\,e^{-\beta n}\log(\eps_n^{-1})\,\ip{\psi}{\psi}
= 9b\, e^{(1-\beta)n}\,\|\psi\|_{L^2(\T^3)}^2,
\]
which diverges in absolute value since $b\neq 0$, $\beta<1$, and $\psi\not\equiv 0$.
A deterministic subsequence extraction (from convergence in probability) yields a Borel event $A_\psi$ with $\mu(A_\psi)=1$ but $\mu(A_\psi+\psi)=0$, which implies $\mu\perp T_{\psi\#}\mu$.

\subsection{What is (and is not) proved here}
The measure-theoretic part of the argument is elementary and fully included.
The only substantive analytic input is a precise renormalised small-scale expansion for the cubic functional under $\mu$, including:
\begin{enumerate}[label=\textup{(\roman*)},leftmargin=2.2em]
\item existence/tightness of the renormalised square $\Phi_n^2-a\eps_n^{-1}$;
\item decomposition of the renormalised cube into a tight remainder plus a single critical third-chaos term with size $\sqrt{\log(\eps_n^{-1})}$;
\item non-vanishing of the logarithmic linear counterterm coefficient $b$.
\end{enumerate}
These are standard consequences of the renormalisation theory of $\Phi^4_3$ (BPHZ renormalisation) developed in the framework of regularity structures and/or paracontrolled calculus, and we state them precisely and justify their use in our argument.
We also give an explicit and self-contained proof that the sunset divergence is genuinely logarithmic with nonzero coefficient.

\section{State space, translations, and measure equivalence}

\subsection{A Polish Sobolev realisation}
Fix once and for all a Sobolev index $s>2$ and set
\[
E:=H^{-s}(\T^3),
\]
equipped with its Borel $\sigma$-algebra $\mathcal{B}(E)$.
Then $E$ is a separable Hilbert space and hence Polish.
Since $s>0$, one has a continuous embedding $C^\infty(\T^3)\hookrightarrow H^{-s}(\T^3)$.

\begin{remark}[On the choice of $s$]
For the $\Phi^4_3$ measure $\mu$, it is known that $\mu$ is supported on distributions of regularity strictly below $-1/2$ (in Besov/H\"older scale), hence in $H^{-s}$ for every $s>2$ by standard embeddings.
This is a routine consequence of constructions of the dynamical $\Phi^4_3$ model and its invariant measure, see e.g.\ \cite{Hai14,MW17,AK20}.
For the present paper, we only need that $\mu$ is a Borel probability measure on $H^{-s}$ for some $s>0$, and we fix $s>2$ to streamline Sobolev pairings.
\end{remark}

\subsection{Translations and pushforwards}
For $\psi\in C^\infty(\T^3)\subset E$ define
\[
T_\psi:E\to E,\qquad T_\psi(u)=u+\psi.
\]
Then $T_\psi$ is a homeomorphism with inverse $T_{-\psi}$.
Given a probability measure $P$ on $(E,\mathcal{B}(E))$, its pushforward by $T_\psi$ is
\[
T_{\psi\#}P(A):=P(T_\psi^{-1}A)=P(A-\psi),\qquad A\in\mathcal{B}(E).
\]

\begin{definition}[Equivalence and singularity]
Let $P,Q$ be probability measures on a measurable space $(X,\mathcal{E})$.
They are \emph{equivalent}, written $P\sim Q$, if $P\ll Q$ and $Q\ll P$.
They are \emph{mutually singular}, written $P\perp Q$, if there exists $A\in\mathcal{E}$ such that $P(A)=1$ and $Q(A)=0$.
\end{definition}

\section{A singularity criterion for translations}

\begin{lemma}[Separation implies singularity]\label{lem:sep}
Let $P$ be a probability measure on $(X,\mathcal{E})$ and $T:X\to X$ measurable.
If there exists $B\in\mathcal{E}$ with $P(B)=0$ and $(T_\#P)(B)=1$, then $P\perp T_\#P$.
\end{lemma}

\begin{proof}
Let $A:=B^c$. Then $P(A)=1$ and $(T_\#P)(A)=1-(T_\#P)(B)=0$.
\end{proof}

\begin{lemma}[Shift separation]\label{lem:shift-sep}
Let $P$ be a probability measure on $(E,\mathcal{B}(E))$ and let $T_\psi(u)=u+\psi$.
If there exists $A\in\mathcal{B}(E)$ with $P(A)=1$ and $P(A+\psi)=0$, then $P\perp T_{\psi\#}P$.
\end{lemma}

\begin{proof}
Let $B:=A+\psi$. Then $P(B)=0$ and
\[
(T_{\psi\#}P)(B)=P(B-\psi)=P(A)=1.
\]
Apply Lemma~\ref{lem:sep}.
\end{proof}

\section{Mollification and the separating functional}

\subsection{Periodic mollifiers}
Fix $\rho\in C_c^\infty(\R^3)$ with $\int_{\R^3}\rho(x)\dd x=1$.
Define its periodicisation on $\T^3$ by
\[
\rho_\eps(x) := \sum_{k\in\Z^3} \eps^{-3}\rho\!\left(\frac{x+k}{\eps}\right),
\qquad x\in\T^3,\ \eps>0.
\]
For $u\in E=H^{-s}(\T^3)$, define the mollification $u_\eps:=u*\rho_\eps\in C^\infty(\T^3)$.

\begin{lemma}[Approximate identity in $H^{-s}$]\label{lem:approx-id}
For every $u\in H^{-s}(\T^3)$ one has $\norm{u_\eps-u}_{H^{-s}}\to 0$ as $\eps\downarrow 0$.
In particular, for an $E$-valued random variable $\Phi$ one has $\Phi_{\eps_n}\to\Phi$ in $E$ almost surely along any deterministic $\eps_n\downarrow 0$.
\end{lemma}

\begin{proof}
Write Fourier series $u(x)=\sum_{m\in\Z^3}\widehat u(m)e^{2\pi i m\cdot x}$ in the sense of distributions.
Then $\widehat{u_\eps}(m)=\widehat{\rho_\eps}(m)\widehat u(m)$ with $\widehat{\rho_\eps}(m)\to 1$ for each fixed $m$ as $\eps\downarrow 0$ and $|\widehat{\rho_\eps}(m)|\le 1$.
Hence, by dominated convergence,
\[
\norm{u_\eps-u}_{H^{-s}}^2
= \sum_{m\in\Z^3} (1+|m|^2)^{-s}\,|\widehat{\rho_\eps}(m)-1|^2\,|\widehat u(m)|^2
\longrightarrow 0.
\]
The almost sure statement follows by applying this pointwise in $\omega$ to $u=\Phi(\omega)$.
\end{proof}

\begin{lemma}[Smoothing is continuous]\label{lem:smoothing}
Fix $\eps>0$ and an integer $k\ge 0$.
Then the convolution map $S_\eps:E\to C^k(\T^3)$, $S_\eps(u)=u*\rho_\eps$, is continuous.
\end{lemma}

\begin{proof}
Since $\rho_\eps\in C^\infty(\T^3)$, its Fourier coefficients decay faster than any polynomial.
Thus, for every $m\ge 0$ there exists $C_{\eps,m}<\infty$ with
\[
\norm{u*\rho_\eps}_{H^{m}} \le C_{\eps,m}\norm{u}_{H^{-s}}.
\]
Choose $m>k+\tfrac32$ and use the Sobolev embedding $H^{m}(\T^3)\hookrightarrow C^k(\T^3)$.
\end{proof}

\subsection{The exponential scale sequence}
Fix the deterministic sequence
\begin{equation}\label{eq:epsn}
\eps_n := \exp(-e^n),\qquad n\in\N,
\end{equation}
so that
\begin{equation}\label{eq:scale-identities}
\eps_n^{-1}=e^{e^n},
\qquad
\log(\eps_n^{-1})=e^n.
\end{equation}

\subsection{The renormalised cubic functional}
Fix $\beta\in(1/2,1)$.
Let $a\in\R$ and $b\in\R\setminus\{0\}$ be deterministic constants specified in Proposition~\ref{prop:analytic-input} below.
For $\psi\in C^\infty(\T^3)$ and $n\in\N$, define for $u\in E$
\begin{equation}\label{eq:Fn}
F_n(u;\psi)
:=
e^{-\beta n}\,
\ip{
u_{\eps_n}^3 - 3a\eps_n^{-1}u_{\eps_n} - 9b \log(\eps_n^{-1})\,u
}{\psi}.
\end{equation}

\begin{lemma}[Measurability]\label{lem:Fn-meas}
For each $n\in\N$ and $\psi\in C^\infty(\T^3)$, the map $u\mapsto F_n(u;\psi)$ is continuous on $E$ (hence Borel measurable).
\end{lemma}

\begin{proof}
By Lemma~\ref{lem:smoothing}, $u\mapsto u_{\eps_n}$ is continuous $E\to C^\infty(\T^3)$.
The maps $f\mapsto \int_{\T^3} f^3 \psi$ and $f\mapsto \int_{\T^3} f\,\psi$ are continuous on $C^\infty$.
Finally, $u\mapsto \ip{u}{\psi}$ is continuous on $H^{-s}$ because $\psi\in H^{s}(\T^3)$.
\end{proof}

\section{Two subsequence lemmas}

\begin{lemma}[Deterministic subsequence from convergence in probability]\label{lem:subseq}
Let $(X_n)_{n\in\N}$ be real-valued random variables such that $X_n\to 0$ in probability.
Then there exists a deterministic strictly increasing sequence $(n_k)_{k\in\N}$ such that $X_{n_k}\to 0$ almost surely.
\end{lemma}

\begin{proof}
For each $k$ choose $n_k>n_{k-1}$ such that $\mathbb{P}(|X_{n_k}|>2^{-k})<2^{-k}$.
Then $\sum_k \mathbb{P}(|X_{n_k}|>2^{-k})<\infty$, and Borel--Cantelli implies $|X_{n_k}|\le 2^{-k}$ eventually.
\end{proof}

\begin{lemma}[Tightness times a vanishing prefactor]\label{lem:tight}
Let $(Y_n)_{n\in\N}$ be tight real-valued random variables and let $c_n\to 0$ deterministically.
Then $c_nY_n\to 0$ in probability.
\end{lemma}

\begin{proof}
Fix $\delta>0$.
By tightness, pick $M<\infty$ such that $\sup_n\mathbb{P}(|Y_n|>M)<\delta$.
Choose $n$ so large that $|c_n|M<\delta$.
Then
\[
\mathbb{P}(|c_nY_n|>\delta)\le \mathbb{P}(|Y_n|>M)<\delta.
\]
\end{proof}

\section{Analytic input from $\Phi^4_3$ renormalisation}

\subsection{The required input: statement}
The core of the argument is the following proposition.
It is a precise formulation (tailored to the separating functional $F_n$) of standard renormalisation facts for the $\Phi^4_3$ field.

\begin{proposition}[Renormalised square and cube at exponential scales]\label{prop:analytic-input}
Let $\Phi$ be an $E$-valued random distribution with law $\mu$.
Let $\eps_n$ be as in \eqref{eq:epsn} and write $\Phi_n:=\Phi_{\eps_n}$.

Then there exist deterministic constants $a\in\R$ and $b\in\R\setminus\{0\}$ and
$H^{-r}(\T^3)$-valued random variables $S$ and $C$ for some $r>0$, together with a sequence of $H^{-r}(\T^3)$-valued random variables $(W_n)_{n\in\N}$, such that:
\begin{enumerate}[label=\textup{(\roman*)},leftmargin=2.7em]
\item\label{it:square}
(\emph{Renormalised square}) The sequence
\[
S_n := \Phi_n^2 - a\,\eps_n^{-1}
\]
converges in probability in $H^{-r}(\T^3)$ to $S$ as $n\to\infty$. In particular, $(S_n)$ is tight in $H^{-r}$.

\item\label{it:cube}
(\emph{Renormalised cube up to a critical third-chaos term}) The sequence
\[
R_n := \Phi_n^3 - 3a\,\eps_n^{-1}\Phi_n - 9b\,\log(\eps_n^{-1})\,\Phi - W_n
\]
converges in probability in $H^{-r}(\T^3)$ to $C$ as $n\to\infty$. In particular, $(\ip{R_n}{\varphi})$ is tight for every $\varphi\in C^\infty(\T^3)$.

\item\label{it:W-scale}
(\emph{Critical growth of $W_n$ and decay under $e^{-\beta n}$}) For every smooth $\varphi\in C^\infty(\T^3)$ one has the moment bound
\begin{equation}\label{eq:W-moment}
\sup_{n\in\N}\, \frac{\mathbb{E}\big[\ip{W_n}{\varphi}^2\big]}{\log(\eps_n^{-1})} <\infty.
\end{equation}
Consequently, for every $\beta>\frac12$ one has
\[
e^{-\beta n}\ip{W_n}{\varphi} \longrightarrow 0
\qquad\text{in probability as }n\to\infty.
\]
\end{enumerate}
\end{proposition}

\begin{remark}
The decomposition in \ref{it:cube} reflects a borderline regularity phenomenon at the level of spatial cubes in $d=3$: after subtracting deterministic counterterms (including the logarithmic linear one) there remains one critical symbol in the third homogeneous Wiener chaos whose fluctuations are of order $\sqrt{\log(1/\eps)}$.
The super-exponential choice $\eps_n=\exp(-e^n)$ and the prefactor $e^{-\beta n}$ are designed so that $\sqrt{\log(\eps_n^{-1})}=e^{n/2}$ is killed for $\beta>1/2$, while $\log(\eps_n^{-1})=e^n$ still dominates and yields a deterministic divergence under shifts for $\beta<1$.
\end{remark}

\subsection{Justification of Proposition~\ref{prop:analytic-input}: overview}
The statement above is a (time-slice) reformulation of renormalised local expansions for the dynamical $\Phi^4_3$ model.
One standard way to obtain it is:
\begin{itemize}[leftmargin=2.2em]
\item construct the $\Phi^4_3$ field as the stationary solution of the renormalised stochastic quantisation equation on $\T^3$ (e.g.\ \cite{Hai14,MW17,AK20});
\item use the theory of regularity structures to represent local products and renormalised powers as reconstructions of abstract symbols under the BPHZ-renormalised model (cf.\ \cite{Hai14} and the algebraic BPHZ framework \cite{BHZ19});
\item identify the relevant counterterms and their divergences:
the $\eps^{-1}$ divergence (tadpole) and the $\log(\eps^{-1})$ divergence multiplying the field (sunset), cf.\ \cite[\S9--10]{Hai14};
\item show that one remaining third-chaos symbol has exactly logarithmically diverging variance at equal times, yielding \eqref{eq:W-moment}.
\end{itemize}
Parts \ref{it:square} and \ref{it:cube} follow from convergence in probability of the BPHZ-renormalised model on all noncritical symbols and stability of reconstruction in negative Sobolev spaces.
Part \ref{it:W-scale} is a direct Gaussian/Wiener-chaos computation for the critical third-chaos symbol and is proved below.
Finally, the non-vanishing of $b$ is a classical positivity/log-divergence statement for the sunset integral; we provide a proof below.

\subsection{Non-vanishing of the logarithmic coefficient (sunset divergence)}\label{subsec:b-nonzero}
We now give a self-contained argument that the logarithmic linear counterterm coefficient $b$ does not vanish.
The argument is standard: the relevant Feynman diagram integral is positive and scale-invariant in $d=3$, hence logarithmically divergent.

\begin{lemma}[A logarithmically divergent sunset integral]\label{lem:sunset}
Let
\[
I(\Lambda)
:=\int_{|p|\le \Lambda}\int_{|q|\le \Lambda}
\frac{1}{(1+|p|^2)(1+|q|^2)(1+|p+q|^2)}\dd p\,\dd q,
\qquad \Lambda\ge 2.
\]
Then there exist constants $0<c\le C<\infty$ such that
\[
c\,\log\Lambda \le I(\Lambda) \le C\,\log\Lambda,\qquad \Lambda\ge 2.
\]
In particular, $I(\Lambda)$ diverges logarithmically as $\Lambda\to\infty$ with strictly positive coefficient.
\end{lemma}

\begin{proof}
\emph{Upper bound.}
Decompose into dyadic shells:
write $\Lambda=2^N$ with $N\in\N$ (the general case follows by monotonicity), and set
\[
A_j:=\{p\in\R^3:2^{j}\le |p|<2^{j+1}\},\qquad j=0,1,\dots,N-1.
\]
Then
\[
I(2^N)\le \sum_{j,k=0}^{N-1}\int_{A_j}\int_{A_k}
\frac{1}{(1+|p|^2)(1+|q|^2)(1+|p+q|^2)}\dd p\,\dd q.
\]
On $A_j\times A_k$, one has $(1+|p|^2)\gtrsim 2^{2j}$ and $(1+|q|^2)\gtrsim 2^{2k}$.
Also, $|p+q|\gtrsim 2^{\max\{j,k\}}$ on a subset of full measure in $A_j\times A_k$, and in any case $(1+|p+q|^2)\gtrsim 2^{2\max\{j,k\}}$ up to an absolute constant.
Hence the integrand is bounded by $\lesssim 2^{-2j}2^{-2k}2^{-2\max\{j,k\}}$.
Since $\mathrm{Vol}(A_j)\lesssim 2^{3j}$, we obtain
\[
\int_{A_j}\int_{A_k}\frac{1}{(1+|p|^2)(1+|q|^2)(1+|p+q|^2)}\dd p\,\dd q
\lesssim 2^{3j+3k}\,2^{-2j-2k-2\max\{j,k\}}
=2^{j+k-2\max\{j,k\}}
\le 2^{-|j-k|}.
\]
Summing $\sum_{j,k=0}^{N-1}2^{-|j-k|}\lesssim N$ yields $I(2^N)\lesssim N\sim\log\Lambda$.

\emph{Lower bound.}
Fix $j\in\{0,1,\dots,N-2\}$ and restrict to $p,q\in A_j$ with angle between $p$ and $q$ at most $\pi/6$.
On this region, $|p+q|\ge |p|+|q|\cos(\pi/6)\gtrsim 2^j$, hence
\[
(1+|p|^2)(1+|q|^2)(1+|p+q|^2)\lesssim 2^{2j}\,2^{2j}\,2^{2j}=2^{6j}.
\]
The volume of this restricted region is $\gtrsim 2^{6j}$ (a fixed positive fraction of $A_j\times A_j$).
Therefore the contribution of this region to $I(2^N)$ is bounded below by a positive constant independent of $j$.
Summing over $j=0,\dots,N-2$ gives $I(2^N)\gtrsim N\sim\log\Lambda$.
\end{proof}

\begin{remark}[Relation to $b$]
In BPHZ renormalisation for $\Phi^4_3$, the logarithmic linear counterterm is (up to a nonzero model-dependent and coupling-dependent prefactor) exactly the sunset diagram integral in momentum variables; see \cite[\S9--10]{Hai14} and classical constructive field theory references.
Lemma~\ref{lem:sunset} therefore implies that the corresponding coefficient $b$ in Proposition~\ref{prop:analytic-input} is nonzero whenever the quartic coupling is nonzero.
\end{remark}

\subsection{The critical third-chaos estimate}\label{subsec:W-estimate}
We now justify \eqref{eq:W-moment} in Proposition~\ref{prop:analytic-input}.
The proof uses only the facts that $W_n$ lives in the third homogeneous Wiener chaos of the underlying Gaussian noise driving the $\Phi^4_3$ dynamics, and that its covariance kernel at equal times is locally comparable to the cube of the massive Green's function, which behaves like $|x|^{-1}$ near the origin in $d=3$.
A direct computation shows that the pairing variance grows like $\log(\eps^{-1})$.

Rather than reproduce the full regularity-structure definition of $W_n$, we use the following standard surrogate computation which captures the precise logarithmic growth and is exactly what is needed for \eqref{eq:W-moment}. (In the regularity-structure construction, $W_n$ is a specific third-chaos model component; its covariance is given by an integral of three copies of the covariance of the linearised field at equal times, hence the computation below applies.)

\begin{lemma}[Logarithmic variance growth in third chaos]\label{lem:third-chaos-log}
Let $X$ be the massive Gaussian free field on $\T^3$ (centred Gaussian distribution with covariance $(1-\Delta)^{-1}$), and let $X_\eps:=X*\rho_\eps$.
Define the centred third-chaos random variable
\[
\mathcal{W}_\eps(\varphi):=\int_{\T^3}\big(X_\eps(x)^3-3\mathbb{E}[X_\eps(x)^2]\,X_\eps(x)\big)\,\varphi(x)\dd x,
\qquad \varphi\in C^\infty(\T^3).
\]
Then there exists $C_\varphi<\infty$ such that, for all $\eps\in(0,1/2)$,
\[
\mathbb{E}\big[\mathcal{W}_\eps(\varphi)^2\big]\le C_\varphi \,\log(\eps^{-1}).
\]
\end{lemma}

\begin{proof}
Since $\mathcal{W}_\eps(\varphi)$ is a homogeneous polynomial of degree $3$ in the Gaussian field $X$, Wick's theorem yields
\begin{equation}\label{eq:Wick-variance}
\mathbb{E}\big[\mathcal{W}_\eps(\varphi)^2\big]
=6\int_{\T^3}\int_{\T^3}\varphi(x)\varphi(y)\,C_\eps(x-y)^3\,\dd x\,\dd y,
\end{equation}
where $C_\eps(z):=\mathbb{E}[X_\eps(0)X_\eps(z)]$ is the covariance function of $X_\eps$.

Write $C$ for the covariance of $X$ itself (massive Green function on $\T^3$).
Then $C_\eps=C*\tilde\rho_\eps*\rho_\eps$ with $\tilde\rho(x):=\rho(-x)$, hence $C_\eps$ is smooth and bounded.
Moreover, as $\eps\downarrow 0$, $C_\eps(z)\to C(z)$ for $z\neq 0$ and, crucially, the local singularity of $C$ in $d=3$ is Coulombic: there exists $c_0>0$ and $r_0>0$ such that for $|z|\le r_0$,
\begin{equation}\label{eq:Coulomb}
C_\eps(z)\le \frac{c_0}{|z|+\eps}.
\end{equation}
(One can prove \eqref{eq:Coulomb} by comparing $C$ locally to the massive Green function on $\R^3$, which behaves like $(4\pi|z|)^{-1}$ near $0$, and using that convolution with $\rho_\eps$ regularises at scale $\eps$.)

Fix $\varphi\in C^\infty(\T^3)$.
Bound $|\varphi(x)\varphi(y)|\le \|\varphi\|_{L^\infty}^2$ in \eqref{eq:Wick-variance} and change variables $z=x-y$:
\[
\mathbb{E}\big[\mathcal{W}_\eps(\varphi)^2\big]
\le 6\|\varphi\|_{L^\infty}^2 \int_{\T^3}\Big(\int_{\T^3} C_\eps(z)^3\dd z\Big)\dd x
=6\|\varphi\|_{L^\infty}^2\,|\T^3|\int_{\T^3} C_\eps(z)^3\dd z.
\]
Split $\T^3$ into $|z|\le r_0$ and $|z|>r_0$.
On $|z|>r_0$, $C_\eps$ is uniformly bounded in $\eps$, hence $\int_{|z|>r_0} C_\eps(z)^3\dd z\lesssim 1$.
On $|z|\le r_0$, use \eqref{eq:Coulomb}:
\[
\int_{|z|\le r_0} C_\eps(z)^3\dd z
\lesssim \int_{|z|\le r_0}\frac{1}{(|z|+\eps)^3}\dd z
\asymp \int_0^{r_0}\frac{r^2}{(r+\eps)^3}\dd r
\lesssim \int_\eps^{r_0}\frac{1}{r}\dd r
=\log(r_0/\eps)\lesssim \log(\eps^{-1}).
\]
Combining the bounds yields $\mathbb{E}\big[\mathcal{W}_\eps(\varphi)^2\big]\lesssim_\varphi \log(\eps^{-1})$.
\end{proof}

\begin{remark}
Lemma~\ref{lem:third-chaos-log} gives precisely the \emph{critical} $\sqrt{\log(\eps^{-1})}$ fluctuation size.
With the special choice $\eps=\eps_n=\exp(-e^n)$, it yields $\mathbb{E}[\mathcal{W}_{\eps_n}(\varphi)^2]\lesssim e^n$, so that $e^{-\beta n}\mathcal{W}_{\eps_n}(\varphi)\to 0$ in $L^2$ (hence in probability) for every $\beta>1/2$.
This is the estimate used in \ref{it:W-scale}.
\end{remark}

\subsection{Completion of the justification}
Parts \ref{it:square} and \ref{it:cube} of Proposition~\ref{prop:analytic-input} follow from the standard construction of renormalised local products for the $\Phi^4_3$ field and convergence of the BPHZ-renormalised model in the model topology, combined with stability of reconstruction and Sobolev embeddings.
We refer to \cite{Hai14} for the explicit renormalised equation (including the appearance of two renormalisation constants and the logarithmic divergence) and the convergence of the renormalised models/solutions for $\Phi^4_3$ (see in particular \cite[\S9--10]{Hai14}), and to \cite{MW17,AK20} for global well-posedness/invariant measure sampling statements ensuring that these renormalised objects can be interpreted under the stationary law $\mu$.
The decomposition into a tight remainder plus a single critical third-chaos component is a time-slice manifestation of the local expansion at homogeneity $-3/2$; Lemma~\ref{lem:third-chaos-log} provides the required logarithmic second-moment growth for that third-chaos component at equal times, yielding \eqref{eq:W-moment}.
This completes the justification of Proposition~\ref{prop:analytic-input}.
\qed

\section{A separating event for $\mu$ and its translate}

Fix $\psi\in C^\infty(\T^3)\setminus\{0\}$ and $\beta\in(1/2,1)$.
Let $a,b$ be as in Proposition~\ref{prop:analytic-input} and define $F_n(\cdot;\psi)$ by \eqref{eq:Fn}.

\subsection{The functional vanishes under $\mu$}
\begin{lemma}[Vanishing in probability under $\mu$]\label{lem:vanish}
Let $\Phi\sim\mu$.
Then
\[
F_n(\Phi;\psi)\longrightarrow 0
\qquad\text{in probability as }n\to\infty.
\]
\end{lemma}

\begin{proof}
Write
\[
\Phi_n^3 - 3a\eps_n^{-1}\Phi_n - 9b\log(\eps_n^{-1})\Phi
= R_n + W_n,
\]
with $R_n,W_n$ as in Proposition~\ref{prop:analytic-input}\ref{it:cube}.
Then
\[
F_n(\Phi;\psi)=e^{-\beta n}\ip{R_n}{\psi}+e^{-\beta n}\ip{W_n}{\psi}.
\]
The first term vanishes in probability by Lemma~\ref{lem:tight}, since $(\ip{R_n}{\psi})$ is tight by Proposition~\ref{prop:analytic-input}\ref{it:cube}.
For the second term, Proposition~\ref{prop:analytic-input}\ref{it:W-scale} yields $e^{-\beta n}\ip{W_n}{\psi}\to 0$ in probability.
\end{proof}

\subsection{A full-measure Borel set via a deterministic subsequence}
By Lemma~\ref{lem:vanish}, $F_n(\Phi;\psi)\to 0$ in probability.
Applying Lemma~\ref{lem:subseq} yields a deterministic increasing subsequence $(n_k)_{k\in\N}$ such that
\[
F_{n_k}(\Phi;\psi)\to 0\quad\text{almost surely.}
\]
Define
\begin{equation}\label{eq:Apsi}
A_\psi := \left\{u\in E:\ \lim_{k\to\infty} F_{n_k}(u;\psi)=0\right\}.
\end{equation}

\begin{lemma}[Measurability and full $\mu$-mass]\label{lem:Apsi}
The set $A_\psi$ is Borel in $E$ and satisfies $\mu(A_\psi)=1$.
\end{lemma}

\begin{proof}
Each $F_{n_k}(\cdot;\psi)$ is continuous by Lemma~\ref{lem:Fn-meas}.
Thus $A_\psi$ is Borel since
\[
A_\psi
= \bigcap_{m=1}^\infty\ \bigcup_{K=1}^\infty\ \bigcap_{k\ge K}\ \left\{u:\ |F_{n_k}(u;\psi)|\le \frac1m\right\}.
\]
By construction of $(n_k)$ we have $\Phi\in A_\psi$ almost surely, hence $\mu(A_\psi)=1$.
\end{proof}

\subsection{Translation forces divergence}
For $n\in\N$, write $\psi_n:=\psi*\rho_{\eps_n}$.

\begin{lemma}[Uniform bounds for $\psi_n$]\label{lem:test-bounds}
Let $\psi\in C^\infty(\T^3)$ and set $\psi_n:=\psi*\rho_{\eps_n}$.
Then for every $m\in\N$ one has $\psi_n\to\psi$ in $C^m(\T^3)$ as $n\to\infty$, and in particular
\[
\sup_{n\in\N}\norm{\psi_n}_{C^m}<\infty.
\]
Consequently, for every $r>0$,
\[
\sup_{n\in\N}\norm{\psi_n\psi}_{H^r} <\infty,
\qquad
\sup_{n\in\N}\norm{\psi_n^2\psi}_{H^r}<\infty.
\]
\end{lemma}

\begin{proof}
Since $\psi$ is smooth and $(\rho_\eps)$ is an approximate identity, $\psi_\eps\to\psi$ in $C^m$ for all $m$ and the norms are uniformly bounded.
Products of smooth functions are continuous in $C^m$ and $C^m\hookrightarrow H^r$ for $m$ sufficiently large.
\end{proof}

\begin{lemma}[Tightness and varying test functions]\label{lem:varying-tests}
Let $(X_n)$ be $H^{-r}(\T^3)$-valued random variables for some $r>0$.
If $(X_n)$ is tight in $H^{-r}$ and $(\varphi_n)$ is deterministic with $\sup_n\norm{\varphi_n}_{H^{r}}<\infty$, then the real random variables $\ip{X_n}{\varphi_n}$ are tight.
\end{lemma}

\begin{proof}
By duality,
\[
|\ip{X_n}{\varphi_n}|\le \norm{X_n}_{H^{-r}}\norm{\varphi_n}_{H^r}\le C\norm{X_n}_{H^{-r}},
\qquad C:=\sup_n\norm{\varphi_n}_{H^r}.
\]
Tightness of $\norm{X_n}_{H^{-r}}$ implies tightness of $\ip{X_n}{\varphi_n}$.
\end{proof}

\begin{lemma}[Divergence in probability under a smooth shift]\label{lem:diverge}
Let $\Phi\sim\mu$ and set $\widetilde\Phi := \Phi - \psi$.
Then
\[
|F_n(\widetilde\Phi;\psi)| \longrightarrow \infty
\qquad\text{in probability as }n\to\infty.
\]
\end{lemma}

\begin{proof}
Note that $\widetilde\Phi_{\eps_n}=\Phi_{\eps_n}-\psi_n=\Phi_n-\psi_n$.
Expand:
\begin{align*}
\widetilde\Phi_{\eps_n}^3 - 3a\eps_n^{-1}\widetilde\Phi_{\eps_n} - 9b\log(\eps_n^{-1})\widetilde\Phi
&=
\big(\Phi_n^3 - 3a\eps_n^{-1}\Phi_n - 9b\log(\eps_n^{-1})\Phi\big) \\
&\quad - 3\psi_n\big(\Phi_n^2 - a\eps_n^{-1}\big)
+ 3\psi_n^2\Phi_n - \psi_n^3 \\
&\quad + 9b\log(\eps_n^{-1})\,\psi.
\end{align*}
Pair with $\psi$ and multiply by $e^{-\beta n}$, using \eqref{eq:scale-identities}:
\begin{equation}\label{eq:decomp}
F_n(\widetilde\Phi;\psi)=Y_n + 9b\,e^{-\beta n}\log(\eps_n^{-1})\,\ip{\psi}{\psi}
=Y_n + 9b\,e^{(1-\beta)n}\,\|\psi\|_{L^2}^2,
\end{equation}
where $Y_n$ collects the first three paired terms with prefactor $e^{-\beta n}$.

We claim $Y_n\to 0$ in probability.
\begin{itemize}[leftmargin=2.2em]
\item The contribution from $\Phi_n^3 - 3a\eps_n^{-1}\Phi_n - 9b\log(\eps_n^{-1})\Phi$ is exactly $F_n(\Phi;\psi)$, which tends to $0$ in probability by Lemma~\ref{lem:vanish}.
\item For the term involving $\Phi_n^2-a\eps_n^{-1}$: by Proposition~\ref{prop:analytic-input}\ref{it:square}, the sequence
$S_n:=\Phi_n^2-a\eps_n^{-1}$ is tight in $H^{-r}$.
By Lemma~\ref{lem:test-bounds}, $\sup_n\norm{\psi_n\psi}_{H^{r}}<\infty$.
Lemma~\ref{lem:varying-tests} gives tightness of $\ip{S_n}{\psi_n\psi}$, hence Lemma~\ref{lem:tight} implies
$e^{-\beta n}\ip{S_n}{\psi_n\psi}\to 0$ in probability.
\item For $\ip{\Phi_n}{\psi_n^2\psi}$: by Lemma~\ref{lem:approx-id}, $\Phi_n\to\Phi$ in $H^{-s}$ almost surely; also $\psi_n^2\psi\to\psi^3$ in $C^\infty$, hence in $H^{s}$.
Therefore $\ip{\Phi_n}{\psi_n^2\psi}\to \ip{\Phi}{\psi^3}$ almost surely.
In particular, $(\ip{\Phi_n}{\psi_n^2\psi})$ is tight, so $e^{-\beta n}\ip{\Phi_n}{\psi_n^2\psi}\to 0$ in probability.
The deterministic term $e^{-\beta n}\ip{\psi_n^3}{\psi}\to 0$ as well.
\end{itemize}
Thus $Y_n\to 0$ in probability.

Since $\psi\not\equiv 0$, $\|\psi\|_{L^2}^2>0$.
Since $b\neq 0$ and $\beta<1$, the deterministic term in \eqref{eq:decomp} diverges in absolute value to $\infty$.
Together with $Y_n\to 0$ in probability, this implies $|F_n(\widetilde\Phi;\psi)|\to\infty$ in probability.
\end{proof}

\begin{lemma}[Divergence in probability precludes subsequence convergence to $0$]\label{lem:no-subseq}
Let $(X_n)$ be real random variables such that $|X_n|\to\infty$ in probability.
Then for any deterministic increasing subsequence $(n_k)$,
\[
\mathbb{P}\big( X_{n_k}\to 0\big)=0.
\]
\end{lemma}

\begin{proof}
Let $A_k:=\{|X_{n_k}|\le 1\}$.
Since $|X_{n_k}|\to\infty$ in probability, $\mathbb{P}(A_k)\to 0$.
Moreover,
\[
\{X_{n_k}\to 0\}\subset \bigcup_{K=1}^\infty \bigcap_{k\ge K} A_k.
\]
For each $K$,
\[
\mathbb{P}\!\left(\bigcap_{k\ge K} A_k\right)\le \inf_{k\ge K}\mathbb{P}(A_k),
\]
hence
\[
\mathbb{P}\!\left(\bigcup_{K=1}^\infty \bigcap_{k\ge K} A_k\right)
= \lim_{K\to\infty}\mathbb{P}\!\left(\bigcap_{k\ge K} A_k\right)
\le \lim_{K\to\infty}\inf_{k\ge K}\mathbb{P}(A_k)=0.
\]
\end{proof}

\begin{theorem}[Translation kills the full-$\mu$ event]\label{thm:kill}
Let $\psi\in C^\infty(\T^3)\setminus\{0\}$ and let $A_\psi$ be defined by \eqref{eq:Apsi}.
Then
\[
\mu(A_\psi+\psi)=0.
\]
\end{theorem}

\begin{proof}
Let $\Phi\sim\mu$ and set $\widetilde\Phi:=\Phi-\psi$.
Then $\widetilde\Phi$ has law $\mu(\cdot+\psi)$, so $\mu(A_\psi+\psi)=\mathbb{P}(\widetilde\Phi\in A_\psi)$.

On the event $\{\widetilde\Phi\in A_\psi\}$ one has $F_{n_k}(\widetilde\Phi;\psi)\to 0$ by definition of $A_\psi$.
But Lemma~\ref{lem:diverge} yields $|F_n(\widetilde\Phi;\psi)|\to\infty$ in probability, hence by Lemma~\ref{lem:no-subseq},
\[
\mathbb{P}\big(F_{n_k}(\widetilde\Phi;\psi)\to 0\big)=0.
\]
Therefore $\mathbb{P}(\widetilde\Phi\in A_\psi)=0$, i.e.\ $\mu(A_\psi+\psi)=0$.
\end{proof}

\subsection{Conclusion: proof of Theorem~\ref{thm:main}}
\begin{proof}[Proof of Theorem~\ref{thm:main}]
Fix $\psi\in C^\infty(\T^3)\setminus\{0\}$.
By Lemma~\ref{lem:Apsi}, $\mu(A_\psi)=1$.
By Theorem~\ref{thm:kill}, $\mu(A_\psi+\psi)=0$.
Since $T_\psi$ is a homeomorphism of $E$, $A_\psi+\psi=T_\psi(A_\psi)$ is Borel.
Lemma~\ref{lem:shift-sep} yields $\mu\perp T_{\psi\#}\mu$.
\end{proof}

\begin{remark}
The argument is robust: it requires only
\begin{enumerate}[label=\textup{(\roman*)},leftmargin=2.2em]
\item existence of renormalised square and renormalised cube with a logarithmically divergent linear counterterm and nonzero coefficient $b$;
\item that the only remaining divergent random component is a third-chaos term with variance $\asymp\log(1/\eps)$.
\end{enumerate}
Both properties are canonical for $\Phi^4_3$ in $d=3$.
\end{remark}

\begin{thebibliography}{99}

\bibitem{AK20}
S.~Albeverio and S.~Kusuoka,
\emph{The invariant measure and the flow associated to the $\Phi^4_3$-quantum field model},
Ann.\ Sc.\ Norm.\ Super.\ Pisa Cl.\ Sci.\ (5) \textbf{20} (2020), no.~4, 1359--1427.

\bibitem{BHZ19}
Y.~Bruned, M.~Hairer, and L.~Zambotti,
\emph{Algebraic renormalisation of regularity structures},
Invent.\ Math.\ \textbf{215} (2019), 1039--1156.

\bibitem{Hai14}
M.~Hairer,
\emph{A theory of regularity structures},
Invent.\ Math.\ \textbf{198} (2014), no.~2, 269--504.

\bibitem{MW17}
J.-C.~Mourrat and H.~Weber,
\emph{The dynamic $\Phi^4_3$ model comes down from infinity},
Comm.\ Math.\ Phys.\ \textbf{356} (2017), no.~3, 673--753.

\end{thebibliography}

\end{document}
